\chapter{PN结的电流特性}
在本章中,我们将
\begin{itemize}
    \item 探讨施加正偏电压时,PN结势垒降低,进而空穴和电子流过PN结产生电流的过程。
    \item 推导N区过剩空穴和P区过剩电子的边界条件,并分析正偏下这些过剩载流子的行为。
    \item 推导正偏PN结二极管的理想电流--电压关系。
    \item 说明并分析PN结二极管的非理想效应,包括:大注入、生成电流、复合电流。
    \item 讨论PN结二极管小信号等效电路。
    \item 讨论PN结二极管大信号开关特性。
    \item 说明隧道二极管这种专用二极管。
\end{itemize}
在上一章中,我们讨论了零偏和反偏状态下PN结的静电特性,确定了内建电势差,计算了空间电荷区的电场和电势,还考虑了PN结的势垒电容。在本章我们主要讨论正偏PN结,确定PN结的电流--电压特性,求出PN结的小信号模型,计算PN结的扩散电阻和扩散电容。

\section{PN结电流}
我们已经熟知,当PN结上加正偏电压时,在PN结内就会产生电流。在本节,我们先定性讨论PN结内的电荷是如何流动的,随后,再考虑PN结电压--电流定量关系的数学推导。

\subsection{PN结内电荷流动的定性描述}
通过\xref{fig:PN结能带图}所示的能带图,我们可以定性地理解PN结电流的机理。\xref{fig:零偏PN结}展示了热平衡状态下的PN结能带图。我们指出,电子遇到的势垒阻碍了N区大量电子流入P区,而完全类似的,空穴遇到的势垒同样阻碍了P区大量空穴流入N区\footnote{空穴能带的正负方向与电子相反,而我们看到的能带图都是从电子视角绘制的。}。由此,势垒维持了热平衡。

那为何正偏时PN结中会产生电流?\xref{fig:正偏PN结}展现了PN结正偏时的能带图。由于正偏时,外加电压削弱了势垒,而更小的势垒意味着耗尽区电场减小,进而导致其无法再阻止P区空穴和N区电子。因此将产生“P区至N区的电子扩散流”和“N区至P区的空穴扩散流”。这些载流子的扩散就在PN结中形成了一个由P区至N区的电流,其方向与电压方向一致。

\begin{Figure}[PN结能带图]
    \begin{FigureSub}[零偏PN结]
        \includegraphics[width=0.45\linewidth]{build/Chapter02A_01.fig.pdf}
    \end{FigureSub}
    \hspace{0.05\linewidth}
    \begin{FigureSub}[正偏PN结]
        \includegraphics[width=0.45\linewidth]{build/Chapter02A_02.fig.pdf}
    \end{FigureSub}
\end{Figure}

注入N区的电子是N区少子,注入P区的电子是P区少子。而少数载流子的行为可以通过在半导体物理中已经学习过的\uwave{连续性方程}(Continous Equations)描述,在这些区域中,过剩载流子将发生漂移--扩散--复合行为,形成电流。而PN结的电压--电流关系将在下一节讨论。

\begin{BoxEquation}[连续性方程]
    连续性方程描述了半导体中少数载流子的时空分布规律。

    对于空穴,这是
    \begin{Equation}
        D_p\pdv[2]{(\fdd{p})}{x}-\mu_p\qty[\E\pdv{(\fdd{p})}{x}+p\pdv{\E}{x}]+g_p'-\frac{\fdd{p}}{\tau_{p0}}=\pdv{(\fdd{p})}{t}
    \end{Equation}
    对于电子,这是
    \begin{Equation}
        D_n\pdv[2]{(\fdd{n})}{x}-\mu_n\qty[\E\pdv{(\fdd{n})}{x}+n\pdv{\E}{x}]+g_n'-\frac{\fdd{n}}{\tau_{n0}}=\pdv{(\fdd{n})}{t}
    \end{Equation}
\end{BoxEquation}

\subsection{PN结的理想假设}
PN结的理想电流--电压关系是在以下四个假设的基础上推导的
\begin{enumerate}
    \item 耗尽区适用突变边界近似,耗尽区之外的半导体视为中性区域,换言之,不存在漂移。
    \item 适用麦克斯韦--玻尔兹曼载流子统计近似,即半导体是非简并的。
    \item 适用小注入和完全电离。
    \item PN结总电流为常量,电子电流与空穴电流,在耗尽区内为常量,在耗尽区外连续。
\end{enumerate}

\subsection{PN结的边界条件}
我们已经知道,PN结的\uwave{内建电势}(Built-in Potential Barrier)$V_{bi}$由以下公式决定。
\begin{BoxFormula}[PN结的内建电势]
    PN结的内建电势$V_{bi}$由以下公式决定
    \begin{Equation}
        V_{bi}=\frac{\kB T}{e}\ln(\frac{N_aN_d}{n_i^2})
    \end{Equation}
\end{BoxFormula}
现在要解决的两个问题是
\begin{enumerate}
    \item PN结在平衡状态下,电子和空穴在其多子侧和少子侧分别为常量,且是已知的。这很显然,电子$n_{n0}=N_d, n_{p0}=n_i^2/N_a$,空穴$p_{p0}=N_a, p_{n0}=n_i^2/N_d$,该结果实际与PN结没有任何关系,仅来自P区和N区作为独立的杂质半导体的特性。但是,这并不妨碍我们将试图用PN结内建电势$V_{bi}$来描述$n_{n0},n_{p0}$以及$p_{p0},p_{n0}$,即平衡载流子多子侧浓度和少子侧浓度间的关系,因为依照\xref{fml:PN结的内建电势},本质上$V_{bi}$仍然是关于$N_d,N_a,n_i^2$的。
    \item PN结在正偏状态下,势垒将被削弱,电子和空穴将从其多子侧注入少子侧,这意味着载流子在少子侧不再是$n_{p0}$或$p_{n0}$的常量,而是随$x$变化的$n_{p},p_{n}$的变量。在本小节关注的是$n_{p},p_{n}$的边界条件,即耗尽区边界处$n_p(-x_p),p_n(+x_n)$与$n_{p0}, p_{n0}$的关系是什么?
    % \item PN结在正偏状态下,势垒将被削弱,电子和空穴将从其多子侧注入少子侧,这意味着载流子在少子侧不再是$n_{p0}$或$p_{n0}$的常量,而是随$x$变化的$n_{p},p_{n}$的变量。正如小节标题所写,这里我们关注的是$n_{p},p_{n}$的边界条件。很明显,当$x\to\pm\infty$时,注入的影响将趋于零,少子浓度将回归其平衡浓度,即$n_p(-\infty)=n_{p0}$和$p_n(+\infty)=p_{n0}$,这就是无穷远处的边界条件。那么,由于势垒减弱导致的电注入(具体而言,势垒减弱后,载流子将从多子侧穿过耗尽区到达少子侧边界,这增加了少子侧边界处载流子浓度),在耗尽区边界处的$n_p(-x_p)$和$p_n(+x_n)$与$n_{p0}, p_{n0}$的关系是什么?相较于$n_{p0}, p_{n0}$增加了多少?
\end{enumerate}

\begin{BoxFormula}[PN结的平衡少子多子比]
    PN结在平衡态时,同种载流子在少子侧和多子侧的浓度比,可用$V_{bi}$表示为
    \begin{Equation}
        n_{p0}=n_{n0}\exp(-\frac{eV_{bi}}{\kB T})\qquad
        p_{n0}=p_{p0}\exp(-\frac{eV_{bi}}{\kB T})
    \end{Equation}
\end{BoxFormula}

\begin{Proof}
    根据\fancyref{fml:PN结的内建电势}
    \begin{Equation}&[1]
        V_{bi}=\frac{\kB T}{e}\ln(\frac{N_aN_d}{n_i^2})
    \end{Equation}
    稍作变换
    \begin{Equation}&[2]
        \frac{N_aN_d}{n_i^2}=\exp(\frac{eV_{bi}}{\kB T})
    \end{Equation}
    或
    \begin{Equation}&[3]
        \frac{n_i^2}{N_aN_d}=\exp(-\frac{eV_{bi}}{\kB T})
    \end{Equation}
    我们已知
    \begin{Equation}&[4]
        \begin{cases}
            n_{n0}=N_d\\
            n_{p0}=n_i^2/N_a
        \end{cases}\qquad
        \begin{cases}
            p_{p0}=N_a\\
            p_{n0}=n_i^2/N_d
        \end{cases}
    \end{Equation}
    将\xrefpeq{4}分别代入\xrefpeq{3}中,分别得到
    \begin{Equation}
        \frac{n_{p0}}{n_{n0}}=\exp(-\frac{eV_{bi}}{\kB T})\qquad
        \frac{p_{n0}}{p_{p0}}=\exp(-\frac{eV_{bi}}{\kB T})
    \end{Equation}
    即
    \begin{Equation}*
        n_{p0}=n_{n0}\exp(-\frac{eV_{bi}}{\kB T})\qquad
        p_{n0}=p_{p0}\exp(-\frac{eV_{bi}}{\kB T})\qedhere
    \end{Equation}
\end{Proof}

\begin{BoxFormula}[PN结的边界条件]
    PN结在正偏时,当外加正向电压为$V_a$时,少子浓度在耗尽区边界满足
    \begin{Equation}&[A]
        n_p(-x_p)=n_{p0}\exp(\frac{eV_a}{\kB T})\qquad
        p_n(+x_n)=p_{n0}\exp(\frac{eV_a}{\kB T})
    \end{Equation}
\end{BoxFormula}

\begin{Proof}
    依据\fancyref{fml:PN结的平衡少子多子比}
    \begin{Equation}&[1]
        n_{p0}=n_{n0}\exp(-\frac{eV_{bi}}{\kB T})\qquad
        p_{n0}=p_{p0}\exp(-\frac{eV_{bi}}{\kB T})
    \end{Equation}
    这是平衡态的结果,在正偏时,势垒$V_{bi}$减小至$V_{bi}-V_a$,即
    \begin{Equation}&[2]
        \qquad\qquad
        n_{p}(-x_p)=n_{n0}\exp[-\frac{e(V_{bi}-V_a)}{\kB T}]\qquad
        p_{n}(+x_n)=p_{p0}\exp[-\frac{e(V_{bi}-V_a)}{\kB T}]
        \qquad\qquad
    \end{Equation}
    稍作整理
    \begin{Equation}&[3]
        n_{p}(-x_p)=n_{n0}\exp(-\frac{eV_{bi}}{\kB T})\exp(\frac{V_a}{\kB T})\qquad
        p_{n}(+x_n)=p_{p0}\exp(-\frac{eV_{bi}}{\kB T})\exp(\frac{V_a}{\kB T})
    \end{Equation}
    将\xrefpeq{1}代入\xrefpeq{3}
    \begin{Equation}*
        n_p(-x_p)=n_{p0}\exp(\frac{eV_a}{\kB T})\qquad
        p_n(+x_n)=p_{n0}\exp(\frac{eV_a}{\kB T})
        \qedhere
    \end{Equation}
\end{Proof}

\subsection{PN结的载流子分布}
在本节,我们将在\xref{fml:PN结的边界条件}的基础上,进一步推出PN结中少数载流子的浓度分布。

\begin{BoxFormula}[PN结的载流子分布]
    PN结在正偏时,少子浓度分别满足
    \begin{Align}[20pt]
        \fdd{p_n(x)}=p_{n0}&\qty[\exp(\frac{eV_a}{\kB T})-1]\exp(\frac{x_n-x}{L_p})\qquad x\geq +x_n\\
        \fdd{n_p(x)}=n_{p0}&\qty[\exp(\frac{eV_a}{\kB T})-1]\exp(\frac{x_p+x}{L_n})\qquad x\leq -x_p
    \end{Align}
\end{BoxFormula}

\begin{Proof}
    根据\fancyref{eqt:连续性方程},空穴的连续性方程满足
    \begin{Equation}&[1]
        D_p\pdv[2]{(\fdd{p_n})}{x}-\mu_p\qty[\E\pdv{(\fdd{p_n})}{x}+p\pdv{\E}{x}]+g_p'-\frac{\fdd{p_n}}{\tau_{p0}}=\pdv{(\fdd{p_n})}{t}
    \end{Equation}
    依照\xref{subsec:PN结的理想假设}的假设,耗尽区外无电场$\E=0$且$g_p'=0$,若再假设稳态$\pdv*{(\fdd{p_n})}{t}=0$
    \begin{Equation}&[2]
        D_p\dv[2]{(\fdd{p_n})}{x}-\frac{\fdd{p_n}}{\tau_{p0}}=0\qquad x>+x_n
    \end{Equation}
    两端同除$D_p$,引入扩散长度$L_p=\sqrt{D_p\tau_{p0}}$
    \begin{Equation}&[3]
        \dv[2]{(\fdd{p_n})}{x}-\frac{\fdd{p_n}}{L_p^2}=0\qquad x>+x_n
    \end{Equation}
    类似亦可得到电子的情形
    \begin{Equation}&[4]
        \dv[2]{(\fdd{n_p})}{x}-\frac{\fdd{n_p}}{L_n^2}=0\qquad x<-x_p
    \end{Equation}
    \xrefpeq{3}的通解为
    \begin{Equation}&[5]
        \fdd{p_n(x)}=P_1\exp(\frac{x}{L_p})+P_2\exp(-\frac{x}{L_p})\qquad x\geq +x_n
    \end{Equation}
    \xrefpeq{4}的通解为
    \begin{Equation}&[6]
        \fdd{n_p(x)}=N_1\exp(\frac{x}{L_n})+N_2\exp(-\frac{x}{L_n})\qquad x\leq -x_p
    \end{Equation}
    依据\fancyref{fml:PN结的边界条件},相应的边界条件为
    \begin{Gather}[12pt]
        \fdd{p_n(+x_n)}=p_{n0}\qty[\exp(\frac{eV_a}{\kB T})-1]\xlabelpeq{7}\\
        \fdd{n_p(-x_p)}=n_{p0}\qty[\exp(\frac{eV_a}{\kB T})-1]\xlabelpeq{8}\\
        \fdd{p_n}(+\infty)=0\xlabelpeq{9}\\
        \fdd{n_p}(-\infty)=0\xlabelpeq{10}
    \end{Gather}
    当少数载流子由空间电荷区边界向中性区扩散时,它们将与多数载流子复合并减少。此处假定P型区和N型区的长度$W_p,W_n$远大于扩散长度$L_p,L_n$,故过剩少数载流子$\fdd{p_n}, \fdd{n_p}$在远离空间电荷区时将趋于零,这才有上述$\fdd{p_n}(+\infty)=0$和$\fdd{n_p}(-\infty)=0$的无穷边界条件。

    以空穴为例求解,将\xrefpeq{9}代入\xrefpeq{5},容易定出$P_1=0$,从而
    \begin{Equation}&[11]
        \fdd{p_n(x)}=P_2\exp(-\frac{x}{L_p})
    \end{Equation}
    特别的,当$x=+x_n$时
    \begin{Equation}&[12]
        \fdd{p_n(+x_n)}=P_2\exp(-\frac{x_n}{L_p})
    \end{Equation}
    而\xrefpeq{7}又告诉我们
    \begin{Equation}&[13]
        \fdd{p_n(+x_n)}=p_{n0}\qty[\exp(\frac{eV_a}{\kB T})-1]
    \end{Equation}
    对比\xrefpeq{12}和\xrefpeq{13}可知
    \begin{Equation}&[14]
        P_2=p_{n0}\exp(\frac{x_n}{L_p})\qty[\exp(\frac{eV_a}{\kB T})-1]
    \end{Equation}
    将\xrefpeq{14}代回\xrefpeq{11}
    \begin{Equation}*
        \fdd{p_n(x)}=p_{n0}\qty[\exp(\frac{eV_a}{\kB T})-1]\exp(\frac{x_n-x}{L_p})
    \end{Equation}
    类似也可以得到
    \begin{Equation}*
        \fdd{n_p(x)}=n_{p0}\qty[\exp(\frac{eV_a}{\kB T})-1]\exp(\frac{x_p+x}{L_n})\qedhere
    \end{Equation}
\end{Proof}

\xref{fig:PN结的载流子分布}形象展示了\xref{subsec:PN结的边界条件}和\xref{subsec:PN结的载流子分布}的工作
\begin{itemize}
    \item 平衡少子浓度$p_{n0}, n_{p0}$。
    \item 平衡多子浓度$p_{p0}, n_{n0}$,在$p_{n0}, n_{p0}$上乘$\exp(eV_{bi})$。
    \item 非平衡态下,少子在耗尽区边界处浓度$p_{n}(+x_n), n_p(-x_p)$,在$p_{n0}, n_{p0}$上乘$\exp(eV_a)$。
    \item 非平衡少子$\fdd{p_n}(x), \fdd{n_p}(x)$由边界$+x_n, -x_p$向无穷远$\pm\infty$以指数方式衰减。
\end{itemize}

\begin{Figure}[PN结的载流子分布]
    \includegraphics[scale=1]{build/Chapter02A_03.fig.pdf}
\end{Figure}

\subsection{PN结的理想电流--电压关系}
推导PN结理想电流--电压关系的关键在于\xref{subsec:PN结的理想假设}第四条假设。试想,PN结的总电流密度$J(x)$是由空穴电流$J_p(x)$和电子电流$J_n(x)$的和组成的。而依据第四条假设
\begin{itemize}
    \item $J(x)$是无关$x$的常数$J$,那我们不妨就在耗尽区内计算总电流密度$J(x)=J$。
    \item $J_p(x)$和$J_n(x)$在耗尽区内也是常数,故不妨取边界处计算,即$J=J_p(+x_n)+J_n(-x_p)$。
    \item $J_p(x)$和$J_n(x)$是由少子扩散形成的扩散电流,正比于$p_n(x), n_p(x)$的梯度。
\end{itemize}

以上就是本小节推导$J, J_p(x), J_n(x)$表达式的基本思路。

\begin{BoxEquation}[PN结的理想电流--电压关系]*
    PN结的\uwave{理想电流--电压关系}(Ideal Current-Voltage Relationship)为
    \begin{Equation}
        J=J_s\qty[\exp(\frac{eV_a}{\kB T})-1]
    \end{Equation}
    其中$J_s$称为\uwave{理想反向饱和电流密度}(Ideal Reverse Saturation Current Density)
    \begin{Equation}
        J_s=\qty[\frac{eD_pp_{n0}}{L_p}+\frac{eD_nn_{p0}}{L_n}]
    \end{Equation}
    空穴电流$J_p(x)$满足
    \begin{Equation}
        \qquad\qquad
        J_p(x)=\frac{eD_pp_{n0}}{L_p}\qty[\exp(\frac{eV_a}{\kB T})-1]\exp(\frac{x_n-x}{L_p})\qquad
        x\geq +x_n
        \qquad\qquad
    \end{Equation}
    电子电流$J_n(x)$满足
    \begin{Equation}
        \qquad\qquad
        J_n(x)=\frac{eD_nn_{p0}}{L_n}\qty[\exp(\frac{eV_a}{\kB T})-1]\exp(\frac{x_p+x}{L_n})\qquad
        x\geq -x_p
        \qquad\qquad
    \end{Equation}
\end{BoxEquation}

\begin{Proof}
    空穴扩散电流为
    \begin{Equation}&[1]
        J_p(x)=-eD_p\dv{p_n(x)}{x}
    \end{Equation}
    由于考虑的是均匀掺杂区域,平衡少子浓度$p_{n0}$为常量,故$p_n(x)$可用$\fdd{n_p(x)}$替代
    \begin{Equation}&[2]
        J_p(x)=-eD_p\dv{(\fdd{p_n(x)})}{x}
    \end{Equation}
    代入\fancyref{fml:PN结的载流子分布}
    \begin{Equation}&[3]
        J_p(x)=\frac{eD_pp_{n0}}{L_p}\qty[\exp(\frac{eV_a}{\kB T})-1]\exp(\frac{x_n-x}{L_p})
    \end{Equation}
    电子扩散电流可以通过类似方法得到
    \begin{Equation}&[4]
        J_n(x)=\frac{eD_nn_{p0}}{L_n}\qty[\exp(\frac{eV_a}{\kB T})-1]\exp(\frac{x_p+x}{L_n})
    \end{Equation}
    在\xrefpeq{3}中取$x=+x_n$
    \begin{Equation}&[5]
        J_p(+x_n)=\frac{eD_pp_{n0}}{L_p}\qty[\exp(\frac{eV_a}{\kB T})-1]
    \end{Equation}
    在\xrefpeq{4}中取$x=-x_p$
    \begin{Equation}&[6]
        J_n(-x_p)=\frac{eD_nn_{p0}}{L_n}\qty[\exp(\frac{eV_a}{\kB T})-1]
    \end{Equation}
    将\xrefpeq{5}和\xrefpeq{6}相加
    \begin{Equation}&[7]
        \qquad\qquad
        J=J_p(+x_n)+J_n(-x_p)=\qty[\frac{eD_pp_{n0}}{L_p}+\frac{eD_nn_{p0}}{L_n}]\qty[\exp(\frac{eV_a}{\kB T})-1]
        \qquad\qquad
    \end{Equation}
    若引入代换变量$J_s$
    \begin{Equation}*
        J_s=\qty[\frac{eD_pp_{n0}}{L_p}+\frac{eD_nn_{p0}}{L_n}]
    \end{Equation}
    则\xrefpeq{7}可以简化为
    \begin{Equation}*
        J=J_s\qty[\exp(\frac{eV_a}{\kB T})-1]\qedhere
    \end{Equation}
\end{Proof}

PN结的理想电流--电压关系如\xref{fig:PN结的理想电流--电压关系}所示。尽管上述推导均是在正偏$V_a>0$的背景下进行的,但是反偏$V_a<0$也完全适用该结论。值得注意的是,当$V_a$反偏超过数个$\kB T/eV$后,电流密度$J$将趋于常量$-J_s$,不再随反偏电压变化,故$J_s$也被称为理想反向饱和电流密度。而同时,当$V_a$正偏超过数个$\kB T/eV$后,则$-1$可以被忽略,此时可认为$J\propto\exp(eV_a/\kB T)$。

\begin{Figure}[PN结的理想电流--电压关系]
    \includegraphics[scale=1]{build/Chapter02A_05.fig.pdf}
\end{Figure}

PN结的空穴扩散电流$J_p(x), x\geq +x_n$和电子扩散电流$J_n(x), x\leq -x_p$在中性区内向两端指数衰减,但根据前面的假设,PN结的总电流密度$J$是一个常量,这该怎么解释呢?总电流密度与少子扩散电流密度间的差,来自多子漂移电流。它们补充了多子因注入而造成的损失。

\begin{Figure}[PN结的电流密度分布]
    \includegraphics[scale=1]{build/Chapter02A_04.fig.pdf}
\end{Figure}

当然,应当指出的是,多子漂移电流的存在意味着我们先前中性区的假设并不严谨,因为存在漂移电流就意味着空间电荷区外仍有电场。不过这电场实际很小,故这个假设仍可以适用。

\subsection{短二极管}
在前面的分析中,我们假设P型区域和N型区域的长度$W_p, W_n$都远大于相应的少子扩散长度$L_n, L_p$,即$W_p\gg L_n$且$W_n\gg L_p$(注意此处两者是相反的)。然而,在许多实际PN结的结构中,有一个区域,反而远小于相应的少子扩散长度,即$W_p\ll L_n$或$W_n\ll L_p$。这类PN结被称为\uwave{短二极管}(The Short Diode),相较于通常的\uwave{长二极管}(The Long Diode)。短二极管将导致许多问题,\fancyref{fml:PN结的载流子分布}中,应用了\empx{过剩载流子浓度在无穷远处减少至零}的边界条件,然而在短二极管的短侧,由于区域长度远小于扩散长度,因此边界条件需要相应改为\empx{过剩载流子浓度在区域边界处减小至零}。在这一新边界条件下,我们将会重新推导出短二极管的载流子分布和电流密度分布函数。\xref{fig:短二极管与长二极管}给出了短二极管的结构示意图。

\begin{Figure}[短二极管与长二极管]
    \begin{FigureSub}[长二极管]
        \includegraphics[scale=0.9]{build/Chapter02A_06.fig.pdf}
    \end{FigureSub}\\ \vspace{0.5cm}
    \begin{FigureSub}[短二极管(N侧)]
        \includegraphics[scale=0.9]{build/Chapter02A_07.fig.pdf}
    \end{FigureSub}\hspace{0.25cm}
    \begin{FigureSub}[短二极管(P侧)]
        \includegraphics[scale=0.9]{build/Chapter02A_08.fig.pdf}
    \end{FigureSub}
\end{Figure}\vspace{0.15cm}

\begin{BoxFormula}[短二极管的载流子分布]
    对于N侧较短$W_n\ll L_p$的短二极管,其过剩空穴分布$\fdd{p_n}$需修正为
    \begin{Equation}
        \fdd{p_n}(x)=p_{n0}\qty[\exp(\frac{eV_a}{\kB T})-1]\qty(\frac{W_n+x_n-x}{W_n})
    \end{Equation}
    对于P侧较短$W_p\ll L_n$的短二极管,其过剩电子分布$\fdd{n_p}$需修正为
    \begin{Equation}
        \fdd{n_p}(x)=n_{p0}\qty[\exp(\frac{eV_a}{\kB T})-1]\qty(\frac{W_p+x_p+x}{W_p})
    \end{Equation}
\end{BoxFormula}

\begin{Proof}
    以空穴的分布为例推导,连续性方程的通解仍然是\xrefpeq[PN结的载流子分布]{5}
    \begin{Equation}&[1]
        \fdd{p_n(x)}=P_1\exp(\frac{x}{L_p})+P_2\exp(-\frac{x}{L_p})
    \end{Equation}
    在耗尽区边界处的边界条件\xrefpeq[PN结的载流子分布]{7}仍然满足
    \begin{Equation}&[2]
        \fdd{p_n(x_n)}=p_{n0}\qty[\exp(\frac{eV_a}{\kB T})-1]
    \end{Equation}
    在无穷远处的边界条件则不再成立,取而代之,$\fdd{p_n}$在N区边界$x_n+W$处衰减至零
    \begin{Equation}&[3]
        \fdd{p_n(x_n+W)}=0
    \end{Equation}
    将\xrefpeq{2}和\xrefpeq{3}代入\xrefpeq{1},构建方程阻
    \begin{Equation}&[4]
        \begin{pmatrix}
            \mal{\exp(\frac{x_n}{L_p})}&
            \mal{\exp(-\frac{x_n}{L_p})}\\[6mm]
            \mal{\exp(\frac{x_n+W_n}{L_p})}&
            \mal{\exp(-\frac{x_n+W_n}{L_p})}
        \end{pmatrix}
        \begin{pmatrix}
            P_1\vphantom{\mal{\qty(\frac{1}{1})}}\\[6mm]
            P_2\vphantom{\mal{\qty(\frac{1}{1})}}
        \end{pmatrix}
        =
        \begin{pmatrix}
            \mal{p_{n0}\qty[\exp(\frac{eV_a}{\kB T})-1]}\\[6mm]
            0\vphantom{\mal{\qty(\frac{1}{1})}}
        \end{pmatrix}
    \end{Equation}
    计算$D$
    \begin{Equation}&[5]
        D=\begin{vmatrix}
            \mal{\exp(\frac{x_n}{L_p})}&
            \mal{\exp(-\frac{x_n}{L_p})}\\[6mm]
            \mal{\exp(\frac{x_n+W_n}{L_p})}&
            \mal{\exp(-\frac{x_n+W_n}{L_p})}
        \end{vmatrix}=
        \exp(-\frac{W_n}{L_p})-\exp(\frac{W_n}{L_p})=-2\sinh(\frac{W_n}{L_p})
    \end{Equation}
    计算$D_1$
    \begin{Equation}&[6]
        D_1=\begin{vmatrix}
            \mal{p_{n0}\qty[\exp(\frac{eV_a}{\kB T})-1]}&
            \mal{\exp(-\frac{x_n}{L_p})}\\[6mm]
            0&
            \mal{\exp(-\frac{x_n+W_n}{L_p})}    
        \end{vmatrix}=
        p_{n0}\qty[\exp(\frac{eV_a}{\kB T})-1]\exp(-\frac{x_n+W_n}{L_p})
    \end{Equation}
    计算$D_2$
    \begin{Equation}&[7]
        D_2=\begin{vmatrix}
            \mal{\exp(\frac{x_n}{L_p})}&
            \mal{p_{n0}\qty[\exp(\frac{eV_a}{\kB T})-1]}\\[6mm]
            \mal{\exp(\frac{x_n+W_n}{L_p})}&
            0
        \end{vmatrix}=
        -p_{n0}\qty[\exp(\frac{eV_a}{\kB T})-1]\exp(\frac{x_n+W_n}{L_p})
    \end{Equation}
    其中,$P_1=D_1/D$,$P_2=D_2/D$,故\xrefpeq{1}表示为
    \begin{Equation}&[8]
        \fdd{p_n(x)}=\frac{D_1}{D}\exp(\frac{x}{L_p})+\frac{D_2}{D}\exp(-\frac{x}{L_p})
    \end{Equation}
    将\xrefpeq{5}、\xrefpeq{6}、\xrefpeq{7}代入\xrefpeq{8}
    \begin{Equation}&[9]
        \fdd{p_n(x)}=p_{n0}\qty[\exp(\frac{eV_a}{\kB T})-1]\qty[\exp(-\frac{x_n+W_n-x}{L_p})-\exp(\frac{x_n+W_n-x}{L_p})]\frac{1}{-2\sinh(W_n/L_p)}
    \end{Equation}
    再次用双曲正弦简化
    \begin{Equation}
        \fdd{p_n(x)}=p_{n0}\qty[\exp(\frac{eV_a}{\kB T})-1]\frac{\sinh[(W_n+x_n-x)/L_p]}{\sinh[W_n/L_p]}
    \end{Equation}
    而当$W_n\ll L_p$时,可以进一步近似为
    \begin{Equation}*
        \fdd{p_n(x)}=p_{n0}\qty[\exp(\frac{eV_a}{\kB T})-1]\qty(\frac{W_n+x_n-x}{W_n})
    \end{Equation}
    而对于$W_p\ll L_n$的情况,类似可以正面$\fdd{n_p(x)}$满足
    \begin{Equation}*
        \fdd{n_p(x)}=n_{p0}\qty[\exp(\frac{eV_a}{\kB T})-1]\qty(\frac{W_p+x_p+x}{W_p})\qedhere
    \end{Equation}
\end{Proof}

\begin{BoxFormula}[短二极管的电流密度]
    对于N侧较短$W_n\ll L_p$的短二极管,其$J_p(x)$应修正为
    \begin{Equation}
        J_p(x)=\frac{eD_pp_{n0}}{W_n}\qty[\exp(\frac{eV_a}{\kB T})-1]
    \end{Equation}
    对于P侧较短$W_p\ll L_n$的短二极管,其$J_n(x)$应修正为
    \begin{Equation}
        J_n(x)=\frac{eD_nn_{p0}}{W_p}\qty[\exp(\frac{eV_a}{\kB T})-1]
    \end{Equation}
\end{BoxFormula}

\begin{Proof}
    以空穴而的分布为例推导,根据\fancyref{fml:短二极管的载流子分布}
    \begin{Equation}&[1]
        \fdd{p_n}(x)=p_{n0}\qty[\exp(\frac{eV_a}{\kB T})-1]\qty(\frac{W_n+x_n-x}{W_n})
    \end{Equation}
    空穴扩散电流为
    \begin{Equation}&[2]
        J_p(x)=-eD_p\dv{(\fdd{p_n(x)})}{x}
    \end{Equation}
    代入\xrefpeq{1}
    \begin{Equation}*
        J_p(x)=\frac{eD_pp_{n0}}{W_n}\qty[\exp(\frac{eV_a}{\kB T})-1]
    \end{Equation}
    电子扩散电流可以类似求得
    \begin{Equation}*
        J_n(x)=\frac{eD_nn_{p0}}{W_p}\qty[\exp(\frac{eV_a}{\kB T})-1]\qedhere
    \end{Equation}
\end{Proof}

通过\fancyref{fml:短二极管的载流子分布}和\fancyref{fml:短二极管的电流密度}
\begin{itemize}
    \item 长二极管至短二极管,载流子分布由指数衰减变为了线性衰减。
    \item 长二极管至短二极管,电流密度分布由指数衰减变为了常量。
\end{itemize}
除此之外,长二极管的公式包含$L_p,L_n$,短二极管的公式则包含$W_n,W_p$,这是因为在短二极管中区域长度相较扩散长度是更重要的因素。同时,我们刚刚已提到,在短二极管中,少数载流子分布变为线性,少子扩散电流密度变为常量,而恒定电流密度表明,\empx{短区中不存在复合}。


\section{PN结的产生--复合电流和大注入}
在\xref{sec:PN结电流},推导理想电流--电压关系时,我们假定了小注入并忽略了耗尽区的影响,然而
\begin{itemize}
    \item 在耗尽区中实际会发生载流子的产生与复合,这并不影响载流子分布,后者是由能带决定的。但是,为了抵消其带来的影响,将有额外电流产生。这就是产生电流和复合电流。
    \item 在过去,我们总是假定PN结是满足\uwave{小注入}(Low-Level Injection)近似的,耗尽区边界处的非平衡载流子,对于少子而言很显著,对于多子而言则可以忽略。但是随着PN结两端的电压增大,非平衡载流子的浓度将同时高于平衡少子浓度和平衡多子浓度,非平衡少子和非平衡多子都需要被充分考虑,这就是所谓的\uwave{大注入}(High-Level Injection)。
\end{itemize}

这些非理想因素会导致PN结的电流--电压关系偏离其理想表达式。

\subsection{间接复合理论}
在开始前,我们简要回顾一下\uwave{间接复合},亦称为\uwave{肖克利--里德--霍尔复合}(Shockley-Read-Hall Recombination, SRH)的理论。在间接复合理论中,电子和空穴的净复合率$R$可以表示为
\begin{BoxFormula}[间接复合率]
    间接复合率可以表示为
    \begin{Equation}
        R=\frac{C_nC_pN_t(np-n_i^2)}{C_n(n+n')+C_p(p+p')}
    \end{Equation}
    其中,$N_t$是掺杂的复合中心的浓度,$C_n,C_p$是电子和空穴的俘获系数。
\end{BoxFormula}

在这里需要说明的$n',p'$的含义,我们知道,$n$和$p$是电子和空穴浓度,可以表示为
\begin{Equation}
    n=N_c\exp(\frac{E_{Fn}-E_c}{\kB T})\qquad
    p=N_v\exp(\frac{E_v-E_{Fp}}{\kB T})
\end{Equation}\goodbreak
而$n',p'$的意义是,当费米能级$E_F$置于复合中心能级$E_t$时对应的“浓度”
\begin{Equation}
    n'=N_c\exp(\frac{E_t-E_c}{\kB T})\qquad
    p'=N_v\exp(\frac{E_v-E_t}{\kB T})
\end{Equation}
而很多时候,近似认为复合中心能级$E_t$位于$E_{Fi}$附近,因此
\begin{Equation}
    n_i=N_c\exp(\frac{E_{Fi}-E_c}{\kB T})=N_v\exp(\frac{E_v-E_{Fi}}{\kB T})=n'=p'
\end{Equation}
该近似是本节推导的一个重要工具。

另外,在间接复合中定义有电子寿命$\tau_{n0}$和空穴寿命$\tau_{p0}$,它们反映了过剩载流子的存在时间。
\begin{BoxDefinition}[载流子的寿命]
    电子寿命$\tau_{n0}$被定义为
    \begin{Equation}
        \tau_{n0}=\frac{1}{N_tC_n}
    \end{Equation}
    空穴寿命$\tau_{p0}$被定义为
    \begin{Equation}
        \tau_{p0}=\frac{1}{N_tC_p}
    \end{Equation}
\end{BoxDefinition}

另还常定义有平均寿命
\begin{BoxDefinition}[载流子的平均寿命]
    过剩载流子的平均寿命$\tau_0$被定义为
    \begin{Equation}
        \tau_0=\frac{\tau_{p0}+\tau_{n0}}{2}
    \end{Equation}
\end{BoxDefinition}

\subsection{反偏产生电流}\setpeq{反偏产生电流}
对于一个反偏状态下的PN结,我们认为其在空间电荷区不存在可以移动的电子和空穴。换言之,在空间电荷区中,可以取$n=p=0$,因此,\fancyref{fml:间接复合率}可以简化为
\begin{Equation}&[1]
    R=\frac{-C_nC_pN_tn_i^2}{C_nn'+C_pp'}
\end{Equation}
这里的负号表明负的净复合率,这意味着,在反偏的空间电荷区中,电子--空穴对实际是在产生而非复合。这很容易理解,因为,在反偏时空间电荷区中电子和空穴的浓度几乎是零,因此电子--空穴对被产生以试图重新建立热平衡。当电子和空穴产生后,它们被电场扫出空间电荷区,它们在扫出的运动过程中就形成了\uwave{反偏产生电流}(Reverse-Biased Generation Current)。由于反偏电压是由N区指向P区的,产生的空穴向P区移动,产生的电子向N区移动,故反偏产生电流$J_{gen}$和原先理想的反偏饱和电流$J_{s}$的方向是相同的,均是由N区指向P区。

我们下面来计算反偏产生电流$J_{gen}$,在\xrefpeq{1}中取$n_i=n'=p'$的近似
\begin{Equation}&[2]
    R=\frac{-C_nC_pN_tn_i^2}{C_nn'+C_pp'}
\end{Equation}
上下同除以$C_nC_pN_tn_i$
\begin{Equation}&[3]
    R=\frac{-n_i}{(1/N_tC_p)+(1/N_tC_n)}
\end{Equation}
应用\fancyref{def:载流子的寿命}代换
\begin{Equation}
    R=\frac{-n_i}{\tau_{p0}+\tau_{n0}}
\end{Equation}
应用\fancyref{def:载流子的平均寿命}代换
\begin{Equation}
    R=-\frac{n_i}{2\tau_0}
\end{Equation}
这是一个负的净复合率,我们将其转用净产生率来表示。
\begin{BoxFormula}[反偏PN结的耗尽区产生率]
    对于一个反偏的PN结,其耗尽区的产生率为
    \begin{Equation}
        G=\frac{n_i}{2\tau_0}
    \end{Equation}
\end{BoxFormula}
那么,产生电流该如何计算呢?显然,单位时间内在空间电荷区产生多少电子--空穴对,产生电流就将有多大,因此,产电流$J_{gen}$即净产生率$G$在空间电荷区上的积分,而$G$又是常数
\begin{Equation}
    J_{gen}=\Int[0][W]eG\dx=\Int[0][W]\frac{en_i}{2\tau_0}\dx=\frac{en_iW}{2\tau_0}
\end{Equation}
这就有
\begin{BoxFormula}[反偏产生电流]
    反偏产生电流$J_{gen}$为
    \begin{Equation}
        J_{gen}=\frac{en_iW}{2\tau_0}
    \end{Equation}
\end{BoxFormula}
而完整的反偏电流是由反偏饱和电流$J_s$和反偏产生电流$J_{gen}$两部分组成
\begin{Equation}
    J_R=J_s+J_{gen}
\end{Equation}
而其中
\begin{Equation}
    J_s=\frac{eD_pp_{n0}}{L_p}+\frac{eD_nn_{p0}}{L_n}\qquad
    J_{gen}=\frac{en_iW}{2\tau_0}
\end{Equation}
形式上$J_s$和$J_{gen}$都是常量,但是应当注意到$J_{gen}$关于耗尽区长度$W$而后者随着反偏电压的增大而增大。因此,实际上PN结的反偏电流并不趋于饱和,而是会随反偏电压略微增大。\goodbreak

\subsection{正偏复合电流}
对于一个正偏状态下的PN结,相反,电子和空穴被注入空间电荷区,因此,空间电荷区存在过剩的载流子。电子和空穴通过空间电荷区的过程中,有一定概率会在空间电荷区内发生复合而无法到达空间电荷区边界参与少子扩散。为此,额外的电子和空穴将被注入以弥补这种损耗,额外注入的这些载流子就形成了\uwave{正偏复合电流}(Forward-Biased Recombination Current)。

和前面的思路一样,我们先来计算复合率,随后计算复合电流的大小。

根据\fancyref{fml:间接复合率}
\begin{Equation}&[1]
    R=\frac{C_nC_pN_t(np-n_i^2)}{C_n(n+n')+C_p(p+p')}
\end{Equation}
上下同除$C_nC_pN_t$
\begin{Equation}&[2]
    R=\frac{np-n_i^2}{(1/C_pN_t)(n+n')+(1/C_nN_t)(p+p')}
\end{Equation}
代入\fancyref{def:载流子的寿命}
\begin{Equation}&[3]
    R=\frac{np-n_i^2}{\tau_{p0}(n+n')+\tau_{n0}(p+p')}
\end{Equation}
作$n_i=n'=p'$的近似
\begin{Equation}&[4]
    R=\frac{np-n_i^2}{\tau_{p0}(n+n_i)+\tau_{n0}(p+n_i)}
\end{Equation}
作$\tau_0=\tau_{p0}=\tau_{n0}$的近似
\begin{Equation}
    R=\frac{np-n_i^2}{\tau_0\qty(2n_i+n+p)}
\end{Equation}

而我们知道,$n,p$可以很容易的用$n_i$表示
\begin{Equation}
    n=n_i\exp(\frac{E_{Fn}-E_{Fi}}{\kB T})
    \qquad
    p=n_i\exp(\frac{E_{Fi}-E_{Fp}}{\kB T})
\end{Equation}
而两者的乘积则可以表示为(参照\xref{fig:PN结能带图},有$E_{Fn}-E_{Fp}=eV_a$)
\begin{Equation}
    np=n_i^2\exp(\frac{E_{Fn}-E_{Fp}}{\kB T})=n_i^2\exp(\frac{eV_a}{\kB T})
\end{Equation}
换言之,$np$是定值。而由\xref{fig:PN结能带图}注意到,$E_{Fp},E_{Fn}$在空间电荷区内是定值,而$E_{Fi}$随着$x$增加而减小,由接近$E_{Fp}$减小至接近$E_{Fn}$,而在该过程中,$p$逐渐减小,$n$逐渐增大。而可以证明的是复合率$R$关于$x$的函数是一个在$x=0$取极值$R_{\max}$的尖峰,因为$x=0$处$p,n$都能取一个较适中的值,使复合率最大。而由于$x=0$处$E_{Fn}-E_{Fi}=E_{Fi}-E_{Fp}=eV_a/2$
\begin{Equation}
    n=p=n_i\exp(\frac{eV_a}{2\kB T})
\end{Equation}

因此$x=0$处的最大复合率$R_{max}$就可以表示为
\begin{Equation}
    R_{\max}=\frac{n_i^2\qty[\exp(eV_a/\kB T)-1]}{2n_i\tau_0[\exp(eV_{a}/2\kB T)+1]}
\end{Equation}
当$V_a$较大时,上式中的$\pm 1$都可以忽略,故
\begin{BoxFormula}[正偏PN结的耗尽区复合率]
    对于一个正偏的PN结,其耗尽区中心的最大复合率为
    \begin{Equation}
        R_{\max}=\frac{n_i}{2\tau_0}\exp(\frac{eV_a}{2\kB T})
    \end{Equation}
\end{BoxFormula}
尽管,我们只计算出了$R$在$x=0$处的最大值$R_{\max}$,但考虑到空间电荷区很窄,我们可以近似认为在空间电荷区中$R=R_{\max}$为常数。这样,计算正偏复合电流$J_{rec}$就同样简单了
\begin{Equation}
    J_{rec}=\Int[0][W]eR\dx=\Int[0][W]eR_{\max}\dx=\Int[0][W]\frac{en_i}{2\tau_0}\exp(\frac{eV_a}{2\kB T})\dx=\frac{en_iW}{2\tau_0}\exp(\frac{eV_a}{2\kB T})
\end{Equation}
这就有
\begin{BoxFormula}[正偏复合电流]
    正偏复合电流$J_{rec}$为
    \begin{Equation}
        J_{rec}=\frac{en_iW}{2\tau_0}\exp(\frac{eV_a}{2\kB T})
    \end{Equation}
    或记为
    \begin{Equation}
        J_{rec}=J_{r0}\exp(\frac{eV_a}{2\kB T})
    \end{Equation}
\end{BoxFormula}
而完整的正偏电流是由扩散电流$J_D$和复合电流两部分组成
\begin{Equation}
    J=J_D+J_{rec}
\end{Equation}
而这两项分别等于(这里省略$J_D$表达式中的$-1$项)
\begin{Equation}&[x]
    J_D=J_s\exp(\frac{eV_a}{\kB T})\qquad J_{rec}=J_{r0}\exp(\frac{eV_a}{2\kB T})
\end{Equation}
事实是系数$J_{r0}>J_D$,这就意味着尽管$J_{rec}\propto\exp(eV_a/2\kB T)$相较$J_{D}\propto\exp(eV_a/\kB T)$增长较慢,理应被忽略,但由于$J_{rec}$前的系数大于$J_D$,电压较小时,复合电流仍将占主导地位
\begin{itemize}
    \item 当电压$V_a$较小时,$J$将主要由复合电流$J_s$决定,正比于$\exp(eV_a/2\kB T)$增长。
    \item 当电压$V_a$较大时,$J$将主要由扩散电流$J_D$决定,正比于$\exp(eV_a/\kB T)$增长。
\end{itemize}
如果在\xrefpeq{x}两端取对数
\begin{Equation}
    \ln J_D=\ln J_s+\frac{eV_a}{\kB T}\qquad
    \ln J_{rec}=\ln J_{r0}+\frac{eV_a}{2\kB T}
\end{Equation}
并将这种关系在对数图上呈现,如\xref{fig:实际PN结的电流--电压关系}所示,上述这种主导关系会显得更为清晰。

\begin{Figure}[实际PN结的电流--电压关系]
    \includegraphics{build/Chapter02B_01.fig.pdf}
\end{Figure}

\subsection{大注入}
在推导PN结的理想电流--电压关系时,假设小注入是成立的
\begin{itemize}
    \item 小注入,非平衡载流子浓度低,远小于平衡多子。仅考虑非平衡少子。
    \item 大注入,非平衡载流子浓度高,远大于平衡多子。需考虑非平衡少子和非平衡多子。
\end{itemize}
当正偏电压增大,非平衡载流子浓度增加,就需要充分考虑大注入带来的非理想因素了。

我们已经很熟悉,对于PN结
\begin{Equation}
    np=n_i^2\exp(\frac{V_a}{\kB T})
\end{Equation}
而$n,p$可以分别表示为$n_0+\fdd{n}$和$p_0+\fdd{p}$
\begin{Equation}
    (n_0+\fdd{n})(p_0+\fdd{p})=n_i^2\exp(\frac{V_a}{\kB T})
\end{Equation}
由于在大注入情况下,$\fdd{n}\gg n_0, \fdd{p}\gg p_0$
\begin{Equation}
    (\fdd{n})(\fdd{p})=n_i^2\exp(\frac{V_a}{\kB T})
\end{Equation}
而又考虑到$\fdd{n}=\fdd{n}$
\begin{Equation}
    \fdd{n}=\fdd{p}=n_i\exp(\frac{V_a}{\kB T})
\end{Equation}
电流密度又正比于非平衡载流子的浓度,因此,在大注入情形下
\begin{Equation}
    J\propto \exp(\frac{V_a}{2\kB T})
\end{Equation}
我们可以将PN结的电流--电压关系统一记为
\begin{Equation}
    J\propto\exp(\frac{V_a}{n\kB T})
\end{Equation}
如\xref{fig:实际PN结的电流--电压关系}所示,随电压增加,复合电流阶段$n=2$,扩散电流阶段$n=1$,大注入阶段$n=2$。

\section{PN结二极管的小信号模型}
前面我们一直在讨论PN结二极管的直流特性。当包含PN结的半导体器件用于线性放大电路中时,交变的正弦信号就会叠加在直流上。因此PN结的小信号特性就显得尤为重要了。

\subsection{扩散电阻}
PN结的理想电流--电压关系已经在\fancyref{eqt:PN结的理想电流--电压关系}给出
\begin{Equation}
    J=J_s\qty[\exp(\frac{eV_a}{\kB T})-1]
\end{Equation}
但作为器件,我们更希望采用电流而不是电流密度,故两端同乘以PN结的面积$A$
\begin{BoxFormula}[二极管的特性方程]
    二极管的电流和电压间的关系是
    \begin{Equation}
        I_D=I_s\qty[\exp(\frac{eV_a}{\kB T})-1]
    \end{Equation}
    其中$I_D$是\uwave{二极管}(Diode)电流,而$I_s$是二极管的反向饱和电流。
\end{BoxFormula}

假定二极管正偏,直流电压$V_0$,直流电流$I_{DQ}$,如果我们在$V_0$上叠加一个幅度很小频率不太高的正弦电压$v_1(t)=\hat{V_1}\e^{\j\omega t}$,那么将产生一个小的正弦电流叠加在$I_{DQ}$上。我们将正弦电流和正弦电压间的比定义为增量电导,由于小信号的幅度很小,因此小信号增量电导事实上就是直流电流$I_{DQ}$对直流电压$V_a$在$V_0$处的导数值。这就是二极管的扩散电导(电阻)的含义。

\begin{BoxDefinition}[二极管的扩散电导]
    二极管的扩散电导定义为
    \begin{Equation}
        g_d=\eval{\dv{I_D}{V_a}}_{V_a=V_0}
    \end{Equation}
    二极管的扩散电阻定义为扩散电导的倒数
    \begin{Equation}
        r_d=\frac{1}{g_d}
    \end{Equation}
\end{BoxDefinition}

\begin{BoxFormula}[二极管的扩散电导]
    二极管的扩散电导可以表示为
    \begin{Equation}
        g_d=\frac{e}{\kB T}I_{DQ}
    \end{Equation}
\end{BoxFormula}

\begin{Proof}
    根据\fancyref{fml:二极管的特性方程}
    \begin{Equation}&[1]
        I_D=I_s\qty[\exp(\frac{eV_a}{\kB T})-1]
    \end{Equation}
    假定$V_a$较大,忽略$-1$
    \begin{Equation}&[2]
        I_D=I_s\exp(\frac{eV_a}{\kB T})
    \end{Equation}
    若取$V_a=V_0$,则$I_D=I_{DQ}$
    \begin{Equation}&[3]
        I_{DQ}=I_s\exp(\frac{eV_0}{\kB T})
    \end{Equation}
    根据\fancyref{def:二极管的扩散电导},先后代入\xrefpeq{2}和\xrefpeq{3}
    \begin{Equation}*
        g_d=\eval{\dv{I_D}{V_a}}_{V_a=V_0}=\frac{e}{\kB T}I_s\exp(\frac{eV_0}{\kB T})=\frac{eI_{DQ}}{\kB T}\qedhere
    \end{Equation}
\end{Proof}

根据\fancyref{fml:二极管的扩散电导},随着二极管的电流增大,其扩散电阻逐渐减小。

\subsection{扩散电容和小信号导纳}
我们或许会很难理解,为什么PN结二极管会有什么扩散电容呢?在定量计算扩散电容前,我们先定性分析一下扩散电容的来源。如\xref{fig:扩散电容的来源}所示,在直流电压下$V_0$下,PN结在两侧分别保持了一定的电荷分布,P区为带负电的过剩电子$-\delt{Q}$,N区为带正电的过剩空穴$+\delt{Q}$,此时边界处的载流子浓度正比于$\exp(e V_0/\kB T)$。在交流电压$v_1(t)=\hat{V_1}\e^{\j\omega t}$的作用下,PN结两端的电压将在$V_0-\hat{V_1}$和$V_0+\hat{V_1}$之间变化,该过程中,耗尽区边界处的载流子浓度也将相应在正比于${\exp}[e(V_0-\hat{V_1})/\kB T]$和正比于${\exp}[e(V_0+\hat{V_1})/\kB T]$间变化,即N区和P区内“存储”的正负电荷,将随交流电压$v_1(t)$的变化而增减,这就是\uwave{扩散电容}(Diffusion Capacitance)。
\begin{BoxFormula}[二极管的扩散电容]
    二极管的扩散电容可以表示为
    \begin{Equation}
        C_d=\frac{e}{2\kB T}\qty(I_{p0}\tau_{p0}+I_{n0}\tau_{n0})
    \end{Equation}
\end{BoxFormula}

\begin{Figure}[扩散电容的来源]
    \includegraphics[width=0.95\linewidth]{build/Chapter02C_01.fig.pdf}
\end{Figure}

我们下面将定量计算PN结的扩散电容$C_d$(实际我们将直接计算小信号下的导纳)。

\begin{BoxFormula}[PN结的小信号导纳]
    PN结的小信号导纳为
    \begin{Equation}
        Y=\frac{e}{\kB T}(I_{p0}+I_{n0})+\j\omega\frac{e}{2\kB T}\qty(I_{p0}\tau_{p0}+I_{n0}\tau_{n0})
    \end{Equation}
    导纳可以表示为
    \begin{Equation}
        Y=g_d+\j\omega C_d
    \end{Equation}
    其中$g_d$是扩散电阻
    \begin{Equation}
        g_d=\frac{e}{\kB T}(I_{p0}+I_{n0})=\frac{e}{\kB T}I_{DQ}
    \end{Equation}
    其中$C_d$是扩散电容
    \begin{Equation}
        C_d=\frac{e}{2\kB T}\qty(I_{p0}\tau_{p0}+I_{n0}\tau_{n0})
    \end{Equation}
    其中$I_{p0}$和$I_{n0}$分别是
    \begin{Gather}[12pt]
        I_{p0}=\frac{eAD_pp_{n0}}{L_p}\exp(\frac{eV_0}{\kB T})\\
        I_{n0}=\frac{eAD_nn_{p0}}{L_n}\exp(\frac{eV_0}{\kB T})
    \end{Gather}
\end{BoxFormula}

\begin{Proof}
    如\xref{fig:扩散电容的来源}所示,我们在直流电压$V_0$上叠加了交流电压$v_1(t)=\hat{V_1}\e^{\j\omega t}$,总电压$V_a$为
    \begin{Equation}&[1]
        V_a=V_0+v_1(t)
    \end{Equation}
    我们以空穴在N区的扩散为例分析,简洁起见,临时指定边界$x_n$处为$x=0$,故
    \begin{Equation}&[2]
        p_n(0,t)=p_{n0}\exp[\frac{e[V_0+v_1(t)]}{\kB T}]
    \end{Equation}
    将其常量和变量部分分离
    \begin{Equation}&[3]
        p_n(0,t)=p_{n0}\exp(\frac{eV_0}{\kB T})\exp(\frac{ev_1(t)}{\kB T})
    \end{Equation}
    或
    \begin{Equation}&[4]
        p_n(0,t)=p_\te{dc}\exp\qty(\frac{ev_1(t)}{\kB T})
    \end{Equation}
    这里$p_\te{dc}$为
    \begin{Equation}&[5]
        p_\te{dc}=p_{n0}\exp(\frac{eV_0}{\kB T})
    \end{Equation}
    让我们回到\xrefpeq{4},假定$|v_1(t)|\ll \kB T/e$,那么可以作一阶泰勒展开
    \begin{Equation}&[6]
        p_n(0,t)=p_\te{dc}\qty[1+\frac{ev_1(t)}{\kB T}]
    \end{Equation}
    我们已知$v_1(t)=\hat{V_1}\e^{\j\omega t}$,代入上式
    \begin{Equation}&[7]
        p_n(0,t)=p_\te{dc}\qty[1+\frac{e\hat{V_1}}{\kB T}\e^{\j\omega t}]
    \end{Equation}
    或者
    \begin{Equation}&[7.5]
        \fdd{p_n(0,t)}=p_\te{dc}\qty[1+\frac{e\hat{V_1}}{\kB T}\e^{\j\omega t}]-p_{n0}
    \end{Equation}
    \xrefpeq{7.5}将作为接下来求解微分方程时的边界条件。

    在N型区中电场为零,故根据\fancyref{eqt:连续性方程},过剩空穴服从以下方程
    \begin{Equation}&[8]
        D_p\pdv[2]{\fdd{p_n}}{x}-\frac{\fdd{p_n}}{\tau_{p0}}=\pdv{\fdd{p_n}}{t}
    \end{Equation}
    在这里,我们期望$\fdd{p_n}(x)$将会是正弦解$\fdd{p_1}(x)\e^{\j\omega t}$叠加在稳态解$\fdd{p_0}(x)$上的形态
    \begin{Equation}&[9]
        \fdd{p_n(x,t)}=\fdd{p_0(x)}+\fdd{p_1(x)}\e^{\j\omega t}
    \end{Equation}
    将\xrefpeq{9}代入\xrefpeq{8}
    \begin{Equation}&[10]
        \qquad\qquad
        \qty[D_p\pdv[2]{\fdd{p_0(x)}}{x}+D_p\pdv[2]{\fdd{p_1(x)}}{x}\e^{\j\omega t}]-
        \qty[\frac{\fdd{p_0(x)}}{\tau_{p0}}+\frac{\fdd{p_1(x)}}{\tau_{p0}}\e^{\j\omega t}]=\j\omega\fdd{p_1(x)}\e^{\j\omega t}
        \qquad\qquad
    \end{Equation}
    将其重新按$\fdd{p_0}(x), \fdd{p_1}(x)$整理
    \begin{Equation}&[11]
        \qquad\qquad
        \qty[D_p\pdv[2]{\fdd{p_0(x)}}{x}-\frac{\fdd{p_0(x)}}{\tau_{p0}}]+\qty[D_p\pdv[2]{\fdd{p_1(x)}}{x}-\frac{\fdd{p_1(x)}}{\tau_{p0}}-\j\omega\fdd{p_1}(x)]\e^{\j\omega t}=0
        \qquad\qquad
    \end{Equation}
    \xrefpeq{11}的第一项就是\xrefpeq{8}中代入$\fdd{p_0(x)}$,而$\fdd{p_0(x)}$是稳态解,对时间的导数为零,故
    \begin{Equation}&[12]
        D_p\dv[2]{\fdd{p_1(x)}}{x}-\frac{\fdd{p_1(x)}}{\tau_{p0}}-\j\omega\fdd{p_1}(x)=0
    \end{Equation}
    整理
    \begin{Equation}&[13]
        D_p\dv[2]{\fdd{p_1(x)}}{x}-\frac{[1+\j\omega\tau_{p0}]\fdd{p_1(x)}}{\tau_{p0}}=0
    \end{Equation}
    两端同除$D_p$
    \begin{Equation}&[14]
        \dv[2]{\fdd{p_1(x)}}{x}-\frac{[1+\j\omega\tau_{p0}]\fdd{p_1(x)}}{D_p\tau_{p0}}=0
    \end{Equation}
    代入$L_p^2=D_p\tau_{p0}$
    \begin{Equation}&[15]
        \dv[2]{\fdd{p_1(x)}}{x}-\frac{[1+\j\omega\tau_{p0}]\fdd{p_1(x)}}{L_p^2}=0
    \end{Equation}
    引入代换变量$C_p^2$
    \begin{Equation}&[16]
        C_p^2=\frac{1+\j\omega\tau_{p0}}{L_p^2}
    \end{Equation}
    这样\xrefpeq{15}就可以表示为
    \begin{Equation}&[17]
        \dv[2]{\fdd{p_1(x)}}{x}-C_p^2\fdd{p_1(x)}=0
    \end{Equation}
    它的通解是
    \begin{Equation}&[18]
        \fdd{p_1(x)}=K_1\e^{-C_px}+K_2\e^{+C_px}
    \end{Equation}
    由于$\fdd{p_1(\infty)=0}$,因此$K_2=0$,故
    \begin{Equation}&[19]
        \fdd{p_1(x)}=K_1\e^{-C_px}
    \end{Equation}
    比较\xrefpeq{7.5}和\xrefpeq{9}
    \begin{Equation}&[20]
        \fdd{p_n(0,t)}=p_\te{dc}\qty[1+\frac{e\hat{V_1}}{\kB T}\e^{\j\omega t}]-p_{n0}\qquad
        \fdd{p_n(0,t)}=\fdd{p_0(0)}+\fdd{p_1(0)\e^{\j\omega t}}
    \end{Equation}
    依据对应系数相等的原则,容易得到
    \begin{Equation}&[21]
        \fdd{p_1(0)}=p_\te{dc}\frac{e\hat{V_1}}{\kB T}
    \end{Equation}
    这样就定出$K_1=p_\text{dc}(e\hat{V_1}/\kB T)$
    \begin{Equation}&[22]
        \fdd{p_1(x)}=p_\te{dc}\frac{e\hat{V_1}}{\kB T}\e^{-C_px}
    \end{Equation}
    而$\fdd{p_0(x)}$我们已经在\fancyref{fml:PN结的载流子分布}中求过一遍了
    \begin{Equation}&[23]
        \fdd{p_0(x)}=p_{n0}\qty[\exp(\frac{eV_0}{\kB T})-1]\e^{-x/L_p}
    \end{Equation}
    现在让我们来计算边界$x=0$处的空穴扩散电流
    \begin{Equation}&[24]
        J_p=-eD_p\eval{\pdv{\fdd{p_n}}{x}}_{x=0}=-eD_p\eval{\pdv{\fdd{p_0}}{x}}_{x=0}-eD_p\eval{\pdv{\fdd{p_1}}{x}}_{x=0}\e^{\j\omega t}
    \end{Equation}
    或者也可以记为
    \begin{Equation}&[25]
        J_{p}=J_{p0}+j_p(t)=J_{p0}+\hat{J_p}\e^{\j\omega t}
    \end{Equation}
    即$J_{p0}$给出稳态空穴分布的扩散电流,而$j_{p0}=\hat{J_p}\e^{\j\omega t}$给出时变空穴分布的扩散电流。需要指出的是,由于我们在研究小信号模型,这里其实并不关心总电流,只关心相量$\hat{J_p}$的表达式。

    此处的$J_{p0}$已经在\fancyref{eqt:PN结的理想电流--电压关系}计算过了
    \begin{Equation}&[26]
        J_{p0}=-eD_p\eval{\pdv{\fdd{p_0}}{x}}_{x=0}=\frac{eD_pp_{n0}}{L_p}\qty[\exp(\frac{eV_0}{\kB T})-1]
    \end{Equation}
    此处的$\hat{J_p}$则对\xrefpeq{22}求导得到
    \begin{Equation}&[27]
        \hat{J_p}=-eD_p\eval{\pdv{\fdd{p_1}}{x}}_{x=0}=eD_pC_pp_\te{dc}\frac{e\hat{V_1}}{\kB T}
    \end{Equation}
    将\xrefpeq{27}转换为电流的形式
    \begin{Equation}&[29]
        \hat{I_p}=eAD_pC_pp_\te{dc}\frac{e\hat{V_1}}{\kB T}
    \end{Equation}
    在\xrefpeq{29}中代入\xrefpeq{16}中关于$C_p$的表达式
    \begin{Equation}&[30]
        \hat{I_p}=\frac{eAD_pp_\te{dc}}{L_p}\sqrt{1+\j\omega\tau_{p0}}\frac{e\hat{V_1}}{\kB T}
    \end{Equation}
    在\xrefpeq{30}中代入\xrefpeq{5}中关于$p_\te{dc}$的表达式
    \begin{Equation}&[31]
        \hat{I_p}=\frac{eAD_pp_\te{n0}}{L_p}\exp(\frac{eV_0}{\kB T})\sqrt{1+\j\omega\tau_{p0}}\frac{e\hat{V_1}}{\kB T}
    \end{Equation}
    这样$\hat{I_p}$就可以最终简化为(此处$I_{p0}$可以认为是$J_{p0}$忽略$-1$后对应的电流)
    \begin{Equation}&[32]
        \hat{I_p}=I_{p0}\sqrt{1+\j\omega\tau_{p0}}\frac{e\hat{V_1}}{\kB T}\qquad
        I_{p0}=\frac{eAD_pp_{n0}}{L_p}\exp(\frac{eV_0}{\kB T})
    \end{Equation}
    这里类似可以导出$\hat{I_n}$的表达式
    \begin{Equation}&[33]
        \hat{I_n}=I_{n0}\sqrt{1+\j\omega\tau_{n0}}\frac{e\hat{V_1}}{\kB T}\qquad
        I_{n0}=\frac{eAD_nn_{p0}}{L_n}\exp(\frac{eV_0}{\kB T})
    \end{Equation}
    而总的交流电流相量$\hat{I}$是$\hat{I_p}$和$\hat{I_n}$的和,基于此,我们可以计算PN结的导纳$Y$
    \begin{Equation}&[34]
        Y=\frac{\hat{I}}{\hat{V_1}}=\frac{\hat{I_p}+\hat{I_n}}{\hat{V_1}}
    \end{Equation}
    代入\xrefpeq{32}和\xrefpeq{33}
    \begin{Equation}&[35]
        Y=\frac{e}{\kB T}\qty[I_{p0}\sqrt{1+\j\omega\tau_{p0}}+I_{n0}\sqrt{1+\j\omega\tau_{n0}}]
    \end{Equation}
    然而,任何线性的(Linear)、集中的(Lumped)、有限的(Finite)、无源的(Passive)、对称的(Bilateral)的电路网络,都无法描述上述\xrefpeq{35}的导纳函数表达式,它实在是太复杂了!

    但是,如果交流信号的频率不是很高,即假设有
    \begin{Equation}&[36]
        \omega\tau_{p0}\ll 1\qquad
        \omega\tau_{n0}\ll 1
    \end{Equation}
    那么可以取平方根的近似
    \begin{Equation}&[37]
        \sqrt{1+\j\omega\tau_{p0}}=1+\frac{\j\omega\tau_{p0}}{2}\qquad
        \sqrt{1+\j\omega\tau_{n0}}=1+\frac{\j\omega\tau_{n0}}{2}
    \end{Equation}
    将\xrefpeq{37}代入\xrefpeq{35}
    \begin{Equation}
        Y=\frac{e}{\kB T}\qty[I_{p0}\qty(1+\frac{\j\omega\tau_{p0}}{2})+I_{n0}\qty(1+\frac{\j\omega\tau_{n0}}{2})]
    \end{Equation}
    这样就可以分离实部和虚部了
    \begin{Equation}
        Y=\frac{e}{\kB T}(I_{p0}+I_{n0})+\j\omega\frac{e}{2\kB T}\qty(I_{p0}\tau_{p0}+I_{n0}\tau_{n0})
    \end{Equation}
    或写为
    \begin{Equation}
        Y=g_d+\j\omega C_d
    \end{Equation}
    这里$g_d$就是扩散电导,其与\fancyref{fml:二极管的扩散电导}的结果是一致的
    \begin{Equation}
        g_d=\frac{e}{\kB T}(I_{p0}+I_{n0})=\frac{eI_{DQ}}{\kB T}
    \end{Equation}
    这里$C_d$就是我们最终想要求得的扩散电容
    \begin{Equation}
        C_d=\frac{e}{2\kB T}\qty(I_{p0}\tau_{p0}+I_{n0}\tau_{n0})
    \end{Equation}
    至此,我们就完成了整个计算过程。
\end{Proof}

根据\fancyref{fml:PN结的小信号导纳},我们可以绘制出二极管的小信号等效电路
\begin{Figure}[二极管的简化小信号等效电路]
    \includegraphics{build/Chapter02C_02.fig.pdf}
\end{Figure}
即二极管在小信号下,可以等效为其扩散电阻$r_d$和扩散电容$C_d$的并联。

若要更为完善一些,那么除了扩散电阻$r_d$和扩散电容$C_d$,势垒电容$C_j$也需要被考虑,势垒电容$C_j$应当与扩散电容$C_d$并联。除此之外,实际上PN结在空间电荷区外的中性区域也是有一定的电阻的,称为体电阻$r_s$,因此,PN结的小信号模型还要包含一个串联的体电阻。

\begin{Figure}[二极管的完整小信号等效电路]
    \includegraphics{build/Chapter02C_03.fig.pdf}
\end{Figure}

应指出,当考虑体电阻$r_s$时,需要区分二极管上的总电压$V_\te{app}$和势垒上的电压$V_a$
\begin{Equation}
    V_\te{app}=V_a+I_Dr_s
\end{Equation}
由此可见,当考虑体电阻$r_s$时,维持相同的电流需要更大的外加电压。