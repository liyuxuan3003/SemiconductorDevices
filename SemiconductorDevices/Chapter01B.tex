\section{零偏PN结}
在\xref{sec:PN结的基本结构}中,我们已经考虑了PN结的结构,并简要的讨论了空间电荷区是如何形成的。而在本节,我们主要讨论在无外加激励和无电流存在的热平衡状态下突变结的格子特性。我们将会推导空间电荷区的宽度、电场强度、电势。本节的分析基于以下两个假设
\begin{enumerate}
    \item 第一个假设为玻尔兹曼分布,即每一个半导体区域均为非简并半导体。
    \item 第二个假设为完全电离,即温度对PN结影响可以忽略。
\end{enumerate}

\subsection{内建电势差}
若假设PN结两端没有外加电压偏执,那PN就就处于热平衡状态,这意味着在整个PN结中费米能级唯一定值。\xref{fig:零偏PN结}展示了零偏PN结的能带图。由于P区与N区之间的导带与价带的相对位置随费米能级的变化而变化,所以,在空间电荷区中,导带和价带将发生弯曲。
\begin{Figure}[零偏PN结的能带图]
    \includegraphics[scale=0.8]{build/Chapter01B_01.fig.pdf}
\end{Figure}

电子从N区导带进入P区导带时将遇到一个势垒,这个势垒称为\uwave{内建电势差}(Built-In Potenial),记为$V_{bi}$,而应当指出的是,由于内建电势差$V_{bi}$在N区和P区间维持了多子和少子的平衡,因此内建电势差$V_{bi}$本身并不产生电流。现在的问题是,内建电势差$V_{bi}$等于什么?

\begin{BoxFormula}[内建电势差]
    内建电势差$V_{bi}$可以被表示为
    \begin{Equation}
        V_{bi}=\frac{\kB T}{e}\ln(\frac{N_aN_d}{n_i^2})
    \end{Equation}
\end{BoxFormula}

\begin{Proof}
    在PN结中,本征费米能级至导带和价带的距离是相同的。因此,内建电势差$V_{bi}$可以由本征费米能级$E_{Fi}$在P区和N区间的差确定。这里,我们可以分别定义$\phi_{Fn}$和$\phi_{Fp}$为本征费米能级$E_{Fi}$与费米能级$E_F$在P区和N区内的电势差$(E_{Fi-E_F})/e$,故有
    \begin{Equation}&[1]
        V_{bi}=|\phi_{Fn}|+|\phi_{Fp}|
    \end{Equation}
    在N型区,导带电子浓度由下式给出
    \begin{Equation}&[2]
        n_0=N_c\exp(\frac{E_F-E_c}{\kB T})
    \end{Equation}
    重写为(这里有$n_0=N_d$成立)
    \begin{Equation}&[3]
        n_0=n_i\exp(\frac{E_F-E_{Fi}}{\kB T})=N_d\qquad E_F-E_{Fi}=\kB T\ln\frac{N_d}{n_i}
    \end{Equation}
    在P型区,价带空穴浓度由下式给出
    \begin{Equation}&[4]
        p_0=N_v\exp(\frac{E_v-E_F}{\kB T})
    \end{Equation}
    重写为(这里有$p_0=N_a$成立)
    \begin{Equation}&[5]
        p_0=n_i\exp(\frac{E_{Fi}-E_F}{\kB T})=N_a\qquad E_{Fi}-E_{F}=\kB T\ln\frac{N_a}{n_i}
    \end{Equation}

    根据$\phi_{Fn}$的定义和\xrefpeq{3},在N型区
    \begin{Equation}&[6]
        \phi_{Fn}=\frac{E_{Fi}-E_F}{e}=-\frac{\kB T}{e}\ln\frac{N_d}{n_i}
    \end{Equation}
    根据$\phi_{Fn}$的定义和\xrefpeq{5},在P型区
    \begin{Equation}&[7]
        \phi_{Fp}=\frac{E_{Fi}-E_F}{e}=+\frac{\kB T}{e}\ln\frac{N_a}{n_i}
    \end{Equation}
    将\xrefpeq{6}和\xrefpeq{7}代入\xrefpeq{1},得到
    \begin{Equation}&[8]
        V_{bi}=|\phi_{Fn}|+|\phi_{Fp}|=\frac{\kB T}{e}\qty(\ln\frac{N_d}{n_i}+\ln\frac{N_a}{n_i})
    \end{Equation}
    或
    \begin{Equation}*
        V_{bi}=\frac{\kB T}{e}\ln(\frac{N_aN_d}{n_i^2})\qedhere
    \end{Equation}
\end{Proof}

应指出,若P区或N区是补偿材料,则$N_a$和$N_d$应代表净受主浓度和净施主浓度。

\subsection{空间电荷区的电场和电势}
空间电荷区电场的产生,是由于正负空间电荷的相互分离。\xref{fig:PN结的电场、电势、电荷密度分布}展示了在均匀掺杂及突变近似的情况下电荷密度的分布,我们假设空间电荷区在$x=+x_n$和$x=-x_p$处突然终止。

\begin{Figure}[PN结的电场、电势、电荷密度分布]
    \begin{FigureSub}[电荷密度分布]
        \includegraphics[width=4.8cm]{build/Chapter01B_02.fig.pdf}
    \end{FigureSub}
    \begin{FigureSub}[电场分布]
        \includegraphics[width=4.8cm]{build/Chapter01B_03.fig.pdf}
    \end{FigureSub}
    \begin{FigureSub}[电势分布]
        \includegraphics[width=4.8cm]{build/Chapter01B_04.fig.pdf}
    \end{FigureSub}
\end{Figure}

显然,在这种情况下,空间电荷区的电荷密度服从以下公式。
\begin{BoxFormula}[空间电荷密度]
    均匀掺杂的PN结的空间电荷密度$\rho(x)$为
    \begin{Equation}
        \rho(x)=\begin{cases}
            -eN_a,& -x_p\leq x\leq 0\\[1ex]
            +eN_d,& 0\leq x\leq +x_n
        \end{cases}
    \end{Equation}
\end{BoxFormula}

现在我们的任务是,通过\xref{fml:空间电荷密度}给出的空间电荷密度$\rho(x)$推导电场$\E(x)$和电势$\phi(x)$的公式。推导的关键是,应用泊松方程$\dv*[2]{\phi}{x}=-\rho/\epsilon_s$和电场是电势负梯度$\E=-\dv*{\phi}{x}$。

\begin{BoxFormula}[空间电荷区的电场]
    空间电荷区的电场$\E(x)$为
    \begin{Equation}
        \E(x)=
        \begin{cases}
            \mal{-\frac{eN_a}{\epsilon_s}(x_p+x)},&-x_p\leq x\leq 0\\[4ex]
            \mal{-\frac{eN_d}{\epsilon_s}(x_n-x)},&0\leq x\leq +x_n
        \end{cases}
    \end{Equation}
    其中$\epsilon_s$是半导体的\uwave{电容率}(Permittivity),同时,应有
    \begin{Equation}&[A]
        N_ax_p=N_dx_n
    \end{Equation}
    这表明P区的负电荷与N区域的正电荷的电荷量相同。
\end{BoxFormula}\goodbreak

\begin{Proof}
    半导体内的电场由泊松方程确定
    \begin{Equation}&[1]
        \dv[2]{\phi(x)}{x}=-\frac{\rho(x)}{\epsilon_s}
    \end{Equation}
    考虑到电场是电势的负梯度$\E(x)=-\dv*{\phi(x)}{x}$
    \begin{Equation}&[2]
        \dv{\E(x)}{x}=\frac{\rho(x)}{\epsilon_s}
    \end{Equation}
    在P型区,$\E(x)$由\xrefpeq{2}的积分给出,代入\fancyref{fml:空间电荷密度}
    \begin{Equation}&[3]
        \E(x)=\Int\frac{\rho(x)}{\epsilon_s}\dx=-\Int\frac{eN_a}{\epsilon_s}\dx=-\frac{eN_a}{\epsilon_s}x+C_1
    \end{Equation}
    在N型区,类似的
    \begin{Equation}&[4]
        \E(x)=\Int\frac{\rho(x)}{\epsilon_s}\dx=+\Int\frac{eN_d}{\epsilon_s}\dx=+\frac{eN_d}{\epsilon_s}x+C_2
    \end{Equation}
    电场被假定在空间电荷区外为零,故
    \begin{Equation}&[5]
        \E(-x_p)=0\qquad \E(+x_n)=0
    \end{Equation}
    通过\xrefpeq{5},我们可以确定\xrefpeq{3}和\xrefpeq{4}中的待定常数
    \begin{Equation}
        \qquad
        \E(x)=-\frac{eN_a}{\epsilon_s}(x_p+x),\ -x_p\leq x\leq 0\qquad
        \E(x)=-\frac{eN_d}{\epsilon_s}(x_n-x),\ 0\leq x\leq +x_n
        \qquad
    \end{Equation}
    并考虑到电场$\E(x)$在$x=0$处的连续性
    \begin{Equation}
        \E(0_{-})=\E(0_{+})
    \end{Equation}
    这将给出
    \begin{Equation}
        -\frac{eN_a}{\epsilon_s}x_p=-\frac{eN_d}{\epsilon_s}x_n
    \end{Equation}
    或
    \begin{Equation}*
        N_ax_p=N_dx_n\qedhere
    \end{Equation}
\end{Proof}

\begin{BoxFormula}[空间电荷区的电势]
    空间电荷区的电势$\phi(x)$为
    \begin{Equation}
        \phi(x)=\begin{cases}
            \mal{\frac{eN_a}{2\epsilon_s}(x_p+x)^2},&-x_p<x<0\\[4mm]
            \mal{V_{bi}-\frac{eN_d}{2\epsilon_s}(x_n-x)^2},&0<x<+x_n
        \end{cases}
    \end{Equation}
    其中$V_{bi}$可以被表示为
    \begin{Equation}&[A]
        V_{bi}=\frac{e}{2\epsilon_s}\qty(N_dx_n^2+N_ax_p^2)
    \end{Equation}
\end{BoxFormula}

\begin{Proof}
    继续\xref{fml:空间电荷区的电场}的证明
    \begin{Equation}&[1]
        \dv{\phi}{x}=-\E(x)=
        \begin{cases}
            \mal{\frac{eN_a}{\epsilon_s}(x_p+x)},&-x_p\leq x\leq 0\\[4ex]
            \mal{\frac{eN_d}{\epsilon_s}(x_n-x)},&0\leq x\leq +x_n
        \end{cases}
    \end{Equation}
    在P型区域
    \begin{Equation}&[2]
        \phi(x)=\Int-\E(x)\dx=\Int\frac{eN_a}{\epsilon_s}(x_p+x)\dx=\frac{eN_a}{2\epsilon_s}(x_p+x)^2+C_1'
    \end{Equation}
    在N型区域
    \begin{Equation}&[3]
        \phi(x)=\Int-\E(x)\dx=\Int\frac{eN_d}{\epsilon_s}(x_n-x)=-\frac{eN_d}{2\epsilon_s}(x_n-x)^2+C_2'
    \end{Equation}
    我们不妨将电势参考点选在$x=-x_p$,这样,就有
    \begin{Equation}&[4]
        \phi(-x_p)=0\qquad
        \phi(+x_n)=V_{bi}
    \end{Equation}
    
    \xrefpeq{2}的常量$C_1'$被确定为
    \begin{Equation}&[5]
        C_1'=0
    \end{Equation}
    \xrefpeq{3}的常量$C_2'$被确定为
    \begin{Equation}&[6]
        C_2'=V_{bi}
    \end{Equation}
    这样一来我们,就有
    \begin{Equation}
        \phi(x)=\begin{cases}
            \mal{\frac{eN_a}{2\epsilon_s}(x_p+x)^2},&-x_p<x<0\\[4mm]
            \mal{V_{bi}-\frac{eN_d}{2\epsilon_s}(x_n-x)^2},&0<x<+x_n
        \end{cases}
    \end{Equation}
    电势$\phi(x)$在$x=0$处连续
    \begin{Equation}
        \phi(0_{-})=\phi(0_{+})
    \end{Equation}
    即得
    \begin{Equation}
        \frac{eN_a}{2\epsilon_s}x_p^2=V_{bi}-\frac{eN_d}{2\epsilon_s}x_n^2
    \end{Equation}
    由此得到$V_{bi}$是
    \begin{Equation}*
        V_{bi}=\frac{e}{2\epsilon_s}(N_ax_p^2+N_dx_n^2)\qedhere
    \end{Equation}
\end{Proof}

\xref{fig:电场分布}是空间电荷区的电场分布图像。电场方向是从N区指向P区,或者说沿着$x$轴的负方向。对于均匀掺杂的PN结而言,其电场是距离的线性函数,并且在交界面处达到最大。

\xref{fig:电势分布}是空间电荷区的电势分布图像。电子的电势能由$E(x)=-e\phi(x)$给出,这表明电子的电势能$E(x)$在空间电荷区内也是关于$x$的二次函数。实际上,这种$E(x)$的二次方关系之前在\xref{fig:零偏PN结的能带图}中已经展现了,尽管那时,我们还尚不清楚能带在空间电荷区中具体是如何弯曲的\footnote{能带中的能量,实际上就是电子的电势能}。

\subsection{空间电荷区宽度}
我们可以计算空间电荷区从交界面处延伸入P区与N区内的总距离,这被称为\uwave{空间电荷区宽度}(Space Charge Width)。空间电荷区宽度常记为$W$,很明显,应有$W=x_n+x_p$成立。

\begin{BoxFormula}[零偏时的空间电荷区宽度]
    在热平衡状态下,空间电荷区宽度$W$可以被写作
    \begin{Equation}
        W=x_n+x_p=\qty[\frac{2\epsilon_sV_{bi}}{e}\qty(\frac{N_a+N_d}{N_aN_d})]^{1/2}
    \end{Equation}
    其中,$x_n$和$x_p$等于
    \begin{Gather}[10pt]
        x_n=\qty[\frac{2\epsilon_sV_{bi}}{e}\qty(\frac{N_a}{N_d})\qty(\frac{1}{N_a+N_d})]^{1/2}\\
        x_p=\qty[\frac{2\epsilon_sV_{bi}}{e}\qty(\frac{N_d}{N_a})\qty(\frac{1}{N_a+N_d})]^{1/2}
    \end{Gather}
\end{BoxFormula}
 
\begin{Proof}
    根据\fancyref{fml:空间电荷区的电场}中的\xrefpeq[空间电荷区的电场]{A}
    \begin{Equation}&[1]
        N_ax_p=N_dx_n
    \end{Equation}
    这就得到
    \begin{Equation}&[2]
        x_p=\frac{N_d}{N_a}x_n\qquad
        x_n=\frac{N_a}{N_d}x_p
    \end{Equation}
    根据\xrefpeq[空间电荷区的电势]{A}的\fancyref{fml:空间电荷区的电势}
    \begin{Equation}&[3]
        V_{bi}=\frac{e}{2\epsilon_s}(N_ax_p^2+N_dx_n^2)
    \end{Equation}

    将\xrefpeq{2}代入\xrefpeq{3}
    \begin{Equation}&[4]
        V_{bi}=\frac{e}{2\epsilon_s}\qty(N_dx_n^2+\frac{N_d^2}{N_a}x_n^2)=\frac{e}{2\epsilon_s}\qty(\frac{N_d}{N_a})\qty(N_a+N_d)x_n^2
    \end{Equation}
    解出$x_n$,得到
    \begin{Equation}&[5]
        x_n=\qty[\frac{2\epsilon_sV_{bi}}{e}\qty(\frac{N_a}{N_d})\qty(\frac{1}{N_a+N_d})]^{1/2}
    \end{Equation}
    将\xrefpeq{2}的左半部分代入\xrefpeq{3}
    \begin{Equation}&[6]
        V_{bi}=\frac{e}{2\epsilon_s}\qty(N_ax_p^2+\frac{N_a^2}{N_d}x_p^2)=\frac{e}{2\epsilon_s}\qty(\frac{N_a}{N_d})\qty(N_a+N_d)x_p^2
    \end{Equation}
    解出$x_p$,得到
    \begin{Equation}&[7]
        x_p=\qty[\frac{2\epsilon_sV_{bi}}{e}\qty(\frac{N_d}{N_a})\qty(\frac{1}{N_a+N_d})]^{1/2}
    \end{Equation}
    空间电荷区的总长度$W$是$x_n$和$x_p$的和
    \begin{Equation}&[8]
        W=x_n+x_p=\qty[(x_n^2+x_p^2)+2x_nx_p]^{1/2}
    \end{Equation}  
    将\xrefpeq{5}和\xrefpeq{7}代入\xrefpeq{8},得到
    \begin{Equation}
        W=\qty[\frac{2\epsilon_sV_{bi}}{e}\qty(\frac{N_a^2+N_d^2}{N_aN_d})\qty(\frac{1}{N_a+N_d})+2\frac{2\epsilon_sV_{bi}}{e}\qty(\frac{1}{N_a+N_d})]^{1/2}
    \end{Equation}
    或
    \begin{Equation}*
        W=\qty[\frac{2\epsilon_sV_{bi}}{e}\qty(\frac{N_a+N_d}{N_aN_d})]^{1/2}\qedhere
    \end{Equation}
\end{Proof}