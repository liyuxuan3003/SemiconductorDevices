\chapter{金属半导体和半导体异质结}
在本章中,我们将
\begin{itemize}
    \item 求出金属--半导体结的能带图。
    \item 研究肖特基势垒二极管(整流金属--半导体结)的静电特性。
    \item 研究肖特基势垒二极管的电流--电压关系。
    \item 探讨肖特基势垒二极管和PN结二极管间的电流机理的差别。
    \item 讨论欧姆接触(非整流金属--半导体结)的特性。
    \item 研究半导体异质结。
\end{itemize}

在前面两章中,我们已经讨论了PN结,它们是由同一种半导体材料组成的,通常称为\uwave{同质结}(Homojunction)。在前两章中,我们研究了结的静电作用并得到了结的电流--电压关系。在这一章中,我们将继续讨论由不同材料组成的\uwave{异质结}(Heterojunction)。这将包括\uwave{金属--半导体结}(Metal–Semiconductor Junction)和\uwave{半导体异质结}(Semiconductor Heterojunction)两类。这两种结也可以制成二极管。值得注意的是,金属--半导体结中金属和半导体的接触分为两种情况:高阻的整流接触(类似PN结),低阻的欧姆接触。半导体器件必须与外部电路的金属相连接,而我们希望这种连接不带来额外的整流特性,此时,就需要应用金半欧姆接触。

\section{金属半导体的整流接触}
历史上,最早于20世纪初使用的第一个实用半导体器件其实就是金属--半导体二极管,这种二极管也被称为\uwave{点接触二极管}(Point Contact Diode),是通过将金属须接触裸露的半导体表面形成的。然而,这类金属--半导体二极管,形成难度大,可靠性也不好,因此在20世纪50年代左右被PN结取代。然而,随着半导体技术和真空技术的发展,稳定可靠的金属半导体结得以实现。在章首语中,我们已经提到过,金属--半导体结中的金半接触有两种类型
\begin{itemize}
    \item 高阻的整流接触(Rectifying Contact),此时金半接触具有类似PN结的整流特性。
    \item 低阻的欧姆接触(Ohmic Contact),此时金半接触具有很低的接触电阻,无附加影响。
\end{itemize}

在本节将先讨论金半整流接触,也就是\uwave{肖特基势垒二极管}(Schottky Barrier Diode)。

\subsection{金半整流接触的定性分析}
通常来说,整流接触都是由金属和N型半导体形成的,因此,本节我们会专注于这种情形。

\xref{fig:接触前}展示了金属和N型半导体接触前的能带图,其中,共同的真空能级$E_0$作为一个参考能级,代表了电子从金属或半导体表面逃逸所需达到的能量。除此之外,$e\phi_m$和$e\phi_s$则分别是金属和半导体中电子的\uwave{功函数}(Work Function),\empx{功函数表征了从费米能级至真空能级所需的能量}。不过,由于半导体的费米能级位于禁带中,在费米能级处并不存在电子,功函数没有实际意义,故半导体中更常用的是以$e\chi$表示的\uwave{电子亲合能}(Electron Affinity),\empx{电子亲合能表征了从导带底至真空能级所需的能量}。而功函数与电子亲合能的差$\phi_s-\chi$则记为$\phi_n$。

\begin{Figure}[金半整流接触的能带图]
    \begin{FigureSub}[接触前]
        \includegraphics[scale=0.8]{build/Chapter03A_01.fig.pdf}
    \end{FigureSub}
    \hspace{0.5cm}
    \begin{FigureSub}[接触后]
        \includegraphics[scale=0.8]{build/Chapter03A_02.fig.pdf}
    \end{FigureSub}
\end{Figure}

这里需要说明的是电势量的命名规律,通常以费米能级$E_F$为下界的电势量均会以$\phi$命名,包括$\phi_m,\phi_s,\phi_n$和后面的$\phi_{B0}$等,而其他的电势量则用其他的名称,包括$\chi$和后面的$V_{bi}$等。

\begin{BoxDefinition}[金属的功函数]
    金属的功函数$e\phi_m$,定义为费米能级至真空能级所需的能量
    \begin{Equation}
        e\phi_m=E_0-E_F
    \end{Equation}
\end{BoxDefinition}

\begin{BoxDefinition}[半导体的功函数]*
    半导体的功函数$e\phi_s$,定义为费米能级至真空能级所需的能量
    \begin{Equation}
        e\phi_s=E_0-E_F
    \end{Equation}
    半导体的电子亲和能$e\chi$,定义为导带底至真空能级所需的能量
    \begin{Equation}
        e\phi_m=E_0-E_c
    \end{Equation}
    而两者的差$\phi_s-\chi$则定义为$\phi_n$,其中$e\phi_n$代表导带底和费米能级的能量差
    \begin{Equation}
        \phi_n=\phi_s-\chi\qquad e\phi_n=E_c-E_F
    \end{Equation}
\end{BoxDefinition}

在\xref{tab:部分金属的功函数}和\xref{tab:部分半导体的电子亲合能}中,列出了部分金属的$\phi_m$和部分半导体的$\chi$。

\begin{Tablex}[部分金属的功函数]{XX}
<金属&金属的功函数$\phi_m$\\>
银\xce{Ag}&4.26\\ 
铝\xce{Al}&4.28\\ 
金\xce{Au}&5.1\\ 
铬\xce{Cr}&4.5\\ 
钼\xce{Mo}&4.6\\ 
镍\xce{Ni}&5.15\\ 
钯\xce{Pd}&5.12\\ 
铂\xce{Pt}&5.65\\ 
钛\xce{Ti}&4.33\\ 
钨\xce{W}&4.55\\ 
\end{Tablex}

\begin{Tablex}[部分半导体的电子亲合能]{XX}
<半导体&半导体的功函数$\chi$\\>
锗\xce{Ge}&4.13\\
硅\xce{Si}&4.01\\
砷化镓\xce{GaAs}&4.07\\
砷化铝\xce{AlAs}&3.5\\
\end{Tablex}

值得注意的是,在\xref{fig:接触前}中,我们假定了$\phi_m>\phi_s$,即,金属的功函数高于半导体的功函数,或者说,金属的费米能级比半导体低。实际上,$\phi_m$和$\phi_s$的关系将决定金半接触的类型
\begin{itemize}
    \item 若$\phi_m>\phi_s$,半导体费米能级较高,产生整流接触(接触后能带向上弯曲)。
    \item 若$\phi_m<\phi_s$,半导体费米能级较低,产生欧姆接触(接触后能带向下弯曲)。
\end{itemize}
由于本节考虑整流接触,故本节总是默认$\phi_m>\phi_s$成立。

\xref{fig:接触后}展示了金属和N型半导体接触后的能带图,由于接触前,半导体的费米能级在金属之上,为了使系统具有统一的常量,两者接触后,半导体表面附近的电子将流入能量更低的金属中,半导体表面的能带将向上弯曲,带正电的施主原子留在半导体中,形成空间电荷区。

\begin{BoxDefinition}[金半接触的肖特基势垒]
    金半接触时,\uwave{肖特基势垒}(Schottky Barrier)是指电子从金属进入半导体所需的能量
    \begin{Equation}
        e\phi_{B0}=(E_{c})_\te{边界}-E_F
    \end{Equation}
    或者可以表示为
    \begin{Equation}
        \phi_{B0}=\phi_m-\chi
    \end{Equation}
\end{BoxDefinition}

\begin{BoxDefinition}[金半接触的内建电势差]
    金半接触时,内建电势差是指电子从半导体进入金属所需的能量
    \begin{Equation}
        eV_{bi}=(E_c)_\te{边界}-E_c
    \end{Equation}
    或者可以表示为
    \begin{Equation}
        eV_{bi}=\phi_m-\chi-\phi_n=\phi_{B0}-\phi_n
    \end{Equation}
\end{BoxDefinition}

简单来说,金半接触时
\begin{itemize}
    \item 电子从金属进入半导体的势垒,称为肖特基势垒,以$\phi_{B0}$或$e\phi_{B0}$表示。
    \item 电子从半导体进入金属的势垒,称为内建电势差,以$V_{bi}$或$eV_{bi}$表示,类似PN结的。
\end{itemize}

现在我们已经比较清楚金半整流接触的能带图了,那么当外加偏置电压时,金半整流接触的能带图将如何变化呢?首先要弄清的是,金半接触时“正向”和“反向”是如何定义的,在整流接触中,金属为正极,N型半导体为负极,换言之,金属取代了PN结的P型半导体。事实是,当外加电压时,外加电压完全落在半导体侧势垒上,金属侧势垒将保持$\phi_{B0}$不变



\begin{itemize}
    \item 当外加正向电压$V_a$时,由于半导体侧势垒由$V_{bi}$降低至$V_{bi}-V_a$,半导体至金属的电子运动更容易了,半导体至金属的电子流将占优势,形成正向电流。同时,由于越大的正向电压意味着半导体侧势垒降低的越多,\empx{正向电流会随正向电压的增大而指数级增大}。
    \item 当外加反向电压$V_R$时,由于半导体侧势垒由$V_{bi}$升高至$V_{bi}+V_R$,这将导致半导体至金属的电子流逐渐减小至零,此时,金属至半导体的电子流将占优势,形成反向电流。反向电流的的大小取决于金属侧势垒$\phi_{B0}$的高度,但由于金属侧势垒$\phi_{B0}$不随外加电压变化,金属至半导体的电流是恒定的,换言之,\empx{反向电流将随反向电压增大趋于定值}。
\end{itemize}

由此可见,金属半导体的整流接触具有与PN结相似的整流特性。

\begin{Table}[金半整流接触外加偏置电压时的能带变化]{|c|c|c|}
    \xcell<c>[1em][0em]
    {\includegraphics[width=4.5cm]{build/Chapter03A_03.fig.pdf}}&
    \xcell<c>[1em][0em]
    {\includegraphics[width=4.5cm]{build/Chapter03A_05.fig.pdf}}&
    \xcell<c>[1em][0em]
    {\includegraphics[width=4.5cm]{build/Chapter03A_04.fig.pdf}}\\
    零偏&正偏(正向电压$V_a$)&反偏(反向电压$V_R$)\\
\end{Table}

\subsection{金半整流接触的理想静电特性}
金半整流接触和PN结是类似的,两者均有均匀掺杂且适用突变近似的空间电荷区,因此,金半接触包括“电场分布、电场分布、空间电荷区宽度、势垒电容”在内的静电特性都可以直接从PN结的相关结论中得来(\xref{fml:空间电荷区的电场}、\xref{fml:空间电荷区的电势}、\xref{fml:反偏时的空间电荷区宽度}、\xref{fml:二极管的势垒电容}),唯一的一项差异在于,金半接触的空间电荷区仅分布在N型半导体,金属中没有空间电荷区,故,金半接触相当于P$^{+}$N结,适用$x_n\gg x_p, W=x_n$以及$N_d\ll N_a$即$N_aN_d/(N_a+N_d)=N_d$的近似。

\begin{BoxFormula}[金半整流接触的电场分布]
    金半接触的电场分布满足
    \begin{Equation}
        \E(x)=-\frac{eN_d}{\varepsilon_s}(x_n-x)
    \end{Equation}
\end{BoxFormula}

\begin{BoxFormula}[金半整流接触的电势分布]
    金半接触的电势分布满足
    \begin{Equation}
        \phi(x)=V_{bi}-\frac{eN_d}{2\varepsilon_s}(x_n-x)^2
    \end{Equation}
\end{BoxFormula}

\begin{BoxFormula}[金半整流接触的空间电荷区宽度]
    金半整流接触的空间电荷区宽度满足
    \begin{Equation}
        W=x_n=\qty[\frac{2\varepsilon_s(V_{bi}+V_R)}{eN_d}]^{1/2}
    \end{Equation}
\end{BoxFormula}

\begin{BoxFormula}[金半整流接触的势垒电容]
    金半整流接触的势垒电容满足
    \begin{Equation}
        C_j=A\qty[\frac{e\varepsilon_sN_d}{2(V_{bi}+V_R)}]^{1/2}
    \end{Equation}
\end{BoxFormula}

\subsection{影响势垒高度的非理想因素}

有一些非理想因素会影响肖特基势垒的高度,包括镜像力和表面态。

\subsubsection{镜像力}
金属具有静电感应的特性,当电子在空间电荷区中运动时,除了受到空间电荷区中电荷的吸引外,还会在金属表面感应出正电荷。空间电荷区的电势我们在\xref{subsec:金半整流接触的理想静电特性}已经用泊松方程计算出来了,然而,金属表面感应正电荷产生的电势却尚没有纳入考虑。但问题在于,金属表面的电荷对电子的吸引该如何计算呢,这是比较复杂的。不过所幸的是,有一些简单的办法可以 解决这个问题。\uwave{镜像法}(Method of Images)指出\cite{wiki:镜像法},\empx{金属表面的感应电荷对点电荷的吸引力,等同于将金属视为一面镜子,镜子中点电荷的镜像电荷对点电荷的吸引力}。换言之,若我们要分析金属表面的感应电荷对电子的吸引,只要假象金属是一面镜子,镜子中有一个电子的\uwave{镜像电荷}(Image Charge),即一个位置上镜面对称但电荷相反(带正电)的点电荷,并考虑镜像电荷对电子的\uwave{镜像力}(Image Force),由此就将表面电荷的作用转换为点电荷的作用。

那么,镜像力将对金半整流接触的势垒发生什么影响呢?
\begin{BoxFormula}[镜像力的势垒降低]
    镜像力将使金半整流接触的势垒高度降低,且势垒的最高点向半导体侧内移。

    势垒的降低量为(对于电势而言是增加量)
    \begin{Equation}
        \delt{\phi}=\frac{1}{4}\sqrt[\uproot{16}\leftroot{-2}4]{\frac{2e^3N_d(V_{bi}+V_R)}{\pi^2\varepsilon_s^3}}
    \end{Equation}
    势垒的最高点内偏,现在位于(原先位于$x=0$处)
    \begin{Equation}
        x_{m}=\frac{1}{4(\pi N_dx_d)^{1/2}}
    \end{Equation}
\end{BoxFormula}

\begin{Proof}
    设电子位于$x$处,那么镜像电荷将位于$-x$处,依照库伦作用,吸引力为
    \begin{Equation}&[1]
        F=-\frac{1}{4\pi\varepsilon_s}\frac{e^2}{(2x)^2}=-\frac{e^2}{16\pi\varepsilon_s}\frac{1}{x^2}
    \end{Equation}
    而另外一方面,电场力和电场的关系是
    \begin{Equation}&[2]
        F=-eE
    \end{Equation}
    镜像电荷产生的电场$\E_M(x)$就为
    \begin{Equation}&[3]
        \E_M(x)=\frac{e}{16\pi\varepsilon_s}\frac{1}{x^2}
    \end{Equation}
    镜像电荷产生的电势$\phi_M(x)$可通过对$\E_M(x)$积分得到
    \begin{Equation}&[4]
        \phi_M(x)=-\Int\E(x)\dx=\frac{e}{16\pi\varepsilon_s}\Int\frac{1}{x^2}\dx=\frac{e}{16\pi\varepsilon_s}\frac{1}{x}+C
    \end{Equation}
    此处定$\phi_M(x)$在无穷远处的电势为零,故
    \begin{Equation}&[5]
        \phi_M(x)=\frac{e}{16\pi\varepsilon_s}\frac{1}{x}
    \end{Equation}
    原先的电势改记为$\phi_0(x)$,根据\fancyref{fml:金半整流接触的电势分布}
    \begin{Equation}&[6]
        \phi_0(x)=\begin{cases}
            \mal{V_{bi}-\frac{eN_d}{2\varepsilon_s}(x_n-x)^2}, &0<x<x_n\\
            \mal{V_{bi}}, &x\geq x_n
        \end{cases}
    \end{Equation}
    将\xrefpeq{5}和\xrefpeq{6}相加,得到总电势$\phi(x)$
    \begin{Equation}&[7]
        \phi(x)=\phi_0(x)+\phi_M(x)=\begin{cases}
            \mal{V_{bi}+\frac{e}{16\pi\varepsilon_s}\frac{1}{x}-\frac{eN_d}{2\varepsilon_s}(x_n-x)^2}, &0<x<x_n\\[5mm]
            \mal{V_{bi}+\frac{e}{16\pi\varepsilon_s}\frac{1}{x}}, &x\geq x_n
        \end{cases}
        \hspace{-0.2cm}
    \end{Equation}
    原先的电场改记为$\E_0(x)$,根据\fancyref{fml:金半整流接触的电场分布}
    \begin{Equation}&[8]
        \E_0(x)=-\frac{eN_d}{\varepsilon_s}(x_n-x)
    \end{Equation}
    镜像电荷的电场$\E_M(x)$已经在\xrefpeq{3}中求出来了
    \begin{Equation}&[9]
        \E_M(x)=\frac{e}{16\pi\varepsilon_s}\frac{1}{x^2}
    \end{Equation}
    镜像电荷对电势的影响,关键在于求出\xrefpeq{7}中电势函数$\phi(x)$的极小值的出现位置$x_m$和极小电势$\delt{\phi}$(这里$\delt{\phi}$是相较$\phi_0(x)$的极小值点而言的,而$\phi_0(x)$的极小值为零),我们愿意相信$x_m\ll x_n$,因此,确定电势的极小值点的位置$x_m$,只需要分析电场$\E_0(x), \E_M(x)$在何处相等即可,因此,我们联立\xrefpeq{8}和\xrefpeq{9},构成方程$\E_0(x_m)=\E_M(x_m)$
    \begin{Equation}&[10]
        \frac{e}{16\pi\varepsilon_s}\frac{1}{x_m^2}=-\frac{eN_d}{\varepsilon_s}(x_n-x_m)
    \end{Equation}
    这是一个关于$x_m$的三次方程,并不好解,援引$x_m\ll x_n$的近似
    \begin{Equation}&[11]
        \frac{e}{16\pi\varepsilon_s}\frac{1}{x_m^2}=-\frac{eN_d}{\varepsilon_s}x_n
    \end{Equation}
    容易得出
    \begin{Equation}&[12]
        x_m^2=\frac{1}{16\pi N_dx_n}
    \end{Equation}
    即
    \begin{Equation}&[13]
        x_m=\frac{1}{4(\pi N_dx_n)^{1/2}}
    \end{Equation}
    现在代入计算$\delt{\phi}=\phi(x_m)$,由于我们确信$x_m\ll x_n$,故\xrefpeq{7}的$\phi(x)$可以仅取第一部分
    \begin{Equation}&[14]
        \phi(x)=V_{bi}+\frac{e}{16\pi\varepsilon_s}\frac{1}{x}-\frac{eN_d}{2\varepsilon_s}(x_n-x)^2
    \end{Equation}
    但为了简化结果,这里代入时忽略$\phi(x)$有关镜像力的项,即回归$\phi_0(x)$
    \begin{Equation}&[15]
        \phi(x)=V_{bi}-\frac{eN_d}{2\varepsilon_s}(x_n-x)^2
    \end{Equation}
    这就得到
    \begin{Equation}&[16]
        \delt{\phi}=V_{bi}-\frac{eN_d}{2\varepsilon_s}(x_n-x_m)^2
    \end{Equation}
    展开
    \begin{Equation}&[17]
        \delt{\phi}=V_{bi}-\frac{eN_d}{2\varepsilon_s}(x_n^2-2xnx_m+x_m^2)
    \end{Equation}
    依据\fancyref{fml:空间电荷区的电势}
    \begin{Equation}&[18]
        V_{bi}=\frac{e}{2\varepsilon_s}x_n^2
    \end{Equation}
    运用\xrefpeq{18}就可以将\xrefpeq{17}简化为
    \begin{Equation}&[19]
        \delt{\phi}=\frac{eN_d}{2\varepsilon_s}\qty(2x_nx_m-x_m^2)
    \end{Equation}
    运用$x_m\ll x_n$忽略二次项$x_m^2$
    \begin{Equation}
        \delt{\phi}=\frac{eN_d}{\varepsilon_s}x_nx_m
    \end{Equation}
    代入\xrefpeq{13}给出的$x_m$
    \begin{Equation}
        \delt{\phi}=\frac{1}{4}\frac{eN_d}{\varepsilon_s}x_n\sqrt{\frac{1}{\pi N_dx_n}}
    \end{Equation}
    将各项整合到根号下
    \begin{Equation}
        \delt{\phi}=\frac{1}{4}\sqrt{\frac{e^2N_dx_n}{\pi\varepsilon_s^2}}
    \end{Equation}
    有趣的是我们还可以将$x_n$代入,根据\fancyref{fml:金半整流接触的空间电荷区宽度}
    \begin{Equation}
        \delt{\phi}=\frac{1}{4}\sqrt{\frac{e^2N_d}{\pi\varepsilon_s^2}\sqrt{\frac{2\varepsilon_s(V_{bi}+V_R)}{eN_d}}}
    \end{Equation}
    将各项整合到内层根号,并合并内外层根号
    \begin{Equation}
        \delt{\phi}=\sqrt[\uproot{16}\leftroot{-2}4]{\frac{2e^4N_d(V_{bi}+V_R)}{\pi^2\varepsilon_s^3}}
    \end{Equation}
    至此,我们就完成了所有的计算。
\end{Proof}

镜像力对电势曲线的影响如\xref{fig:镜像力对电势曲线的影响}所示,其中标注了$\delt{\phi}$和实际近似计算的$\delt{\phi}$的位置。

\begin{Figure}[镜像力对电势曲线的影响]
    \includegraphics{build/Chapter03A_06.fig.pdf}
\end{Figure}

镜像力的影响反映在能带图上的效果如\xref{tab:镜像力对势垒的影响}所示,其实际就是将\xref{fig:镜像力对电势曲线的影响}上下颠倒了一下(电势能与电势相差$-e$倍)。由\xref{tab:镜像力对势垒的影响}可以看出,
\empx{镜像力削弱了势垒并使势垒向半导体侧发生偏移}。

\begin{Table}[镜像力对势垒的影响]{|c|c|}
    \xcell<c>[1em][0em]
    {\includegraphics[width=4.5cm]{build/Chapter03A_03.fig.pdf}}&
    \xcell<c>[1em][0em]
    {\includegraphics[width=4.5cm]{build/Chapter03A_07.fig.pdf}}\\
    理想状态&镜像力的影响\\
\end{Table}

镜像力对电势的影响$\delt{\phi}$其实还是比较微弱的,随着反向电压$V_R$的增大缓慢增大。
