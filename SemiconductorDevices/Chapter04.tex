\chapter{双极结型晶体管}

在本章中,我们将
\begin{itemize}
    \item 讨论双极晶体管的物理结构。
    \item 讨论双极晶体管的工作原理和可能的工作方式。
    \item 推导双极晶体管在各种工作方式下少子浓度的表达式。
    \item 推导双极晶体管中各个电流分量的表达式。
    \item 定义共基极和共射极电流增益。
    \item 定义限制因素,分析共基极电流增益的构成。
    \item 讨论双极晶体管的集中非理想效应。
    \item 讨论双极晶体管的等效电路。
    \item 定义并推导频率限制因素的表达式。
\end{itemize}

\uwave{晶体管}(Transistor)是一种与其他电路元件结合使用时,可以产生电流增益、电压增益、信号功率增益的多结半导体器件,因此,晶体管也被称为\uwave{有源器件}(Active Device),而作为对比,二极管和电阻电容等被称为\uwave{无源器件}(Passive Device)。晶体管,无论其类型,其最基本的工作方式都是:\empx{通过在晶体管各极间的电压控制流经晶体管的电流}。只不过原理不同罢了。

晶体管有诸多类型,但主要可以分为\uwave{双极结型晶体管}(Bipolar Junction Transistor, BJT)和\uwave{场效应晶体管}(Field Effect Transistor, FET)。后者可以进一步分为JFET和MOSFET,这两者将在后面两章中相继介绍。本章,我们将先集中研究BJT,也就是双极晶体管的原理。

\section{BJT的基本结构}

\subsection{BJT的结构}

BJT的结构如\xref{fig:BJT的结构简图}所示,其具有三个性质不同的掺杂区,包含两个PN结。



\begin{Figure}[BJT的结构简图]
    \begin{FigureSub}[NPN的结构简图]
        \includegraphics[scale=0.87]{build/Chapter04A_01.fig.pdf}
    \end{FigureSub}
    \hspace{0.2cm}
    \begin{FigureSub}[PNP的结构简图]
        \includegraphics[scale=0.87]{build/Chapter04A_02.fig.pdf}
    \end{FigureSub}
\end{Figure}

BJT可以分为NPN和PNP两种类型,我们主要以NPN型为例研究

\begin{itemize}
    \item 对于NPN型,三个掺杂区依次为N型、P型、N型,即两个共阳极的PN结。
    \item 对于PNP\hspace{0.4em}型,三个掺杂区依次为P型、N型、P型,\hspace{0.1em}即两个共阴极的PN结。
\end{itemize}

BJT的三个区域依次为:\uwave{发射极}(Emitter, E)、\uwave{基极}(Base, B)、\uwave{集电极}(Collector, C)。但是,由于原理上的需要,这三个区域的性质是不对称的,这包含区域的掺杂浓度和宽度两方面
\begin{itemize}
    \item 就掺杂浓度而言,有$\te{E}>\te{B}>\te{C}$,如\xref{fig:NPN的结构简图},三者用$n^{++}, p^{+},n$表示。通常而言,三者的典型掺杂浓度是$\si{10^{19} cm^{-3}}$、$\si{10^{17} cm^{-3}}$、$\si{10^{15} cm^{-3}}$,其中尤为重要的是发射区重掺。
    \item 就区域宽度而言,有$\te{C}>\te{E}>\te{B}$,其中尤为重要的是基区的宽度必须非常窄。
\end{itemize}
由此可见,BJT在结构上并不对称,发射极和集电极是不对等的,不可以交换。

BJT中,BE结称为\uwave{发射结}(Emitter Junction),BC结称为\uwave{集电结}(Collector Junction)。我们后面会看到,取决于这两个结是正偏和反偏,将最终给出BJT的四种工作状态。但眼下我们不需要了解这么多,我们只需要知道,在BJT最常用的工作区,即正向放大区下,两个结的工作状态是:\empx{发射结正偏,集电结反偏}。发射结发挥了重要的作用,因此,如\xref{fig:BJT的电路符号}中,我们用箭头标识发射结的正偏方向,有箭头一侧的为E,无箭头的一侧为B,而箭头的方向能帮助我们区分该BJT到底是NPN还是PNP,由E至B即为NPN,由B至E即为PNP。

\begin{Figure}[BJT的电路符号]
    \begin{FigureSub}[NPN的电路符号]
        \includegraphics{build/Chapter04A_09.fig.pdf}
    \end{FigureSub}
    \hspace{1cm}
    \begin{FigureSub}[PNP的电路符号]
        \includegraphics{build/Chapter04A_10.fig.pdf}
    \end{FigureSub}
\end{Figure}

然而,我们要说明的是,\xref{fig:BJT的结构简图}只是一个简化示意图。实际的结构要复杂很多。如\xref{fig:BJT的结构}所示,首先,为了将BJT作在平面上,各区域并不是简单的线性排列。其次,集电极的主体尽管是轻掺的$n$型,但是,和金属接触处仍需要$n^{++}$重掺以构成欧姆接触。最后,底部为降低半导体的电阻,还需添加$n^{++}$重掺的掩埋层。不过\xref{fig:BJT的结构简图}对于的简图理论学习仍然是很有帮助的。

\begin{Figure}[BJT的结构]
    \begin{FigureSub}[NPN的结构]
        \includegraphics[scale=0.87]{build/Chapter04A_03.fig.pdf}
    \end{FigureSub}
    \hspace{0.2cm}
    \begin{FigureSub}[PNP的结构]
        \includegraphics[scale=0.87]{build/Chapter04A_04.fig.pdf}
    \end{FigureSub}
\end{Figure}

\subsection{BJT的原理}
现在的问题是,人们总是声称BJT可以放大,然而BJT到底如何工作?这里先来做一个简要的讨论。如\xref{fig:BJT的核心原理}所示,其中,$V_{BE}$和$V_{BC}$表示发射结和集电结的偏压,但由于我们更关心放大区,故用$+V_{BE}$和$-V_{BC}$标识“发射结正偏压”和“集电结负偏压”。$I_{E},I_{B},I_C$则是各极的电流。好!现在让我们先忘掉集电区的存在,只考虑发射结,很明显发射结不过是一个我们很熟悉的正偏PN结,从基区至发射区将有较大的正偏电流,且显然有$I_E=I_B$成立。由于发射结中,相对而言,发射区是$n^{++}$重掺的,因此,发射结的电流中主要以发射区至基区的电子电流为主。而引入集电区的存在后,一切都改变了。从表面上看,集电结不过是一个简单的反偏PN结,照道理基区和集电区之间应该是“不导通”的。问题是,反偏电流很小的原因仅仅在于少子很少,并没有什么实际阻碍,只要能补充少子,那么反偏电流同样可以很大。

现在让我们联合起来考虑NPN的特性:\empx{发射结被设计为重掺},发射结正偏时向基区输送了巨量电子,这些电子原本应通过基极被导出,然而,\empx{基区被设计的很窄},电子中只有很少的一部分会从基极流出,电子大部分都迅速穿过了整个基极到达了边界,这为反偏集电结提供了大量电子。在边界上,集电结反偏产生的强电场将这些电子扫入集电区,产生很大的反偏电流。

\begin{Figure}[BJT的核心原理]
    \includegraphics{build/Chapter04A_11.fig.pdf}
\end{Figure}

总结起来,在BJT中,正偏发射结的正偏电流$I_E$的电子只有很少的部分流向$I_B$,大部分都继续流向了集电区形成$I_C$,原因是,反偏集电结的强电场从基区中“窃取”了大部分电子流。
\begin{Equation}
    I_{E}\gg I_{B}\qquad I_{E}\approx I_C
\end{Equation}
这个比例是相当悬殊的,通常我们记
\begin{Equation}
    I_C=\beta I_B
\end{Equation}
这里的$\beta$通常可以达到50至200,是一个相当大的电流增益,这就是BJT的放大原理!简而言之,BJT可以将基极电流$I_B$放大将近百倍形成集电极电流$I_C$,以小电流控制大电流。

这里我们看到,“发射结”和“集电结”这两个名称是非常恰当的
\begin{itemize}
    \item 发射结向基区“发射”电子,形成正偏电流$I_E$。
    \item 集电结从基区“收集”电子,形成反偏电流$I_C$。
\end{itemize}
有关电流的朝向可能会让我们困惑,在\xref{fig:BJT的核心原理}中,我们可能会觉得是$I_E$是由$I_C$和$I_B$汇聚而成的。但通过上述讨论我们知道,电子流在这里发挥主要作用,电子流的方向与电流相反,因此从电子流的观点看,实质上是$I_E$的电子流分散为$I_C$和$I_B$。这种矛盾并不是特别要紧,我们可以“谁产生谁”或“谁导致谁”的因果思维直观理解物理过程,但是,这之后,我们要始终牢记物理的本质是数量关系而非因果。这里唯一真实的是$I_E=I_{B}+I_C$而不是汇聚或分散。


% 最后,让我们讨论一些一般性的东西。实际上,包括BJT、JFET、MOSFET在内的任何晶体管,从形式上而言都是两个方向相反的PN结,但由于这两个相反的PN结在结构上通过某种方式“紧密联系”起来,其所连接的两端(BJT称射集E,C,FET称源漏S,D)可以在第三端(BJT称基B,FET称栅G)的控制下导通,具有完全不同于孤立相反PN结的性质。

% 而所谓晶体管的类型,无非就是这种“紧密联系”的方式方法不同,

% \begin{Figure}[晶体管的一般结构]
%     \begin{FigureSub}[N型器件]
%         \includegraphics[scale=0.87]{build/Chapter04A_05.fig.pdf}
%     \end{FigureSub}
%     \hspace{0.2cm}
%     \begin{FigureSub}[N型器件]
%         \includegraphics[scale=0.87]{build/Chapter04A_06.fig.pdf}
%     \end{FigureSub}
% \end{Figure}

\subsection{BJT的能带}
令人惊讶的是,尽管BJT的原理比较费解,但是BJT的能带图却非常简单。

\begin{Figure}[BJT的能带]
    \begin{FigureSub}[截止区]
        \includegraphics[scale=0.9]{build/Chapter04A_07.fig.pdf}
    \end{FigureSub}
    \begin{FigureSub}[放大区]
        \includegraphics[scale=0.9]{build/Chapter04A_08.fig.pdf}
    \end{FigureSub}
\end{Figure}

在\xref{fig:截止区}和\xref{fig:放大区}中,分别展示了BJT在零偏和处于正向放大区(发射结正偏,集电结反偏)时的图像,我们看到,两者不过就是两个处于相应偏置的PN结的能带图的拼接罢了。

\subsection{BJT的四个工作区}
本节最后论述一下BJT的四个工作区。截止区是易于理解的,发射结反偏,集电结反偏,我们不可能指望从中得到任何东西,整个BJT处于关断状态。正向放大区前面在\xref{subsec:BJT的原理}中已经做了充分的论述,发射结正偏向基区注入电子,集电结反偏从基区抽取电子。饱和区则是很有意思的一个概念,关于“饱和区”中“饱和”的含义是一个常见的争议,它和MOSFET的饱和区完全不同,并非是指什么电流饱和了。而是指\cite{se:BJT饱和},当BJT由正向放大区转入饱和区时,集电结由反偏转入正偏,此时,集电结的耗尽区电场逐渐减弱,以至于集电结已经不足矣将基区大部分的电子抽取出来了。也就是说,BJT的饱和区,\empx{饱和的是集电结的集电能力}!
\begin{Figure}[BJT的四个工作区]
    \includegraphics{build/Chapter04A_12.fig.pdf}
\end{Figure}

除此之外,反向放大区和正向放大区的关系也是很有意思的,两者刚好相反
\begin{itemize}
    \item 正向放大区中,发射结正偏注入电子,集电结反偏抽取电子。
    \item 反向放大区中,集电结正偏抽取电子,发射结反偏抽取电子。
\end{itemize}
也就是说,反向放大区完全就是将正向放大区的模式颠倒过来,对调了发射结和集电结的功能。前面我们提到,BJT的结构并不对称,但平心而论,\empx{BJT的结构组成确实是对称的},然而,\empx{BJT的结构参数则是非对称的},现有的BJT的结构设计都是基于正向放大的使用的,例如发射结重掺就是为了能发射更多的电子至基区以提高增益$\beta$。因此,反向放大区从理论上确实能以相仿的原理工作,但是电流增益$\beta$等特性参数将很差。所以,不要使用反向放大区!


\section{BJT的少子分布}

我们很感兴趣的是计算BJT中的电流,而这一点上BJT和\xref{chap:PN结的电流特性}中的PN结一致,扩散电流是少子浓度分布的梯度,因此,若要计算扩散电流,就要求我们先计算出少子的浓度分布。

我们先研究最重要的,即BJT在正向放大区下的少子分布。\xref{fig:BJT在正向放大区的少子分布}先给出了正向放大区少子分布的图像,在展开数学计算前,我们先做一些说明和约定。首先,BJT的少子分布的计算其实并不比PN结复杂,仍然是求解扩散方程,只不过各区域的边界条件会有些不同。其次,为便于计算,我们为BJT的每个区域建立独立的坐标体系,基区B、发射区E、集电区C的横坐标分别记为$x, x', x''$,分别是向右(自E向C)、向左(向外)、向右(向外)。这三个区域的少子分别为电子、空穴、空穴,故三者的少子浓度分别以$n_B(x),\ p_E(x'),\ p_{C}(x'')$表示。

\begin{Figure}[BJT在正向放大区的少子分布]
    \includegraphics[scale=1.2]{build/Chapter04B_01a.fig.pdf}
\end{Figure}

\subsection{放大区BJT的基区少子分布}
\begin{BoxFormula}[放大区BJT的基区少子分布]
    当BJT处于放大区时,其基区过剩少子浓度$\fdd{n_B}(x)$满足
    \begin{Equation}
        \qquad
        \fdd{n_B}(x)=\frac{n_{B0}}{\sinh(x_B/L_B)}\qty\Bigg{\qty[\exp(\frac{eV_{BE}}{\kB T})-1]\sinh(\frac{x_B-x}{L_B})-\sinh(\frac{x}{L_B})}
        \qquad
    \end{Equation}
    当$x_B\ll L_B$时,通过双曲正弦的近似,得到
    \begin{Equation}
        \fdd{n_B}(x)=\frac{n_{B0}}{x_B}\qty\Bigg{\qty[\exp(\frac{eV_{BE}}{\kB T})-1](x_B-x)-x}
    \end{Equation}
\end{BoxFormula}

\begin{Proof}
    在基区,扩散方程的形式为
    \begin{Equation}&[1]
        D_B\pdv[2]{\fdd{n_B(x)}}{x}-\frac{\fdd{n_B}(x)}{\tau_{B0}}=0
    \end{Equation}
    通解是我们熟悉的,其中$L_B=\sqrt{D_B\tau_{B0}}$
    \begin{Equation}&[2]
        \fdd{n_B(x)}=A\exp(\frac{+x}{L_B})+B\exp(\frac{-x}{L_B})
    \end{Equation}
    左侧边界,发射结正偏,因此
    \begin{Equation}&[3]
        \fdd{n_B}(0)=n_{B0}\qty[\exp(\frac{eV_{BE}}{\kB T})-1]
    \end{Equation}
    右侧边界,集电结反偏,因此
    \begin{Equation}&[4]
        \fdd{n_B}(x_B)=-n_{B0}
    \end{Equation}
    将\xrefpeq{3}和\xrefpeq{4}代入\xrefpeq{2}
    \begin{Equation}&[5]
        \begin{pmatrix}
            1&1\\
            \exp(+x_B/L_B)&\exp(-x_B/L_B)\\
        \end{pmatrix}
        \begin{pmatrix}
            A\\
            B\\
        \end{pmatrix}
        =
        \begin{pmatrix}
            n_{B0}\qty[\exp(eV_{BE}/\kB T)-1]\\
            -n_{B0}
        \end{pmatrix}
    \end{Equation}
    计算$D$
    \begin{Equation}&[6]
        D=\exp(-x_{B}/L_B)-\exp(+x_{B}/L_B)=-2\sinh(x_B/L_B)
    \end{Equation}
    计算$D_A$
    \begin{Equation}&[7]
        D_A=+n_{B0}\qty[\exp(eV_{BE}/\kB T)-1]\exp(-x_B/L_B)+n_{B0}
    \end{Equation}
    计算$D_B$
    \begin{Equation}&[8]
        D_B=-n_{B0}\qty[\exp(eV_{BE}/\kB T)-1]\exp(+x_B/L_B)-n_{B0}
    \end{Equation}
    故$A$为
    \begin{Equation}&[9]
        A=\frac{D_A}{D}=-\frac{n_{B0}\qty[\exp(eV_{BE}/\kB T)-1]\exp(-x_B/L_B)+n_{B0}}{2\sinh(x_B/L_B)}
    \end{Equation}
    故$B$为
    \begin{Equation}&[10]
        B=\frac{D_B}{D}=+\frac{n_{B0}\qty[\exp(eV_{BE}/\kB T)-1]\exp(+x_B/L_B)+n_{B0}}{2\sinh(x_B/L_B)}
    \end{Equation}
    将\xrefpeq{9}和\xrefpeq{10}代回\xrefpeq{2}
    \begin{Split}[12pt]
        \qquad\fdd{n_B(x)}=\frac{n_{B0}}{2\sinh(x_B/L_B)}\Bigg\{
        &\qty[\exp(\frac{eV_{BE}}{\kB T})-1]\qty[\exp(\frac{x_B-x}{L_B})-\exp(\frac{x-x_B}{L_B})]\qquad\\
        +&\qty[\exp(\frac{-x}{L_B})-\exp(\frac{+x}{L_B})]\Bigg\}
    \end{Split}
    运用双曲正弦简化
    \begin{Equation}
        \qquad
        \fdd{n_B(x)}=\frac{n_{B0}}{\sinh(x_B/L_B)}\qty\Bigg{\qty[\exp(\frac{eV_{BE}}{kT})-1]\sinh(\frac{x_B-x}{L_B})-\sinh(\frac{x}{L_B})}
        \qquad
    \end{Equation}
    运用双曲正弦$\sinh(x)\approx x$的近似
    \begin{Equation}
        \fdd{n_B(x)}=\frac{n_{B0}}{x_B/L_B}\qty\Bigg{\qty[\exp(\frac{eV_{BE}}{kT})-1]\qty(\frac{x_B-x}{L_B})-\qty(\frac{x}{L_B})}
    \end{Equation}
    即
    \begin{Equation}*
        \fdd{n_B(x)}=\frac{n_{B0}}{x_B}\qty\Bigg{\qty[\exp(\frac{eV_{BE}}{kT})-1]\qty(x_B-x)-x}\qedhere
    \end{Equation}
\end{Proof}

\subsection{放大区BJT的发射区少子分布}
\begin{BoxFormula}[放大区BJT的发射区少子分布]
    当BJT处于放大区时,其发射区过剩少子浓度$\fdd{p_E}(x')$满足
    \begin{Equation}
        \qquad\qquad\quad
        \fdd{p_E}(x')=\frac{p_{E0}}{\sinh(x_E/L_E)}\qty\Bigg{\qty[\exp(\frac{eV_{BE}}{\kB T})-1]\sinh(\frac{x_E-x'}{L_E})}
        \qquad\qquad\quad
    \end{Equation}
    当$x_E\ll L_E$时,通过双曲正弦的近似,得到
    \begin{Equation}
        \fdd{p_E}(x')=\frac{p_{E0}}{x_E}\qty\Bigg{\qty[\exp(\frac{eV_{BE}}{\kB T})-1](x_E-x')}
    \end{Equation}
\end{BoxFormula}

\begin{Proof}
    在发射区,扩散方程的形式为
    \begin{Equation}
        D_E\pdv[2]{\fdd{p_E(x')}}{{x'}}-\frac{p_E(x')}{\tau_{E0}}=0
    \end{Equation}
    通解是我们熟悉的,其中$L_E=\sqrt{D_E\tau_{E0}}$
    \begin{Equation}
        \fdd{p_E(x')}=A\exp(\frac{+x'}{L_B})+B\exp(\frac{-x'}{L_E})
    \end{Equation}
    内侧边界,发射结正偏,因此
    \begin{Equation}
        \fdd{p_E(0)}=p_{E0}\qty[\exp(\frac{eV_{BE}}{\kB T})-1]
    \end{Equation}
    外侧边界,在$x'=x_B$处过剩少子浓度降至零
    \begin{Equation}
        \fdd{p_E(x_B)}=0
    \end{Equation}
    这就是短PN结的格局,参照\fancyref{fml:短二极管的载流子分布}的过程,结果应为
    \begin{Equation}
        \qquad\qquad\qquad
        \fdd{p_E}(x')=\frac{p_{E0}}{\sinh(x_E/L_E)}\qty\Bigg{\qty[\exp(\frac{eV_{BE}}{\kB T})-1]\sinh(\frac{x_E-x'}{L_E})}
        \qquad\qquad\qquad
    \end{Equation}
    运用双曲正弦$\sinh(x)\approx x$的近似
    \begin{Equation}
        \fdd{p_E}(x')=\frac{p_{E0}}{x_E/L_E}\qty\Bigg{\qty[\exp(\frac{eV_{BE}}{\kB T})-1]\qty(\frac{x_E-x'}{L_E})}
    \end{Equation}
    即
    \begin{Equation}*
        \fdd{p_E}(x')=\frac{p_{E0}}{x_E}\qty\Bigg{\qty[\exp(\frac{eV_{BE}}{\kB T})-1](x_E-x')}\qedhere
    \end{Equation}
\end{Proof}

\subsection{放大区BJT的集电区少子分布}
\begin{BoxFormula}
    当BJT处于放大区时,其集电区过剩少子浓度$\fdd{p_C(x'')}$满足
    \begin{Equation}
        \fdd{p_C(x'')}=-p_{C0}\exp(\frac{-x''}{L_C})
    \end{Equation}
\end{BoxFormula}

\begin{Proof}
    在集电区,扩散方程的形式为
    \begin{Equation}
        D_C\pdv[2]{\fdd{p_C(x'')}}{{x''}}-\frac{p_C(x'')}{\tau_{C0}}=0
    \end{Equation}
    通解是我们熟悉的,其中$L_C=\sqrt{D_C\tau_{C0}}$
    \begin{Equation}
        \fdd{p_C(x'')}=A\exp(\frac{+x''}{L_C})+B\exp(\frac{-x''}{L_C})
    \end{Equation}
    内侧边界,集电结反偏,因此
    \begin{Equation}
        \fdd{p_C(0)}=-p_{C0}
    \end{Equation}
    外侧边界,在$x'=x_C$处过剩少子浓度降至零,但集电区域很长,可近似认为$x_C=\infty$
    \begin{Equation}
        \fdd{p_0(\infty)}=0
    \end{Equation}
    这就是一个标准的反偏PN结,很容易解出
    \begin{Equation}
        A=0\qquad B=-p_{C0}
    \end{Equation}
    故
    \begin{Equation}*
        \fdd{p_C(x'')}=-p_{C0}\exp(\frac{-x''}{L_C})\qedhere
    \end{Equation}
\end{Proof}

至此,\xref{fig:BJT在正向放大区的少子分布}中的各段曲线的解析式就求出来了(其中,实线为精确值,虚线为近似值)。

\subsection{BJT在各工作区的少子分布}
通过前面的讨论,我们看到BJT的少子分布没有什么特别的,无非就是在不同边值条件下求解同一个扩散方程,这非常无趣。因此,对于BJT的其他工作区,我们不再手工计算,而是直接通过Mathematica软件求解并令其绘图,通过图像直观理解。\xref{tab:BJT在各工作区的少子分布}展示了这些工作。

\begin{Tablex}[BJT在各工作区的少子分布]{|Y|Y|}
\xcell<Y>[1ex][-1ex]{\includegraphics[width=7cm]{build/Chapter04B_01e.fig.pdf}}&
\xcell<Y>[1ex][-1ex]{\includegraphics[width=7cm]{build/Chapter04B_01b.fig.pdf}}\\
\xcell<Y>[0ex][-1ex]{截止区\\ \footnotesize (发射结反偏\quad 集电结反偏)}&
\xcell<Y>[0ex][-1ex]{正向放大区\\ \footnotesize (发射结正偏\quad 集电结反偏)}\\ \hlinelig
\xcell<Y>[1ex][-1ex]{\includegraphics[width=7cm]{build/Chapter04B_01c.fig.pdf}}&
\xcell<Y>[1ex][-1ex]{\includegraphics[width=7cm]{build/Chapter04B_01d.fig.pdf}}\\
\xcell<Y>[0ex][-1ex]{反向放大区\\ \footnotesize (发射结反偏\quad 集电结正偏)}&
\xcell<Y>[0ex][-1ex]{饱和区\\ \footnotesize (发射结正偏\quad 集电结正偏)}\\ \hlinelig
\end{Tablex}
\section{BJT的电流关系}

\subsection{BJT的电流组成}
现在我们来讨论BJT的电流,\xref{fig:BJT的电流组成}展示了BJT处于放大区时的电流构成
\begin{itemize}
    \item $J_{nE}, J_{pE}$分别是$\te{E}\to\te{B}$和$\te{B}\to\te{E}$的电子电流和空穴电流。
    \item $J_{nC}, J_{pC}$分别是$\te{C}\to\te{B}$和$\te{B}\to\te{C}$的电子电流和空穴电流。
    \item $J_R$是正偏发射结的复合电流,参见\xref{subsec:正偏复合电流}。
    \item $J_G$是反偏发射结的产生电流,参见\xref{subsec:反偏产生电流}。
\end{itemize}
这里唯一比较费解的是$J_{pB}$,它到底是什么?\xref{fig:PN结的电流密度分布}或许给了我们答案。我们知道,在正偏PN结中,随着电子在P区的扩散,电子的扩散电流会逐渐转换为空穴的漂移电流,然而,这种转换是需要一个过程的,在BJT的基区,发射极注入基区的电子扩散流$J_{nE}$在只有很少的部分转换为空穴电流$J_{pB}$时就到达了基区边界,而剩余电流则仍然会保持电子电流$J_{nC}$的形式进入集电区。这也为先前我们称“发射区电子流,少部分从基区流出,大部分流向集电区”提供了更完善的理论支持,即$J_{nE}=J_{nC}+J_{pB}$中,对于进入基区的电子扩散电流$J_{nE}$
\begin{itemize}
    \item 那部分在通过基区中被转化为空穴电流的部分$J_{pB}$,将从基极流出。
    \item 那部分到达基区边界时还剩下的电子电流的部分$J_{nC}$,将进入集电区。
\end{itemize}

\begin{Figure}[BJT的电流组成]
    \includegraphics[width=\linewidth]{build/Chapter04B_02.fig.pdf}
\end{Figure}

这里将BJT的电流组成整理如下
\begin{BoxFormula}[BJT的发射极电流组成]
    BJT的发射极电流$J_E$的构成为
    \begin{Equation}
        J_E=J_{pE}+J_{nE}+J_R
    \end{Equation}
\end{BoxFormula}
\begin{BoxFormula}[BJT的集电极电流组成]
    BJT的集电极电流$J_C$的构成为
    \begin{Equation}
        J_C=J_{pC}+J_{nC}+J_G
    \end{Equation}
\end{BoxFormula}
\begin{BoxFormula}[BJT的基极电流组成]
    BJT的基极电流$J_B$的构成为
    \begin{Equation}
        J_B=J_{pB}+(J_{pE}+J_R)-(J_{pC}+J_G)
    \end{Equation}
\end{BoxFormula}

\subsection{BJT的电流增益}
在\xref{sec:BJT的基本结构}中,我们已经提到了增益$\beta$的概念,实际上BJT上有$\alpha,\beta$两个常用的增益指标。

\begin{BoxDefinition}[共基极电流增益]
    共基极电流增益,定义为集电极电流$I_C$与射极电流$I_E$的比(正向放大区)
    \begin{Equation}
        \alpha=\frac{I_C}{I_E}=\frac{J_C}{J_E}
    \end{Equation}
    通常共基极电流增益$\alpha$是一个略小于$1$的值。
\end{BoxDefinition}
\begin{BoxDefinition}[共射极电流增益]
    共射极电流增益,定义为集电极电流$I_C$与基极电流$I_B$的比(正向放大区)
    \begin{Equation}
        \beta=\frac{I_C}{I_B}=\frac{J_C}{J_B}
    \end{Equation}
    通常共射极电流增益$\beta$的典型值在$50$至$200$左右。
\end{BoxDefinition}

而考虑到$I_E=I_C+I_B$的关系(参见\xref{fig:BJT的核心原理}),实际上$\alpha$和$\beta$并不是独立的。

我们知道
\begin{Equation}
    I_E-I_C-I_B=0
\end{Equation}
根据\xref{def:共基极电流增益}和\xref{def:共射极电流增益},有
\begin{Equation}
    \qty(\frac{1}{\alpha}-\frac{1}{\beta}-1)I_C=0
\end{Equation}
即
\begin{Equation}
    \frac{1}{\alpha}-\frac{1}{\beta}=1
\end{Equation}
若想通过$\beta$计算$\alpha$
\begin{Equation}
    \alpha=\qty(\frac{1}{\beta}+1)^{-1}=\qty(\frac{1+\beta}{\beta})^{-1}=\frac{\beta}{1+\beta}
\end{Equation}
若想通过$\alpha$计算$\beta$
\begin{Equation}
    \beta=\qty(\frac{1}{\alpha}-1)^{-1}=\qty(\frac{1-\alpha}{\alpha})^{-1}=\frac{\alpha}{1-\alpha}
\end{Equation}
我们将结论整理一下
\begin{BoxFormula}[BJT电流增益间的关系]*
    BJT的电流增益$\alpha, \beta$间的关系是
    \begin{Equation}
        \frac{1}{\alpha}-\frac{1}{\beta}=1
    \end{Equation}
    计算$\alpha$
    \begin{Equation}
        \alpha=\frac{\beta}{1+\beta}
    \end{Equation}
    计算$\beta$
    \begin{Equation}
        \beta=\frac{\alpha}{1-\alpha}
    \end{Equation}
\end{BoxFormula}
作为一个参考,若$\alpha=0.99$,则$\beta=99$,因此$\alpha$需要非常接近$1$。现在的问题是,增益$\alpha$到底与那些因素有关?这就需要让我们考察电流了,依照\fancyref{def:共基极电流增益}
\begin{Equation}
    \alpha=\frac{J_C}{J_E}
\end{Equation}
根据\fancyref{fml:BJT的发射极电流组成}和\fancyref{fml:BJT的集电极电流组成}
\begin{Equation}
    \alpha=\frac{J_{pC}+J_{nC}+J_G}{J_{pE}+J_{nE}+J_R}
\end{Equation}
这里$J_{pC}$和$J_G$是通常的反偏电流,远远小于$J_{pC}$,可以忽略
\begin{Equation}
    \alpha=\frac{J_{nC}}{J_{pE}+J_{nE}+J_R}
\end{Equation}
我们愿意将$\alpha$剥离为关系清晰的三个部分
\begin{Equation}
    \alpha=\qty(\frac{J_{nC}}{J_{nE}})\frac{J_{nE}}{J_{pE}+J_{nE}+J_R}
\end{Equation}
% 第一个被剥离的部分$J_{nC}/J_{nE}$被称为基区输运系数,其衡量了电子流经过基区时的损失。

而继续剥离可以得到
\begin{Equation}
    \alpha=\qty(\frac{J_{nC}}{J_{nE}})\qty(\frac{J_{nE}}{J_{nE}+J_{pE}})\qty(\frac{J_{nE}+J_{pE}}{J_{nE}+J_{pE}+J_{R}})
\end{Equation}

% 第二个部分$J_{nE}/(J_{nE}+J_{pE})$被称为注入系数,它是发射结电子电流与电子电流和空穴电流和的比,换言之,它代表了对BJT工作有帮助的,注入基区的电子电流在理想电流中的比例。

% 第三个部分$(J_{nE}+J_{pE})/(J_{nE}+J_{pE}+J_{R})$被称为复合系数,它是理想电流占总电流的比。

这三部分依次被称为:输运系数$\alpha_T$、注入系数$\gamma$、复合系数$\delta$。

\begin{BoxDefinition}[输运系数]
    \uwave{输运系数}(Transport Factor)衡量了电子流通过基区时的损耗
    \begin{Equation}
        \alpha_T=\frac{J_{nC}}{J_{nE}}
    \end{Equation}
\end{BoxDefinition}

\begin{BoxDefinition}[注入系数]
    \uwave{注入系数}(Injection Factor)衡量了发射结的电子流和空穴流中,注入电子流的占比
    \begin{Equation}
        \gamma=\frac{J_{nE}}{J_{nE}+J_{pE}}
    \end{Equation}
\end{BoxDefinition}

\begin{BoxDefinition}[复合系数]
    \uwave{复合系数}(Recombination Factor)衡量了发射结正偏复合电流的影响
    \begin{Equation}
        \delta=\frac{J_{nE}+J_{pE}}{J_{nE}+J_{pE}+J_{R}}
    \end{Equation}
\end{BoxDefinition}

因此,共基极电流增益$\alpha$就可以被分解为以上三个部分。
\begin{BoxFormula}[共基极电流增益的分解]
    共基极电流增益$\alpha$可以分解为
    \begin{Equation}
        \alpha=\alpha_T\cdot \gamma\cdot \delta
    \end{Equation}
\end{BoxFormula}
我们期望的一个好的BJT应当具有较高的共射极电流增益$\beta$,这就要求共基极电流增益$\alpha$尽可能的接近1,而$\alpha=\alpha_T\cdot \gamma\cdot \delta$,这就要求$\alpha_T, \gamma, \delta$都很接近$1$。因此现在的任务就是,依次计算出输运系数$\alpha_T$、注入系数$\gamma$、复合系数$\delta$的表达式,并考察哪些参数会对三者有影响。

\subsection{输运系数}
\begin{BoxFormula}[输运系数]
    输运系数$\alpha_T$可以表示为
    \begin{Equation}
        \alpha_T=\frac{1}{\cosh(x_B/L_B)}
    \end{Equation}
    或近似为
    \begin{Equation}
        \alpha_T=1-\frac{1}{2}(x_B/L_B)^2
    \end{Equation}
\end{BoxFormula}

\begin{Proof}
    根据\fancyref{def:输运系数}
    \begin{Equation}&[1]
        \alpha_T=\frac{J_{nC}}{J_{nE}}
    \end{Equation}
    参照\xref{fig:BJT的电流组成},此处$J_{nE}$和$J_{nC}$分别是$\fdd{n_B(x)}$在$x=0$和$x=x_B$处的取值。

    $J_{nE}$应当表示为
    \begin{Equation}&[2]
        J_{nE}=-eD_B\eval{\dv{\fdd{n_B(x)}}{x}}_{x=0}
    \end{Equation}
    $J_{nC}$应当表示为
    \begin{Equation}&[3]
        \hspace{0.5em}J_{nC}=-eD_B\eval{\dv{\fdd{n_B(x)}}{x}}_{x=x_B}
    \end{Equation}
    而根据\fancyref{fml:放大区BJT的基区少子分布}
    \begin{Equation}&[3.5]
        \qquad
        \fdd{n_B}(x)=\frac{n_{B0}}{\sinh(x_B/L_B)}\qty\Bigg{\qty[\exp(\frac{eV_{BE}}{\kB T})-1]\sinh(\frac{x_B-x}{L_B})-\sinh(\frac{x}{L_B})}
        \qquad
    \end{Equation}
    求导结果为
    \begin{Equation}&[4]
        \qquad
        \dv{\fdd{n_B}(x)}{x}=\frac{-n_{B0}}{L_B\sinh(x_B/L_B)}\qty\Bigg{\qty[\exp(\frac{eV_{BE}}{\kB T})-1]\cosh(\frac{x_B-x}{L_B})+\cosh(\frac{x}{L_B})}
        \qquad
    \end{Equation}
    对于$J_{nE}$,应乘以$-eD_B$并取$x=0$
    \begin{Equation}&[5]
        J_{nE}=\frac{eD_Bn_{B0}}{L_B}\qty\Bigg{\frac{\qty[\exp(eV_{BE}/\kB T)-1]}{\tanh(x_B/L_B)}+\frac{1}{\sinh(x_B/L_B)}}
    \end{Equation}
    对于$J_{nC}$,应乘以$-eD_B$并取$x=x_B$
    \begin{Equation}&[6]
        J_{nC}=\frac{eD_Bn_{B0}}{L_B}\qty\Bigg{\frac{\qty[\exp(eV_{BE}/\kB T)-1]}{\sinh(x_B/L_B)}+\frac{1}{\tanh(x_B/L_B)}}
    \end{Equation}
    将\xrefpeq{5}通分为以$\sinh(x_B/L_B)$为分母
    \begin{Equation}&[7]
        \qquad\qquad\quad
        J_{nE}=\frac{eD_{B}n_{B0}}{L_{B}\sinh(x_B/L_B)}\qty\Big{\qty[\exp(eV_{BE}/\kB T)-1]\cosh(x_B/L_B)+1}
        \qquad\qquad\quad
    \end{Equation}
    将\xrefpeq{5}通分为以$\sinh(x_B/L_B)$为分母
    \begin{Equation}&[8]
        \qquad\qquad\quad
        J_{nC}=\frac{eD_{B}n_{B0}}{L_{B}\sinh(x_B/L_B)}\qty\Big{\qty[\exp(eV_{BE}/\kB T)-1]+\cosh(x_B/L_B)}
        \qquad\qquad\quad
    \end{Equation}
    将\xrefpeq{7}和\xrefpeq{8}代入\xrefpeq{1}
    \begin{Equation}&[9]
        \alpha_T=\frac{J_{nC}}{J_{nE}}=\frac{[\exp(eV_{BE}/\kB T)-1]+\cosh(x_B/L_B)}{[\exp(eV_{BE}/\kB T)-1]\cosh(x_B/L_B)+1}
    \end{Equation}
    应用发射结正偏电压足够大$V_{BE}\gg \kB T/e$的近似,这使得$\exp(eV_{BE}/\kB T)\gg 1$
    \begin{Equation}&[10]
        \alpha_T=\frac{\exp(eV_{BE}/\kB T)+\cosh(x_B/L_B)}{\exp(eV_{BE}/\kB T)\cosh(x_B/L_B)+1}
    \end{Equation}

    应用基区很窄$x_B\ll L_B$,这使得$\cosh(x_B/L_B)$只是略大于$1$,而同时,按照上面的讨论,我们又有$\exp(eV_{BE}/\kB T)\gg 1$,故\xrefpeq{10}中,分母的$1$和分子的$\cosh(x_B/L_B)$可以忽略
    \begin{Equation}
        \alpha_T=\frac{\exp(eV_{BE}/\kB T)}{\exp(eV_{BE}/\kB T)\cosh(x_B/L_B)}
    \end{Equation}
    即
    \begin{Equation}
        \alpha_T=\frac{1}{\cosh(x_B/L_B)}
    \end{Equation}
    依据$x_B\ll L_B$的条件,作泰勒展开近似$\cosh(\xi)=1+(\xi)^2/2$
    \begin{Equation}
        \alpha_T=\frac{1}{1+(x_B/L_B)^2/2}
    \end{Equation}
    依据$x_B\ll L_B$的条件,作泰勒展开近似$1/(1+\xi)=1-\xi$
    \begin{Equation}
        \alpha_T=1-\frac{1}{2}(x_B/L_B)^2
    \end{Equation}
    至此,我们就完成了$\alpha_T$的计算和近似。
\end{Proof}

输运系数$\alpha_T$解释了为何BJT的基区需要很窄。输运系数$\alpha_T$代表了发射结注入的电子流能有多少能最终到达集电结,换言之$1-\alpha_T$就代表了电子流在基区的损失状况,而\xref{fml:输运系数}告诉我们$1-\alpha_T\propto x_B^2$的关系,换言之,基区的宽度$x_B$和电子流在基区的损失是呈平方关系的!因此,降低基区宽度$x_B$即令基区很窄可以显著改善增益,使$\alpha_T$并最终使$\alpha$更接近$1$。

\subsection{注入系数}
\begin{BoxFormula}[注入系数]
    注入系数$\gamma$可以表示为
    \begin{Equation}
        \gamma=\qty[1+\frac{D_E}{D_B}\frac{N_B}{N_E}\frac{L_B\tanh(x_B/L_B)}{L_E\tanh(x_E/L_E)}]^{-1}
    \end{Equation}
    或近似为
    \begin{Equation}
        \gamma=\qty[1+\frac{D_E}{D_B}\frac{N_B}{N_E}\frac{x_B}{x_E}]^{-1}
    \end{Equation}
\end{BoxFormula}
\begin{Proof}
    根据\fancyref{def:注入系数}
    \begin{Equation}&[1]
        \gamma=\frac{J_{nE}}{J_{nE}+J_{pE}}
    \end{Equation}
    不妨上下同除$J_{nE}$化简为
    \begin{Equation}&[2]
        \gamma=\frac{1}{1+J_{pE}/J_{nE}}=\qty[1+\frac{J_{pE}}{J_{nE}}]^{-1}
    \end{Equation}
    参照\xref{fig:BJT的电流组成},此处$J_{pE}$是$\fdd{p_E(x')}$在$x'=0$处的取值,故
    \begin{Equation}&[3]
        J_{pE}=-eD_{E}\eval{\dv{\fdd{p_E(x')}}{x'}}_{x'=0}
    \end{Equation}
    而根据\fancyref{fml:放大区BJT的发射区少子分布}
    \begin{Equation}&[4]
        \qquad\qquad\qquad
        \fdd{p_E}(x')=\frac{p_{E0}}{\sinh(x_E/L_E)}\qty\Bigg{\qty[\exp(\frac{eV_{BE}}{\kB T})-1]\sinh(\frac{x_E-x'}{L_E})}
        \qquad\qquad\qquad
    \end{Equation}
    求导结果为
    \begin{Equation}&[5]
        \qquad\qquad\quad
        \dv{\fdd{p_E(x')}}{x'}=\frac{-p_{E0}}{L_E\sinh(x_E/L_E)}\qty\Bigg{\qty[\exp(\frac{eV_{BE}}{\kB T})-1]\cosh\qty(\frac{x_E-x'}{L_E})}
        \qquad\qquad\quad
    \end{Equation}
    这样$J_{pE}$就可以表示为(乘以$-eD_E$并取$x'=0$)
    \begin{Equation}&[6]
        J_{pE}=\frac{eD_Ep_{E0}}{L_E}\qty\Bigg{\frac{[\exp(eV_{BE}/\kB T)-1]}{\tanh(x_E/L_E)}}
    \end{Equation}
    
    这里$J_{nE}$我们已经在\fancyref{fml:输运系数}中的\xrefpeq[输运系数]{5}推导过了
    \begin{Equation}&[7]
        J_{nE}=\frac{eD_Bn_{B0}}{L_B}\qty\Bigg{\frac{\qty[\exp(eV_{BE}/\kB T)-1]}{\tanh(x_B/L_B)}+\frac{1}{\sinh(x_B/L_B)}}
    \end{Equation}
    关于\xrefpeq{7},这里我们可以忽略$1/\sinh(x_B/L_B)$项
    \begin{Equation}&[8]
        J_{nE}=\frac{eD_Bn_{B0}}{L_B}\qty\Bigg{\frac{\qty[\exp(eV_{BE}/\kB T)-1]}{\tanh(x_B/L_B)}}
    \end{Equation}
    简化后,计算$J_{pE}$和$J_{nE}$的比就容易很多了,将\xrefpeq{6}和\xrefpeq{8}代入\xrefpeq{2}
    \begin{Equation}&[9]
        \gamma=\qty[1+\frac{J_{pE}}{J_{nE}}]^{-1}=\qty[1+\frac{D_Ep_{E0}}{D_Bn_{B0}}\frac{L_B\tanh(x_B/L_B)}{L_E\tanh(x_E/L_E)}]^{-1}
    \end{Equation}
    由于$p_{E0}$和$n_{B0}$分别为发射区和基区的少子,应用
    \begin{Equation}&[10]
        p_{E0}=\frac{n_i^2}{N_E}\qquad
        n_{B0}=\frac{n_i^2}{N_B}
    \end{Equation}
    这里$N_E$和$N_B$分别表示发射区和基区的掺杂浓度,将\xrefpeq{10}代入\xrefpeq{9}
    \begin{Equation}
        \gamma=\qty[1+\frac{D_E}{D_B}\frac{N_B}{N_E}\frac{L_B\tanh(x_B/L_B)}{L_E\tanh(x_E/L_E)}]^{-1}
    \end{Equation}
    由于$x_B\ll L_B$,可以将$\tanh(xB/L_B)$近似为$x_B/L_B$
    \begin{Equation}
        \gamma=\qty[1+\frac{D_E}{D_B}\frac{N_B}{N_E}\frac{x_B}{x_E}]^{-1}
    \end{Equation}
    至此,我们就完成了$\gamma$的计算和近似。
\end{Proof}

注入系数$\gamma$解释了为何BJT的发射区需重掺而基区需很窄,在\xref{fml:注入系数}中,我们看到,发射区和基区的掺杂比$N_E/N_B$和宽度比$x_E/x_B$越高(前者由发射区重掺提高,后者由基区很窄提高),注入系数$\gamma$就越接近$1$,换言之,发射结理想电流中对BJT放大有帮助的电子电流就占越高的比例(即那部分对BJT放大无帮助,由基区流向发射区的空穴电流,就会越少)。

\subsection{复合系数}
\begin{BoxFormula}[复合系数]
    复合系数$\delta$可以表示为
    \begin{Equation}
        \delta=\qty[1+\frac{J_{r0}}{J_{s0}}\exp(-\frac{eV_{BE}}{2\kB T})]^{-1}
    \end{Equation}
    其中$J_{r0}, J_{s0}$分别代表复合电流和理想电流的系数
    \begin{Equation}
        J_{r0}=\frac{eW_{BE}n_i}{2\tau_0}\qquad
        J_{s0}=\frac{eD_Bn_{B0}}{L_B\tanh(x_B/L_B)}
    \end{Equation}
    其中$W_{BE}$是发射结空间电荷区的宽度。
\end{BoxFormula}

\begin{Proof}
    根据\fancyref{def:复合系数}
    \begin{Equation}&[1]
        \delta=\frac{J_{nE}+J_{pE}}{J_{nE}+J_{pE}+J_R}
    \end{Equation}
    复合系数主要关注的是复合电流$J_R$的影响,因此可以忽略$J_{pE}$
    \begin{Equation}&[2]
        \delta=\frac{J_{nE}}{J_{nE}+J_R}
    \end{Equation}
    不妨上下同除$J_{nE}$化简为
    \begin{Equation}&[3]
        \delta=\frac{1}{1+J_R/J_{nE}}=\qty[1+\frac{J_R}{J_{nE}}]^{-1}
    \end{Equation}
    $J_{nE}$我们援引\xrefpeq[注入系数]{8}的近似式,并再近似掉指数后的$-1$
    \begin{Equation}&[4]
        J_{nE}=\frac{eD_Bn_{B0}}{L_B\tanh(x_B/L_B)}\exp(\frac{eV_{BE}}{\kB T})=J_{s0}\exp(\frac{eV_{BE}}{\kB T})
    \end{Equation}
    $J_R$是PN结非理想效应的复合电流,我们引用\fancyref{fml:正偏复合电流}
    \begin{Equation}&[5]
        J_{R}=\frac{eW_{BE}n_i}{2\tau_0}\exp(\frac{eV_{BE}}{2\kB T})=J_{r0}\exp(\frac{eV_{BE}}{2\kB T})
    \end{Equation}
    将\xrefpeq{4}和\xrefpeq{5}代入\xrefpeq{3}
    \begin{Equation}
        \delta=\qty[1+\frac{J_{r0}}{J_{s0}}\exp(-\frac{eV_{BE}}{2\kB T})]^{-1}
    \end{Equation}
    至此,我们就完成了$\delta$的计算。
\end{Proof}
复合系数$\delta$表征了复合电流的影响,复合电流和空穴电流一样,都会冲淡电子电流在发射结电流中的占比。我们注意到,当发射结正偏电压$V_{BE}$较大时,复合系数$\delta$就会较接近$1$,因此,只要适当增大$V_{BE}$就可以基本避免复合电流的影响。这很合理,因为在\xref{fig:实际PN结的电流--电压关系}中就已经提到过,PN结正偏时,正偏电压较小时由复合电流主导,正偏电压较大时由扩散电流主导。
\section{BJT的电路模型}
在本节,我们将分别讨论BJT的两个模型,它们分别适用于大信号和小信号。

\subsection{BJT的大信号模型}

Ebers-Moll模型,或简称EM模型,是适用于BJT大信号的电路模型,也是BJT的经典模型之一。Ebers-Moll模型是以PN结间的相互作用为基础,可以适用于BJT的任何工作区。

Ebers-Moll模型中规定的电流方向和作为自变量的电压的选取,和通常所使用的共射极接法是有所不同的。原因主要是EM模型需要对等的考虑包括正向放大区和反向放大区的各个工作区,因此,电流和电压的朝向的选取应以BJT的结构为中心,而不能顺着正向放大区的习惯来。如\xref{fig:BJT在共射极接法中的方向约定}所示,通常的规定是,$I_B,I_C$流入BJT,$I_E$流出BJT,电压则以“共射”为基础选择为$V_{BE}$和$V_{CE}$。而在Ebers-Moll模型中,如\xref{fig:BJT在EM模型中的方向约定}所示,电流$I_B',I_C',I_E'$均被指定为流入BJT,电压则选取了直接与BJT工作区划分有关,即决定发射结和集电结偏置状态的$V_{BE}$和$V_{BC}$。某种意义上,EM模型的电流电压选取,是更接近BJT的器件物理本质的。
\begin{Figure}[BJT的电流和电压方向约定]
    \begin{FigureSub}[BJT在共射极接法中的方向约定]
        \includegraphics{build/Chapter04D_05.fig.pdf}
    \end{FigureSub}
    \hspace{1cm}
    \begin{FigureSub}[BJT在EM模型中的方向约定]
        \includegraphics{build/Chapter04D_06.fig.pdf}
    \end{FigureSub}
\end{Figure}
将Ebers-Moll模型中的这种方向指定,整理如下
\begin{BoxFormula}[Ebers-Moll模型中的变量代换]*
    Ebers-Moll模型中,电流选取为
    \begin{Equation}
        I_B'=I_B\qquad I_C'=I_C\qquad I_E'=-I_E
    \end{Equation}
    Ebers-Moll模型中,电压选取为
    \begin{Equation}
        V_{BE}=V_{BE}\qquad V_{BC}=V_{BE}-V_{CE}
    \end{Equation}
\end{BoxFormula}
现在的问题是,作为一个能覆盖BJT各个工作区的模型,Ebers-Moll模型到底是如何被构建的?关键在于平等看待BJT的结构!前面我们已经系统学过了正向放大区,而相同的想法也适用反向放大区。因为\xref{subsec:BJT的四个工作区}就提到过,BJT的结构组成是对称的,只是参数不对称!
\begin{itemize}
    \item 在正向放大区,若$\te{B}\to\te{E}$结正偏电流为$I_F$,则$\te{C}\to\te{B}$的电流为$\alpha_FI_F$。
    \item 在反向放大区,若$\te{B}\to\te{C}$结正偏电流为$I_R$,则$\te{E}\to\te{B}$的电流为$\alpha_RI_R$。
\end{itemize}

\begin{Figure}[Ebers-Moll模型]
    \includegraphics[scale=0.9]{build/Chapter04D_04.fig.pdf}
\end{Figure}
而事实是,在BJT中,这两套放大机制是同时叠加存在的!\xref{fig:Ebers-Moll模型}展示的就是Ebers-Moll模型的电路,它包含两个二极管和两个受控电流源(这里的两个二极管相互独立,原先BJT中的发射结和集电结间的那种耦合关系已经被两个受控源表示了),分别代表了两套放大机制。

% 而在这种想法下,各工作区就很容易统一在一起了
\begin{itemize}
    \item 当处于正向放大区时,由于集电结反偏$I_R\approx 0$,事实只有正向放大机制在运作。
    \item 当处于反向放大区时,由于发射结反偏$I_F\approx 0$,事实只有反向放大机制在运作。
    \item 当处于饱和区时,即两套放大机制同时工作。
    \item 当处于截止区时,即两套放大机制均不工作。
\end{itemize}
这其实就是Ebers-Moll模型能普遍适用BJT任何工作区的秘密:对等考虑正向放大特性和反向放大特性,认为两者具有相同的放大机制并同时存在,将各工作区视为这两种机制的叠加。\setpeq{EM模型}

接下来进行数学上的分析,显然$I_E',I_B',I_C'$间满足
\begin{Equation}&[1]
    I_E'+I_B'+I_C'=0
\end{Equation}
这是基尔霍夫电流定律的结果(注意$I_E',I_B',I_C'$的方向都被定义为流入BJT的方向)。\goodbreak

集电极电流通常可以写为
\begin{Equation}&[2]
    I_C'=\alpha_FI_F-I_R
\end{Equation}
发射极电流通常可以写为
\begin{Equation}&[3]
    I_E'=\alpha_RI_R-I_F
\end{Equation}
这里$\alpha_F$和$\alpha_R$分别是正向放大和反向放大时的共基极增益,实际上BJT的参数不对称最终就反映在$\alpha_F$和$\alpha_R$的不同上,通常比较两者依$\beta=\alpha/(1-\alpha)$对应的$\beta_F$和$\beta_R$的值,通常来说,$\beta_F$在$20$至$200$间,$\beta_R$在$0$至$20$间。这也是为何我们认为反向放大区的性能很差。

$I_F$和$I_R$是二极管上的电流,根据\fancyref{fml:PN结的理想电流--电压关系}
\begin{Equation}&[4]
    I_F=I_{FS}\qty[\exp(\frac{eV_{BE}}{\kB T})-1]\qquad
    I_R=I_{RS}\qty[\exp(\frac{eV_{BC}}{\kB T})-1]
\end{Equation}
$I_{FS}, I_{RS}, \alpha_F, \alpha_R$是Ebers-Moll模型的四个参数,但只有三个是独立的
\begin{Equation}&[5]
    \alpha_FI_{FS}=\alpha_RI_{RS}
\end{Equation}
由于$\alpha_F$和$\alpha_R$都是很接近$1$的参数,故$I_{FS}$和$I_{RS}$几乎是相等的。

至将\xrefpeq{4}代入\xrefpeq{2}和\xrefpeq{3},并结合\xrefpeq{1}和\xrefpeq{4},就得到了EM模型的全貌了。

\begin{BoxFormula}[Ebers-Moll模型]
    在Ebers-Moll模型中,集电极电流$I_C'$为
    \begin{Equation}
        I_C'=\alpha_FI_{FS}\qty[\exp(\frac{eV_{BE}}{\kB T})-1]-I_{RS}\qty[\exp(\frac{eV_{BC}}{\kB T})-1]
    \end{Equation}
    在Ebers-Moll模型中,发射极电流$I_E'$为
    \begin{Equation}
        I_E'=\alpha_RI_{RS}\qty[\exp(\frac{eV_{BE}}{\kB T})-1]-I_{FS}\qty[\exp(\frac{eV_{BC}}{\kB T})-1]
    \end{Equation}
    $I_E',I_B',I_C'$间满足
    \begin{Equation}
        I_E'+I_B'+I_C'=0
    \end{Equation}
    $I_{FS}, I_{RS}, \alpha_F, \alpha_R$间存在约束关系
    \begin{Equation}
        \alpha_FI_{FS}=\alpha_RI_{RS}
    \end{Equation}
\end{BoxFormula}

在\xref{fig:Ebers-Moll模型}中,展示了Ebers-Moll模型下BJT的特性方程$I_C'=f(V_{BE},V_{BC})$。绘图时,我们取参数$\beta_F=100$和$\beta_R=10$,为避免恼人的数量级问题,所有图像中的$I_C'$均以$I_C'/I_{FS}$的形式绘制。\xref{fig:Ebers-Moll模型的特性曲面}展示了特性曲面,联合\xref{fig:BJT的四个工作区},可以清楚看到BJT在各工作区的特性。

\begin{Figure}[Ebers-Moll模型]
    \begin{FigureSub}[Ebers-Moll模型的特性曲面]
        \includegraphics{build/Chapter04D_01b.fig.pdf}
    \end{FigureSub}\\ \vspace{1.5cm}
    \begin{FigureSub}[关于集电结电压的曲线]
        \includegraphics[scale=0.8]{build/Chapter04D_01c.fig.pdf}
    \end{FigureSub}
    \hspace{0.25cm}
    \begin{FigureSub}[关于发射结电压的曲线]
        \includegraphics[scale=0.8]{build/Chapter04D_01e.fig.pdf}
    \end{FigureSub}\\ \vspace{1cm}
    \begin{FigureSub}[Ebers-Moll模型的共射增益]
        \includegraphics{build/Chapter04D_01g.fig.pdf}
    \end{FigureSub}
    % \begin{FigureSub}[共射增益的异常解释]
    %     \includegraphics{build/Chapter04D_01h.fig.pdf}
    % \end{FigureSub}
\end{Figure}

\xref{fig:关于集电结电压的曲线}绘制了$I_C'$关于$V_{BC}$的图像,注意到$I_C'$会随集电结反偏电压$-V_{BC}$的增大而趋于饱和。这是因为,集电结正偏时,集电结的集电能力是饱和的,能抽取多少电子取决于集电结的反偏电压有多少,故这一阶段被称为“饱和区”。此时$I_C'$会随$-V_{BC}$的快速增长,即该阶段中,集电结电流反而是非饱和的。而随着$-V_{BC}$的增大,集电结进入反偏,电场足够强,以至于发射结注入多少电子集电结就能抽取多少电子,因而,集电结的电流饱和了。此时$I_C'$不再随$-V_{BC}$变化,变成了仅关于$V_{BE}$的函数,形成了稳定的放大关系,故被称为“放大区”。

\xref{fig:关于发射结电压的曲线}绘制了$I_C'$关于$V_{BE}$的图像,它告诉我们BJT的$I_C'=f(V_{BE})$的放大关系是指数的,这一点和MOSFET很不同,MOSFET是平方放大,BJT是指数放大。当然,这都是大信号上的讨论,由于放大都是对小信号而言,在静态工作点附近,两者都可视为线性放大。

除此之外,\xref{fig:Ebers-Moll模型的共射增益}还全面展示了共射增益$\beta=I_C'/I_B'$,可以看到,截止区中$\beta=0$(截止区边界的异常突变是由于$I_B'$由负转正导致的,并不重要,可以视为是Ebers-Moll模型构建时的缺陷),正向放大区$\beta=\beta_F$,反向放大区$\beta=\beta_R$,饱和区中$\beta$则包含了$\beta_F$至$\beta_R$的过渡。

\begin{Figure}[BJT的特性方程]
    \begin{FigureSub}[BJT的特性曲面]
        \includegraphics{build/Chapter04D_01a.fig.pdf}
    \end{FigureSub}\\ \vspace{0.5cm}
    \begin{FigureSub}[BJT的转移特性]
        \includegraphics[scale=0.8]{build/Chapter04D_01d.fig.pdf}
    \end{FigureSub}
    \hspace{0.25cm}
    \begin{FigureSub}[BJT的输出特性]
        \includegraphics[scale=0.8]{build/Chapter04D_01f.fig.pdf}
    \end{FigureSub}
\end{Figure}

当然,我们也可以将\xref{fml:Ebers-Moll模型中的变量代换}代入\xref{fml:Ebers-Moll模型},从而得到在通常的BJT的电流电压习惯下的特性方程$I_C=f(V_{BE},V_{CE})$和相应的特性曲面,如\xref{fig:BJT的特性方程}所示。这样绘制的图像其实离BJT的物理机制更远了,例如饱和区和放大区的界线就变得更难以理解了。但补偿好处是,在这种坐标系下BJT的特性方程$I_C=f(V_{BE},V_{CE})$与MOSFET的$I_D=f(V_{GS},V_{DS})$就能对应了。

\subsection{BJT的小信号模型}
Hybrid Pi模型,是适用于BJT小信号的电路模型,其电路图如\xref{fig:Hybrid Pi模型}所示。
\begin{Figure}[Hybrid Pi模型]
    \begin{FigureSub}[完整模型]
        \includegraphics[scale=0.9]{build/Chapter04D_02.fig.pdf}   
    \end{FigureSub}\\ \vspace{0.75cm}
    \begin{FigureSub}[简化模型]
        \includegraphics[scale=0.9]{build/Chapter04D_03.fig.pdf}   
    \end{FigureSub}
\end{Figure}

Hybrid Pi模型看起来很复杂,但若划分为几个部分,则仍然是很好理解的
\begin{itemize}
    \item 电阻$r_e,r_c,r_b$分别是发射区、集电区、基各自的体电阻。
    \item 电容$C_s$是集电区(如\xref{fig:BJT的实际结构}所示,集电区在最外侧)和衬底间的电容。
    \item EC间的受控电流源$g_mV_{b'e'}$是小信号放大的核心,而$r_0$是厄利效应导致的输出电阻。
    \item BE间的$r_{\pi},C_{\pi},C_{je}$即正偏BE结的小信号模型(扩散电阻、扩散电容、势垒电容)。
    \item BC间的$r_{\mu},C_{\mu}$即反偏BE结的小信号模型(扩散电阻、势垒电容),参见\xref{fig:二极管的完整小信号等效电路}。
\end{itemize}
而如\xref{fig:简化模型}所示的简化模型中,可以只保留受控电流源,以及发射结的扩散电容和电阻。