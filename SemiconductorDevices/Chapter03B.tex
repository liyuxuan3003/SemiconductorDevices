\section{金属半导体的欧姆接触}

\subsection{整流接触与欧姆接触}

在\xref{sec:金属半导体的整流接触}中,我们已经学习了金半接触的整流接触的相关理论。但是,很多情况下我们需要另外一种接触方式,例如,任何半导体器件或半导体集成电路,最终都需要连接到金属上通过封装引脚连接到电路板上,但是在该过程中,我们希望金属和半导体只是简单的“连接”在一起,并不希望产生任何额外的整流特性。而这“另外一种”接触方式,就是所谓的欧姆接触。

\begin{Figure}[整流接触与欧姆接触]
    \begin{FigureSub}[整流接触($\phi_m>\phi_s$);整流接触]
        \includegraphics[scale=0.9]{build/Chapter03A_03.fig.pdf}
    \end{FigureSub}
    \hspace{0.5cm}
    \begin{FigureSub}[欧姆接触($\phi_m<\phi_s$);欧姆接触]
        \includegraphics[scale=0.9]{build/Chapter03B_01.fig.pdf}
    \end{FigureSub}
\end{Figure}

在\xref{subsec:金半整流接触的定性分析}中,我们已经了解决定整流接触与欧姆接触的要素了
\begin{itemize}
    \item 如\xref{fig:整流接触}所示,若$\phi_m>\phi_s$,则接触后能带上弯,形成整流接触。
    \item 如\xref{fig:欧姆接触}所示,若$\phi_m<\phi_s$,则接触后能带下弯,形成欧姆接触。
\end{itemize}

欧姆接触具有许多与整流接触弯曲不同的性质。首先,对于半导体侧而言,由于能带下弯,半导体的电子进入金属将不存在任何势垒。形象的说,在欧姆接触时,半导体中的电子可以不受任何阻碍的“俯冲”至金属中。其次,对于金属侧而言,尽管仍然存在$e\phi_{Bn}$的势垒,但不同的是,整流接触中$\phi_{B0}=\phi_n+V_{bi}$,欧姆接触中由于能带下弯$\phi_{B0}=\phi_n$。即,欧姆接触中的金属侧势垒只包含费米能级与导带的差,这样一来$\phi_{B0}$其实非常小,金属中的电子进入半导体也不会受到太大的阻碍。由此可见,欧姆接触“连接”金属和半导体而不引入额外特性。

\subsection{整流接触的隧道效应}
然而,欧姆接触在实践中并没有理论上这么容易实现。原因有二,首先,欧姆接触的关键条件在于$\phi_m<\phi_s$,这对于使用的金属和半导体材料的功函数有要求,然而很多时候,我们并不能随心所欲的选取材料,所需要连接的金属和半导体从功函数上恰好就无法形成欧姆接触。其次,受到表面态的影响,费米能级将被表面态钉扎,无论选取何种材料,费米能级都无法像预期的那样满足欧姆接触的形成条件。综上,介于欧姆接触常常难以实现,需要一种替代方案。

实际上,我们可以仍然使用整流接触,但通过一些方式使它具有类似欧姆接触的特性。我们知道,整流的含义就是“正向导通、反向截止”,所以现在我们的目的就是在整流接触的条件下使结在反偏下仍然能导通。金半整流接触反向截止的原因是,肖特基势垒$e\phi_{B0}$阻碍了金属中电子向半导体的运动,但是,如果我们对半导体侧进行重掺,依据\xref{fml:金半整流接触的空间电荷区宽度},势垒宽度反比于掺杂浓度的平方根,换言之,重掺时,势垒将变得非常窄,此时,隧道效应将变得很显著。金属侧电子不再需要越过整个肖特基势垒$e\phi_{B0}$,只要获得$e\phi_n$的能量使电子的能量高于肖特基势垒另一侧的导带底,金属侧电子就有机会通过隧穿直接进入半导体。重掺不存在任何适用性困难。所以,实践中的欧姆接触,往往都是通过重掺在整流接触中形成隧道效应实现的。