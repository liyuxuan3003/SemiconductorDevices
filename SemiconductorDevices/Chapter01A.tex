\section{PN结电流}
我们已经熟知,当PN结上加正偏电压时,在PN结内就会产生电流。在本节,我们先定性讨论PN结内的电荷是如何流动的,随后,再考虑PN结电压--电流定量关系的数学推导。

\subsection{PN结内电荷流动的定性描述}
通过\xref{fig:PN结能带图}所示的能带图,我们可以定性地理解PN结电流的机理。\xref{fig:零偏PN结}展示了热平衡状态下的PN结能带图。我们指出,电子遇到的势垒阻碍了N区大量电子流入P区,而完全类似的,空穴遇到的势垒同样阻碍了P区大量空穴流入N区\footnote{空穴能带的正负方向与电子相反,而我们看到的能带图都是从电子视角绘制的。}。由此,势垒维持了热平衡。

那为何正偏时PN结中会产生电流?\xref{fig:正偏PN结}展现了PN结正偏时的能带图。由于正偏时,外加电压削弱了势垒,而更小的势垒意味着耗尽区电场减小,进而导致其无法再阻止P区空穴和N区电子。因此将产生“P区至N区的电子扩散流”和“N区至P区的空穴扩散流”。这些载流子的扩散就在PN结中形成了一个由P区至N区的电流,其方向与电压方向一致。

\begin{Figure}[PN结能带图]
    \begin{FigureSub}[零偏PN结]
        \includegraphics[width=0.45\linewidth]{build/Chapter01A_01.fig.pdf}
    \end{FigureSub}
    \hspace{0.05\linewidth}
    \begin{FigureSub}[正偏PN结]
        \includegraphics[width=0.45\linewidth]{build/Chapter01A_02.fig.pdf}
    \end{FigureSub}
\end{Figure}

注入N区的电子是N区少子,注入P区的电子是P区少子。而少数载流子的行为可以通过在半导体物理中已经学习过的\uwave{连续性方程}(Continous Equations)描述,在这些区域中,过剩载流子将发生漂移--扩散--复合行为,形成电流。而PN结的电压--电流关系将在下一节讨论。

\begin{BoxEquation}[连续性方程]
    连续性方程描述了半导体中少数载流子的时空分布规律。

    对于空穴,这是
    \begin{Equation}
        D_p\pdv[2]{(\fdd{p})}{x}-\mu_p\qty[\E\pdv{(\fdd{p})}{x}+p\pdv{\E}{x}]+g_p'-\frac{\fdd{p}}{\tau_{p0}}=\pdv{(\fdd{p})}{t}
    \end{Equation}
    对于电子,这是
    \begin{Equation}
        D_n\pdv[2]{(\fdd{n})}{x}-\mu_n\qty[\E\pdv{(\fdd{n})}{x}+n\pdv{\E}{x}]+g_n'-\frac{\fdd{n}}{\tau_{n0}}=\pdv{(\fdd{n})}{t}
    \end{Equation}
\end{BoxEquation}

\subsection{PN结的理想假设}
PN结的理想电流--电压关系是在以下四个假设的基础上推导的
\begin{enumerate}
    \item 耗尽区适用突变边界近似,耗尽区之外的半导体视为中性区域,换言之,不存在漂移。
    \item 适用麦克斯韦--玻尔兹曼载流子统计近似,即半导体是非简并的。
    \item 适用小注入和完全电离。
    \item PN结总电流为常量,电子电流与空穴电流,在耗尽区内为常量,在耗尽区外连续。
\end{enumerate}

\subsection{PN结的边界条件}
我们已经知道,PN结的\uwave{内建电势}(Built-in Potential Barrier)$V_{bi}$由以下公式决定。
\begin{BoxFormula}[PN结的内建电势]
    PN结的内建电势$V_{bi}$由以下公式决定
    \begin{Equation}
        V_{bi}=\frac{\kB T}{e}\ln(\frac{N_aN_d}{n_i^2})
    \end{Equation}
\end{BoxFormula}
现在要解决的两个问题是
\begin{enumerate}
    \item PN结在平衡状态下,电子和空穴在其多子侧和少子侧分别为常量,且是已知的。这很显然,电子$n_{n0}=N_d, n_{p0}=n_i^2/N_a$,空穴$p_{p0}=N_a, p_{n0}=n_i^2/N_d$,该结果实际与PN结没有任何关系,仅来自P区和N区作为独立的杂质半导体的特性。但是,这并不妨碍我们将试图用PN结内建电势$V_{bi}$来描述$n_{n0},n_{p0}$以及$p_{p0},p_{n0}$,即平衡载流子多子侧浓度和少子侧浓度间的关系,因为依照\xref{fml:PN结的内建电势},本质上$V_{bi}$仍然是关于$N_d,N_a,n_i^2$的。
    \item PN结在正偏状态下,势垒将被削弱,电子和空穴将从其多子侧注入少子侧,这意味着载流子在少子侧不再是$n_{p0}$或$p_{n0}$的常量,而是随$x$变化的$n_{p},p_{n}$的变量。在本小节关注的是$n_{p},p_{n}$的边界条件,即耗尽区边界处$n_p(-x_p),p_n(+x_n)$与$n_{p0}, p_{n0}$的关系是什么?
    % \item PN结在正偏状态下,势垒将被削弱,电子和空穴将从其多子侧注入少子侧,这意味着载流子在少子侧不再是$n_{p0}$或$p_{n0}$的常量,而是随$x$变化的$n_{p},p_{n}$的变量。正如小节标题所写,这里我们关注的是$n_{p},p_{n}$的边界条件。很明显,当$x\to\pm\infty$时,注入的影响将趋于零,少子浓度将回归其平衡浓度,即$n_p(-\infty)=n_{p0}$和$p_n(+\infty)=p_{n0}$,这就是无穷远处的边界条件。那么,由于势垒减弱导致的电注入(具体而言,势垒减弱后,载流子将从多子侧穿过耗尽区到达少子侧边界,这增加了少子侧边界处载流子浓度),在耗尽区边界处的$n_p(-x_p)$和$p_n(+x_n)$与$n_{p0}, p_{n0}$的关系是什么?相较于$n_{p0}, p_{n0}$增加了多少?
\end{enumerate}

\begin{BoxFormula}[PN结的平衡少子多子比]
    PN结在平衡态时,同种载流子在少子侧和多子侧的浓度比,可用$V_{bi}$表示为
    \begin{Equation}
        n_{p0}=n_{n0}\exp(-\frac{eV_{bi}}{\kB T})\qquad
        p_{n0}=p_{p0}\exp(-\frac{eV_{bi}}{\kB T})
    \end{Equation}
\end{BoxFormula}

\begin{Proof}
    根据\fancyref{fml:PN结的内建电势}
    \begin{Equation}&[1]
        V_{bi}=\frac{\kB T}{e}\ln(\frac{N_aN_d}{n_i^2})
    \end{Equation}
    稍作变换
    \begin{Equation}&[2]
        \frac{N_aN_d}{n_i^2}=\exp(\frac{eV_{bi}}{\kB T})
    \end{Equation}
    或
    \begin{Equation}&[3]
        \frac{n_i^2}{N_aN_d}=\exp(-\frac{eV_{bi}}{\kB T})
    \end{Equation}
    我们已知
    \begin{Equation}&[4]
        \begin{cases}
            n_{n0}=N_d\\
            n_{p0}=n_i^2/N_a
        \end{cases}\qquad
        \begin{cases}
            p_{p0}=N_a\\
            p_{n0}=n_i^2/N_d
        \end{cases}
    \end{Equation}
    将\xrefpeq{4}分别代入\xrefpeq{3}中,分别得到
    \begin{Equation}
        \frac{n_{p0}}{n_{n0}}=\exp(-\frac{eV_{bi}}{\kB T})\qquad
        \frac{p_{n0}}{p_{p0}}=\exp(-\frac{eV_{bi}}{\kB T})
    \end{Equation}
    即
    \begin{Equation}*
        n_{p0}=n_{n0}\exp(-\frac{eV_{bi}}{\kB T})\qquad
        p_{n0}=p_{p0}\exp(-\frac{eV_{bi}}{\kB T})\qedhere
    \end{Equation}
\end{Proof}

\begin{BoxFormula}[PN结的边界条件]
    PN结在正偏时,当外加正向电压为$V_a$时,少子浓度在耗尽区边界满足
    \begin{Equation}&[A]
        n_p(-x_p)=n_{p0}\exp(\frac{eV_a}{\kB T})\qquad
        p_n(+x_n)=p_{n0}\exp(\frac{eV_a}{\kB T})
    \end{Equation}
\end{BoxFormula}

\begin{Proof}
    依据\fancyref{fml:PN结的平衡少子多子比}
    \begin{Equation}&[1]
        n_{p0}=n_{n0}\exp(-\frac{eV_{bi}}{\kB T})\qquad
        p_{n0}=p_{p0}\exp(-\frac{eV_{bi}}{\kB T})
    \end{Equation}
    这是平衡态的结果,在正偏时,势垒$V_{bi}$减小至$V_{bi}-V_a$,即
    \begin{Equation}&[2]
        \qquad\qquad
        n_{p}(-x_p)=n_{n0}\exp[-\frac{e(V_{bi}-V_a)}{\kB T}]\qquad
        p_{n}(+x_n)=p_{p0}\exp[-\frac{e(V_{bi}-V_a)}{\kB T}]
        \qquad\qquad
    \end{Equation}
    稍作整理
    \begin{Equation}&[3]
        n_{p}(-x_p)=n_{n0}\exp(-\frac{eV_{bi}}{\kB T})\exp(\frac{V_a}{\kB T})\qquad
        p_{n}(+x_n)=p_{p0}\exp(-\frac{eV_{bi}}{\kB T})\exp(\frac{V_a}{\kB T})
    \end{Equation}
    将\xrefpeq{1}代入\xrefpeq{3}
    \begin{Equation}*
        n_p(-x_p)=n_{p0}\exp(\frac{eV_a}{\kB T})\qquad
        p_n(+x_n)=p_{n0}\exp(\frac{eV_a}{\kB T})
        \qedhere
    \end{Equation}
\end{Proof}

\subsection{PN结的载流子分布}
在本节,我们将在\xref{fml:PN结的边界条件}的基础上,进一步推出PN结中少数载流子的浓度分布。

\begin{BoxFormula}[PN结的载流子分布]
    PN结在正偏时,少子浓度分别满足
    \begin{Align}[20pt]
        \fdd{p_n(x)}=p_{n0}&\qty[\exp(\frac{eV_a}{\kB T})-1]\exp(\frac{x_n-x}{L_p})\qquad x\geq +x_n\\
        \fdd{n_p(x)}=n_{p0}&\qty[\exp(\frac{eV_a}{\kB T})-1]\exp(\frac{x_p+x}{L_n})\qquad x\leq -x_p
    \end{Align}
\end{BoxFormula}

\begin{Proof}
    根据\fancyref{eqt:连续性方程},空穴的连续性方程满足
    \begin{Equation}&[1]
        D_p\pdv[2]{(\fdd{p_n})}{x}-\mu_p\qty[\E\pdv{(\fdd{p_n})}{x}+p\pdv{\E}{x}]+g_p'-\frac{\fdd{p_n}}{\tau_{p0}}=\pdv{(\fdd{p_n})}{t}
    \end{Equation}
    依照\xref{subsec:PN结的理想假设}的假设,耗尽区外无电场$\E=0$且$g_p'=0$,若再假设稳态$\pdv*{(\fdd{p_n})}{t}=0$
    \begin{Equation}&[2]
        D_p\dv[2]{(\fdd{p_n})}{x}-\frac{\fdd{p_n}}{\tau_{p0}}=0\qquad x>+x_n
    \end{Equation}
    两端同除$D_p$,引入扩散长度$L_p=\sqrt{D_p\tau_{p0}}$
    \begin{Equation}&[3]
        \dv[2]{(\fdd{p_n})}{x}-\frac{\fdd{p_n}}{L_p^2}=0\qquad x>+x_n
    \end{Equation}
    类似亦可得到电子的情形
    \begin{Equation}&[4]
        \dv[2]{(\fdd{n_p})}{x}-\frac{\fdd{n_p}}{L_n^2}=0\qquad x<-x_p
    \end{Equation}
    \xrefpeq{3}的通解为
    \begin{Equation}&[5]
        \fdd{p_n(x)}=P_1\exp(\frac{x}{L_p})+P_2\exp(-\frac{x}{L_p})\qquad x\geq +x_n
    \end{Equation}
    \xrefpeq{4}的通解为
    \begin{Equation}&[6]
        \fdd{n_p(x)}=N_1\exp(\frac{x}{L_n})+N_2\exp(-\frac{x}{L_n})\qquad x\leq -x_p
    \end{Equation}
    依据\fancyref{fml:PN结的边界条件},相应的边界条件为
    \begin{Gather}[12pt]
        \fdd{p_n(+x_n)}=p_{n0}\qty[\exp(\frac{eV_a}{\kB T})-1]\xlabelpeq{7}\\
        \fdd{n_p(-x_p)}=n_{p0}\qty[\exp(\frac{eV_a}{\kB T})-1]\xlabelpeq{8}\\
        \fdd{p_n}(+\infty)=0\xlabelpeq{9}\\
        \fdd{n_p}(-\infty)=0\xlabelpeq{10}
    \end{Gather}
    当少数载流子由空间电荷区边界向中性区扩散时,它们将与多数载流子复合并减少。此处假定P型区和N型区的长度$W_p,W_n$远大于扩散长度$L_p,L_n$,故过剩少数载流子$\fdd{p_n}, \fdd{n_p}$在远离空间电荷区时将趋于零,这才有上述$\fdd{p_n}(+\infty)=0$和$\fdd{n_p}(-\infty)=0$的无穷边界条件。

    以空穴为例求解,将\xrefpeq{9}代入\xrefpeq{5},容易定出$P_1=0$,从而
    \begin{Equation}&[11]
        \fdd{p_n(x)}=P_2\exp(-\frac{x}{L_p})
    \end{Equation}
    特别的,当$x=+x_n$时
    \begin{Equation}&[12]
        \fdd{p_n(+x_n)}=P_2\exp(-\frac{x_n}{L_p})
    \end{Equation}
    而\xrefpeq{7}又告诉我们
    \begin{Equation}&[13]
        \fdd{p_n(+x_n)}=p_{n0}\qty[\exp(\frac{eV_a}{\kB T})-1]
    \end{Equation}
    对比\xrefpeq{12}和\xrefpeq{13}可知
    \begin{Equation}&[14]
        P_2=p_{n0}\exp(\frac{x_n}{L_p})\qty[\exp(\frac{eV_a}{\kB T})-1]
    \end{Equation}
    将\xrefpeq{14}代回\xrefpeq{11}
    \begin{Equation}*
        \fdd{p_n(x)}=p_{n0}\qty[\exp(\frac{eV_a}{\kB T})-1]\exp(\frac{x_n-x}{L_p})
    \end{Equation}
    类似也可以得到
    \begin{Equation}*
        \fdd{n_p(x)}=n_{p0}\qty[\exp(\frac{eV_a}{\kB T})-1]\exp(\frac{x_p+x}{L_n})\qedhere
    \end{Equation}
\end{Proof}

\xref{fig:PN结的载流子分布}形象展示了\xref{subsec:PN结的边界条件}和\xref{subsec:PN结的载流子分布}的工作
\begin{itemize}
    \item 平衡少子浓度$p_{n0}, n_{p0}$。
    \item 平衡多子浓度$p_{p0}, n_{n0}$,在$p_{n0}, n_{p0}$上乘$\exp(eV_{bi})$。
    \item 非平衡态下,少子在耗尽区边界处浓度$p_{n}(+x_n), n_p(-x_p)$,在$p_{n0}, n_{p0}$上乘$\exp(eV_a)$。
    \item 非平衡少子$\fdd{p_n}(x), \fdd{n_p}(x)$由边界$+x_n, -x_p$向无穷远$\pm\infty$以指数方式衰减。
\end{itemize}

\begin{Figure}[PN结的载流子分布]
    \includegraphics[scale=1]{build/Chapter01A_03.fig.pdf}
\end{Figure}

\subsection{PN结的理想电流--电压关系}
推导PN结理想电流--电压关系的关键在于\xref{subsec:PN结的理想假设}第四条假设。试想,PN结的总电流密度$J(x)$是由空穴电流$J_p(x)$和电子电流$J_n(x)$的和组成的。而依据第四条假设
\begin{itemize}
    \item $J(x)$是无关$x$的常数$J$,那我们不妨就在耗尽区内计算总电流密度$J(x)=J$。
    \item $J_p(x)$和$J_n(x)$在耗尽区内也是常数,故不妨取边界处计算,即$J=J_p(+x_n)+J_n(-x_p)$。
    \item $J_p(x)$和$J_n(x)$是由少子扩散形成的扩散电流,正比于$p_n(x), n_p(x)$的梯度。
\end{itemize}

以上就是本小节推导$J, J_p(x), J_n(x)$表达式的基本思路。

\begin{BoxEquation}[PN结的理想电流--电压关系]*
    PN结的\uwave{理想电流--电压关系}(Ideal Current-Voltage Relationship)为
    \begin{Equation}
        J=J_S\qty[\exp(\frac{eV_a}{\kB T})-1]
    \end{Equation}
    其中$J_S$称为\uwave{理想反向饱和电流密度}(Ideal Reverse Saturation Current Density)
    \begin{Equation}
        J_S=\qty[\frac{eD_pp_{n0}}{L_p}+\frac{eD_nn_{p0}}{L_n}]
    \end{Equation}
    空穴电流$J_p(x)$满足
    \begin{Equation}
        \qquad\qquad
        J_p(x)=\frac{eD_pp_{n0}}{L_p}\qty[\exp(\frac{eV_a}{\kB T})-1]\exp(\frac{x_n-x}{L_p})\qquad
        x\geq +x_n
        \qquad\qquad
    \end{Equation}
    电子电流$J_n(x)$满足
    \begin{Equation}
        \qquad\qquad
        J_n(x)=\frac{eD_nn_{p0}}{L_n}\qty[\exp(\frac{eV_a}{\kB T})-1]\exp(\frac{x_p+x}{L_n})\qquad
        x\geq -x_p
        \qquad\qquad
    \end{Equation}
\end{BoxEquation}

\begin{Proof}
    空穴扩散电流为
    \begin{Equation}&[1]
        J_p(x)=-eD_p\dv{p_n(x)}{x}
    \end{Equation}
    由于考虑的是均匀掺杂区域,平衡少子浓度$p_{n0}$为常量,故$p_n(x)$可用$\fdd{n_p(x)}$替代
    \begin{Equation}&[2]
        J_p(x)=-eD_p\dv{(\fdd{p_n(x)})}{x}
    \end{Equation}
    代入\fancyref{fml:PN结的载流子分布}
    \begin{Equation}&[3]
        J_p(x)=\frac{eD_pp_{n0}}{L_p}\qty[\exp(\frac{eV_a}{\kB T})-1]\exp(\frac{x_n-x}{L_p})
    \end{Equation}
    电子扩散电流可以通过类似方法得到
    \begin{Equation}&[4]
        J_n(x)=\frac{eD_nn_{p0}}{L_n}\qty[\exp(\frac{eV_a}{\kB T})-1]\exp(\frac{x_p+x}{L_n})
    \end{Equation}
    在\xrefpeq{3}中取$x=+x_n$
    \begin{Equation}&[5]
        J_p(+x_n)=\frac{eD_pp_{n0}}{L_p}\qty[\exp(\frac{eV_a}{\kB T})-1]
    \end{Equation}
    在\xrefpeq{4}中取$x=-x_p$
    \begin{Equation}&[6]
        J_n(-x_p)=\frac{eD_nn_{p0}}{L_n}\qty[\exp(\frac{eV_a}{\kB T})-1]
    \end{Equation}
    将\xrefpeq{5}和\xrefpeq{6}相加
    \begin{Equation}&[7]
        \qquad\qquad
        J=J_p(+x_n)+J_n(-x_p)=\qty[\frac{eD_pp_{n0}}{L_p}+\frac{eD_nn_{p0}}{L_n}]\qty[\exp(\frac{eV_a}{\kB T})-1]
        \qquad\qquad
    \end{Equation}
    若引入代换变量$J_s$
    \begin{Equation}*
        J_S=\qty[\frac{eD_pp_{n0}}{L_p}+\frac{eD_nn_{p0}}{L_n}]
    \end{Equation}
    则\xrefpeq{7}可以简化为
    \begin{Equation}*
        J=J_S\qty[\exp(\frac{eV_a}{\kB T})-1]\qedhere
    \end{Equation}
\end{Proof}

PN结的理想电流--电压关系如\xref{fig:PN结的理想电流--电压关系}所示。尽管上述推导均是在正偏$V_a>0$的背景下进行的,但是反偏$V_a<0$也完全适用该结论。值得注意的是,当$V_a$反偏超过数个$\kB T/eV$后,电流密度$J$将趋于常量$-J_S$,不再随反偏电压变化,故$J_S$也被称为理想反向饱和电流密度。而同时,当$V_a$正偏超过数个$\kB T/eV$后,则$-1$可以被忽略,此时可认为$J\propto\exp(eV_a/\kB T)$。

\begin{Figure}[PN结的理想电流--电压关系]
    \includegraphics[scale=1]{build/Chapter01A_05.fig.pdf}
\end{Figure}

PN结的空穴扩散电流$J_p(x), x\geq +x_n$和电子扩散电流$J_n(x), x\leq -x_p$在中性区内向两端指数衰减,但根据前面的假设,PN结的总电流密度$J$是一个常量,这该怎么解释呢?总电流密度与少子扩散电流密度间的差,来自多子漂移电流。它们补充了多子因注入而造成的损失。

\begin{Figure}[PN结的电流密度分布]
    \includegraphics[scale=1]{build/Chapter01A_04.fig.pdf}
\end{Figure}

当然,应当指出的是,多子漂移电流的存在意味着我们先前中性区的假设并不严谨,因为存在漂移电流就意味着空间电荷区外仍有电场。不过这电场实际很小,故这个假设仍可以适用。

\subsection{短二极管}
在前面的分析中,我们假设P型区域和N型区域的长度$W_p, W_n$都远大于相应的少子扩散长度$L_n, L_p$,即$W_p\gg L_n$且$W_n\gg L_p$(注意此处两者是相反的)。然而,在许多实际PN结的结构中,有一个区域,反而远小于相应的少子扩散长度,即$W_p\ll L_n$或$W_n\ll L_p$。这类PN结被称为\uwave{短二极管}(The Short Diode),相较于通常的\uwave{长二极管}(The Long Diode)。短二极管将导致许多问题,\fancyref{fml:PN结的载流子分布}中,应用了\empx{过剩载流子浓度在无穷远处减少至零}的边界条件,然而在短二极管的短侧,由于区域长度远小于扩散长度,因此边界条件需要相应改为\empx{过剩载流子浓度在区域边界处减小至零}。在这一新边界条件下,我们将会重新推导出短二极管的载流子分布和电流密度分布函数。\xref{fig:短二极管与长二极管}给出了短二极管的结构示意图。

\begin{Figure}[短二极管与长二极管]
    \begin{FigureSub}[长二极管]
        \includegraphics[scale=0.9]{build/Chapter01A_06.fig.pdf}
    \end{FigureSub}\\ \vspace{0.5cm}
    \begin{FigureSub}[短二极管(N侧)]
        \includegraphics[scale=0.9]{build/Chapter01A_07.fig.pdf}
    \end{FigureSub}\hspace{0.25cm}
    \begin{FigureSub}[短二极管(P侧)]
        \includegraphics[scale=0.9]{build/Chapter01A_08.fig.pdf}
    \end{FigureSub}
\end{Figure}\vspace{0.15cm}

\begin{BoxFormula}[短二极管的载流子分布]
    对于N侧较短$W_n\ll L_p$的短二极管,其过剩空穴分布$\fdd{p_n}$需修正为
    \begin{Equation}
        \fdd{p_n}(x)=p_{n0}\qty[\exp(\frac{eV_a}{\kB T})-1]\qty(\frac{W_n+x_n-x}{W_n})
    \end{Equation}
    对于P侧较短$W_p\ll L_n$的短二极管,其过剩电子分布$\fdd{n_p}$需修正为
    \begin{Equation}
        \fdd{n_p}(x)=n_{p0}\qty[\exp(\frac{eV_a}{\kB T})-1]\qty(\frac{W_p+x_p+x}{W_p})
    \end{Equation}
\end{BoxFormula}

\begin{Proof}
    以空穴的分布为例推导,连续性方程的通解仍然是\xrefpeq[PN结的载流子分布]{5}
    \begin{Equation}&[1]
        \fdd{p_n(x)}=P_1\exp(\frac{x}{L_p})+P_2\exp(-\frac{x}{L_p})
    \end{Equation}
    在耗尽区边界处的边界条件\xrefpeq[PN结的载流子分布]{7}仍然满足
    \begin{Equation}&[2]
        \fdd{p_n(x_n)}=p_{n0}\qty[\exp(\frac{eV_a}{\kB T})-1]
    \end{Equation}
    在无穷远处的边界条件则不再成立,取而代之,$\fdd{p_n}$在N区边界$x_n+W$处衰减至零
    \begin{Equation}&[3]
        \fdd{p_n(x_n+W)}=0
    \end{Equation}
    将\xrefpeq{2}和\xrefpeq{3}代入\xrefpeq{1},构建方程阻
    \begin{Equation}&[4]
        \begin{pmatrix}
            \mal{\exp(\frac{x_n}{L_p})}&
            \mal{\exp(-\frac{x_n}{L_p})}\\[6mm]
            \mal{\exp(\frac{x_n+W_n}{L_p})}&
            \mal{\exp(-\frac{x_n+W_n}{L_p})}
        \end{pmatrix}
        \begin{pmatrix}
            P_1\vphantom{\mal{\qty(\frac{1}{1})}}\\[6mm]
            P_2\vphantom{\mal{\qty(\frac{1}{1})}}
        \end{pmatrix}
        =
        \begin{pmatrix}
            \mal{p_{n0}\qty[\exp(\frac{eV_a}{\kB T})-1]}\\[6mm]
            0\vphantom{\mal{\qty(\frac{1}{1})}}
        \end{pmatrix}
    \end{Equation}
    计算$D$
    \begin{Equation}&[5]
        D=\begin{vmatrix}
            \mal{\exp(\frac{x_n}{L_p})}&
            \mal{\exp(-\frac{x_n}{L_p})}\\[6mm]
            \mal{\exp(\frac{x_n+W_n}{L_p})}&
            \mal{\exp(-\frac{x_n+W_n}{L_p})}
        \end{vmatrix}=
        \exp(-\frac{W_n}{L_p})-\exp(\frac{W_n}{L_p})=-2\sinh(\frac{W_n}{L_p})
    \end{Equation}
    计算$D_1$
    \begin{Equation}&[6]
        D_1=\begin{vmatrix}
            \mal{p_{n0}\qty[\exp(\frac{eV_a}{\kB T})-1]}&
            \mal{\exp(-\frac{x_n}{L_p})}\\[6mm]
            0&
            \mal{\exp(-\frac{x_n+W_n}{L_p})}    
        \end{vmatrix}=
        p_{n0}\qty[\exp(\frac{eV_a}{\kB T})-1]\exp(-\frac{x_n+W_n}{L_p})
    \end{Equation}
    计算$D_2$
    \begin{Equation}&[7]
        D_2=\begin{vmatrix}
            \mal{\exp(\frac{x_n}{L_p})}&
            \mal{p_{n0}\qty[\exp(\frac{eV_a}{\kB T})-1]}\\[6mm]
            \mal{\exp(\frac{x_n+W_n}{L_p})}&
            0
        \end{vmatrix}=
        -p_{n0}\qty[\exp(\frac{eV_a}{\kB T})-1]\exp(\frac{x_n+W_n}{L_p})
    \end{Equation}
    其中,$P_1=D_1/D$,$P_2=D_2/D$,故\xrefpeq{1}表示为
    \begin{Equation}&[8]
        \fdd{p_n(x)}=\frac{D_1}{D}\exp(\frac{x}{L_p})+\frac{D_2}{D}\exp(-\frac{x}{L_p})
    \end{Equation}
    将\xrefpeq{5}、\xrefpeq{6}、\xrefpeq{7}代入\xrefpeq{8}
    \begin{Equation}&[9]
        \fdd{p_n(x)}=p_{n0}\qty[\exp(\frac{eV_a}{\kB T})-1]\qty[\exp(-\frac{x_n+W_n-x}{L_p})-\exp(\frac{x_n+W_n-x}{L_p})]\frac{1}{-2\sinh(W_n/L_p)}
    \end{Equation}
    再次用双曲正弦简化
    \begin{Equation}
        \fdd{p_n(x)}=p_{n0}\qty[\exp(\frac{eV_a}{\kB T})-1]\frac{\sinh[(W_n+x_n-x)/L_p]}{\sinh[W_n/L_p]}
    \end{Equation}
    而当$W_n\ll L_p$时,可以进一步近似为
    \begin{Equation}*
        \fdd{p_n(x)}=p_{n0}\qty[\exp(\frac{eV_a}{\kB T})-1]\qty(\frac{W_n+x_n-x}{W_n})
    \end{Equation}
    而对于$W_p\ll L_n$的情况,类似可以正面$\fdd{n_p(x)}$满足
    \begin{Equation}*
        \fdd{n_p(x)}=n_{p0}\qty[\exp(\frac{eV_a}{\kB T})-1]\qty(\frac{W_p+x_p+x}{W_p})\qedhere
    \end{Equation}
\end{Proof}

\begin{BoxFormula}[短二极管的电流密度]
    对于N侧较短$W_n\ll L_p$的短二极管,其$J_p(x)$应修正为
    \begin{Equation}
        J_p(x)=\frac{eD_pp_{n0}}{W_n}\qty[\exp(\frac{eV_a}{\kB T})-1]
    \end{Equation}
    对于P侧较短$W_p\ll L_n$的短二极管,其$J_n(x)$应修正为
    \begin{Equation}
        J_n(x)=\frac{eD_nn_{p0}}{W_p}\qty[\exp(\frac{eV_a}{\kB T})-1]
    \end{Equation}
\end{BoxFormula}

\begin{Proof}
    以空穴而的分布为例推导,根据\fancyref{fml:短二极管的载流子分布}
    \begin{Equation}&[1]
        \fdd{p_n}(x)=p_{n0}\qty[\exp(\frac{eV_a}{\kB T})-1]\qty(\frac{W_n+x_n-x}{W_n})
    \end{Equation}
    空穴扩散电流为
    \begin{Equation}&[2]
        J_p(x)=-eD_p\dv{(\fdd{p_n(x)})}{x}
    \end{Equation}
    代入\xrefpeq{1}
    \begin{Equation}*
        J_p(x)=\frac{eD_pp_{n0}}{W_n}\qty[\exp(\frac{eV_a}{\kB T})-1]
    \end{Equation}
    电子扩散电流可以类似求得
    \begin{Equation}*
        J_n(x)=\frac{eD_nn_{p0}}{W_p}\qty[\exp(\frac{eV_a}{\kB T})-1]\qedhere
    \end{Equation}
\end{Proof}

通过\fancyref{fml:短二极管的载流子分布}和\fancyref{fml:短二极管的电流密度}
\begin{itemize}
    \item 长二极管至短二极管,载流子分布由指数衰减变为了线性衰减。
    \item 长二极管至短二极管,电流密度分布由指数衰减变为了常量。
\end{itemize}
除此之外,长二极管的公式包含$L_p,L_n$,短二极管的公式则包含$W_n,W_p$,这是因为在短二极管中区域长度相较扩散长度是更重要的因素。同时,我们刚刚已提到,在短二极管中,少数载流子分布变为线性,少子扩散电流密度变为常量,而恒定电流密度表明,\empx{短区中不存在复合}。

