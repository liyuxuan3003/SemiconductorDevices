\section{BJT的基本结构}

\subsection{BJT的结构}

BJT的结构如\xref{fig:BJT的结构简图}所示,其具有三个性质不同的掺杂区,包含两个PN结。



\begin{Figure}[BJT的结构简图]
    \begin{FigureSub}[NPN的结构简图]
        \includegraphics[scale=0.87]{build/Chapter04A_01.fig.pdf}
    \end{FigureSub}
    \hspace{0.2cm}
    \begin{FigureSub}[PNP的结构简图]
        \includegraphics[scale=0.87]{build/Chapter04A_02.fig.pdf}
    \end{FigureSub}
\end{Figure}

BJT可以分为NPN和PNP两种类型,我们主要以NPN型为例研究

\begin{itemize}
    \item 对于NPN型,三个掺杂区依次为N型、P型、N型,即两个共阳极的PN结。
    \item 对于PNP\hspace{0.4em}型,三个掺杂区依次为P型、N型、P型,\hspace{0.1em}即两个共阴极的PN结。
\end{itemize}

BJT的三个区域依次为:\uwave{发射极}(Emitter, E)、\uwave{基极}(Base, B)、\uwave{集电极}(Collector, C)。但是,由于原理上的需要,这三个区域的性质是不对称的,这包含区域的掺杂浓度和宽度两方面
\begin{itemize}
    \item 就掺杂浓度而言,有$\te{E}>\te{B}>\te{C}$,如\xref{fig:NPN的结构简图},三者用$n^{++}, p^{+},n$表示。通常而言,三者的典型掺杂浓度是$\si{10^{19} cm^{-3}}$、$\si{10^{17} cm^{-3}}$、$\si{10^{15} cm^{-3}}$,其中尤为重要的是发射区重掺。
    \item 就区域宽度而言,有$\te{C}>\te{E}>\te{B}$,其中尤为重要的是基区的宽度必须非常窄。
\end{itemize}
由此可见,BJT在结构上并不对称,发射极和集电极是不对等的,不可以交换。

BJT中,BE结称为\uwave{发射结}(Emitter Junction),BC结称为\uwave{集电结}(Collector Junction)。我们后面会看到,取决于这两个结是正偏和反偏,将最终给出BJT的四种工作状态。但眼下我们不需要了解这么多,我们只需要知道,在BJT最常用的工作区,即正向放大区下,两个结的工作状态是:\empx{发射结正偏,集电结反偏}。发射结发挥了重要的作用,因此,如\xref{fig:BJT的电路符号}中,我们用箭头标识发射结的正偏方向,有箭头一侧的为E,无箭头的一侧为B,而箭头的方向能帮助我们区分该BJT到底是NPN还是PNP,由E至B即为NPN,由B至E即为PNP。

\begin{Figure}[BJT的电路符号]
    \begin{FigureSub}[NPN的电路符号]
        \includegraphics{build/Chapter04A_09.fig.pdf}
    \end{FigureSub}
    \hspace{1cm}
    \begin{FigureSub}[PNP的电路符号]
        \includegraphics{build/Chapter04A_10.fig.pdf}
    \end{FigureSub}
\end{Figure}

然而,我们要说明的是,\xref{fig:BJT的结构简图}只是一个简化示意图。实际的结构要复杂很多。如\xref{fig:BJT的结构}所示,首先,为了将BJT作在平面上,各区域并不是简单的线性排列。其次,集电极的主体尽管是轻掺的$n$型,但是,和金属接触处仍需要$n^{++}$重掺以构成欧姆接触。最后,底部为降低半导体的电阻,还需添加$n^{++}$重掺的掩埋层。不过\xref{fig:BJT的结构简图}对于的简图理论学习仍然是很有帮助的。

\begin{Figure}[BJT的结构]
    \begin{FigureSub}[NPN的结构]
        \includegraphics[scale=0.87]{build/Chapter04A_03.fig.pdf}
    \end{FigureSub}
    \hspace{0.2cm}
    \begin{FigureSub}[PNP的结构]
        \includegraphics[scale=0.87]{build/Chapter04A_04.fig.pdf}
    \end{FigureSub}
\end{Figure}

\subsection{BJT的原理}
现在的问题是,人们总是声称BJT可以放大,然而BJT到底如何工作?这里先来做一个简要的讨论。如\xref{fig:BJT的核心原理}所示,其中,$V_{BE}$和$V_{BC}$表示发射结和集电结的偏压,但由于我们更关心放大区,故用$+V_{BE}$和$-V_{BC}$标识“发射结正偏压”和“集电结负偏压”。$I_{E},I_{B},I_C$则是各极的电流。好!现在让我们先忘掉集电区的存在,只考虑发射结,很明显发射结不过是一个我们很熟悉的正偏PN结,从基区至发射区将有较大的正偏电流,且显然有$I_E=I_B$成立。由于发射结中,相对而言,发射区是$n^{++}$重掺的,因此,发射结的电流中主要以发射区至基区的电子电流为主。而引入集电区的存在后,一切都改变了。从表面上看,集电结不过是一个简单的反偏PN结,照道理基区和集电区之间应该是“不导通”的。问题是,反偏电流很小的原因仅仅在于少子很少,并没有什么实际阻碍,只要能补充少子,那么反偏电流同样可以很大。

现在让我们联合起来考虑NPN的特性:\empx{发射结被设计为重掺},发射结正偏时向基区输送了巨量电子,这些电子原本应通过基极被导出,然而,\empx{基区被设计的很窄},电子中只有很少的一部分会从基极流出,电子大部分都迅速穿过了整个基极到达了边界,这为反偏集电结提供了大量电子。在边界上,集电结反偏产生的强电场将这些电子扫入集电区,产生很大的反偏电流。

\begin{Figure}[BJT的核心原理]
    \includegraphics{build/Chapter04A_11.fig.pdf}
\end{Figure}

总结起来,在BJT中,正偏发射结的正偏电流$I_E$的电子只有很少的部分流向$I_B$,大部分都继续流向了集电区形成$I_C$,原因是,反偏集电结的强电场从基区中“窃取”了大部分电子流。
\begin{Equation}
    I_{E}\gg I_{B}\qquad I_{E}\approx I_C
\end{Equation}
这个比例是相当悬殊的,通常我们记
\begin{Equation}
    I_C=\beta I_B
\end{Equation}
这里的$\beta$通常可以达到50至200,是一个相当大的电流增益,这就是BJT的放大原理!简而言之,BJT可以将基极电流$I_B$放大将近百倍形成集电极电流$I_C$,以小电流控制大电流。

这里我们看到,“发射结”和“集电结”这两个名称是非常恰当的
\begin{itemize}
    \item 发射结向基区“发射”电子,形成正偏电流$I_E$。
    \item 集电结从基区“收集”电子,形成反偏电流$I_C$。
\end{itemize}
有关电流的朝向可能会让我们困惑,在\xref{fig:BJT的核心原理}中,我们可能会觉得是$I_E$是由$I_C$和$I_B$汇聚而成的。但通过上述讨论我们知道,电子流在这里发挥主要作用,电子流的方向与电流相反,因此从电子流的观点看,实质上是$I_E$的电子流分散为$I_C$和$I_B$。这种矛盾并不是特别要紧,我们可以“谁产生谁”或“谁导致谁”的因果思维直观理解物理过程,但是,这之后,我们要始终牢记物理的本质是数量关系而非因果。这里唯一真实的是$I_E=I_{B}+I_C$而不是汇聚或分散。


% 最后,让我们讨论一些一般性的东西。实际上,包括BJT、JFET、MOSFET在内的任何晶体管,从形式上而言都是两个方向相反的PN结,但由于这两个相反的PN结在结构上通过某种方式“紧密联系”起来,其所连接的两端(BJT称射集E,C,FET称源漏S,D)可以在第三端(BJT称基B,FET称栅G)的控制下导通,具有完全不同于孤立相反PN结的性质。

% 而所谓晶体管的类型,无非就是这种“紧密联系”的方式方法不同,

% \begin{Figure}[晶体管的一般结构]
%     \begin{FigureSub}[N型器件]
%         \includegraphics[scale=0.87]{build/Chapter04A_05.fig.pdf}
%     \end{FigureSub}
%     \hspace{0.2cm}
%     \begin{FigureSub}[N型器件]
%         \includegraphics[scale=0.87]{build/Chapter04A_06.fig.pdf}
%     \end{FigureSub}
% \end{Figure}

\subsection{BJT的能带}
令人惊讶的是,尽管BJT的原理比较费解,但是BJT的能带图却非常简单。

\begin{Figure}[BJT的能带]
    \begin{FigureSub}[截止区]
        \includegraphics[scale=0.9]{build/Chapter04A_07.fig.pdf}
    \end{FigureSub}
    \begin{FigureSub}[放大区]
        \includegraphics[scale=0.9]{build/Chapter04A_08.fig.pdf}
    \end{FigureSub}
\end{Figure}

在\xref{fig:截止区}和\xref{fig:放大区}中,分别展示了BJT在零偏和处于正向放大区(发射结正偏,集电结反偏)时的图像,我们看到,两者不过就是两个处于相应偏置的PN结的能带图的拼接罢了。

\subsection{BJT的四个工作区}
本节最后论述一下BJT的四个工作区。截止区是易于理解的,发射结反偏,集电结反偏,我们不可能指望从中得到任何东西,整个BJT处于关断状态。正向放大区前面在\xref{subsec:BJT的原理}中已经做了充分的论述,发射结正偏向基区注入电子,集电结反偏从基区抽取电子。饱和区则是很有意思的一个概念,关于“饱和区”中“饱和”的含义是一个常见的争议,它和MOSFET的饱和区完全不同,并非是指什么电流饱和了。而是指\cite{se:BJT饱和},当BJT由正向放大区转入饱和区时,集电结由反偏转入正偏,此时,集电结的耗尽区电场逐渐减弱,以至于集电结已经不足矣将基区大部分的电子抽取出来了。也就是说,BJT的饱和区,\empx{饱和的是集电结的集电能力}!
\begin{Figure}[BJT的四个工作区]
    \includegraphics{build/Chapter04A_12.fig.pdf}
\end{Figure}

除此之外,反向放大区和正向放大区的关系也是很有意思的,两者刚好相反
\begin{itemize}
    \item 正向放大区中,发射结正偏注入电子,集电结反偏抽取电子。
    \item 反向放大区中,集电结正偏抽取电子,发射结反偏抽取电子。
\end{itemize}
也就是说,反向放大区完全就是将正向放大区的模式颠倒过来,对调了发射结和集电结的功能。前面我们提到,BJT的结构并不对称,但平心而论,\empx{BJT的结构组成确实是对称的},然而,\empx{BJT的结构参数则是非对称的},现有的BJT的结构设计都是基于正向放大的使用的,例如发射结重掺就是为了能发射更多的电子至基区以提高增益$\beta$。因此,反向放大区从理论上确实能以相仿的原理工作,但是电流增益$\beta$等特性参数将很差。所以,不要使用反向放大区!

