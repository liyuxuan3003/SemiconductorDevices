\section{JFET的器件特性}
现在让我们来推导JFET的器件特性。我们或许已经注意到了JFET是上下对称的,即所谓的“双边JFET”。我们可以只考虑半个JFET来让计算更简单,即“单边HFET”,如\xref{fig:JFET的}

\begin{Figure}[JFET的简化]
    \begin{FigureSub}[双边JFET]
        \includegraphics[scale=0.85]{build/Chapter05B_01.fig.pdf}
    \end{FigureSub}\\ \vspace{0.5cm}
    \begin{FigureSub}[单边JFET]
        \includegraphics[scale=0.85]{build/Chapter05B_02.fig.pdf}
    \end{FigureSub}
\end{Figure}
以$I_{D2}$和$I_{D1}$分别表示双边JFET和单边JFET上的电流,显然
\begin{Equation}
    I_{D2}=2I_{D1}
\end{Equation}
这是因为双边JFET可以视为两个单边JFET的并联。后面的分析都对单边JFET进行。

\subsection{栅源夹断电压}
这一小节的目的是,推导$V_{GS}$为何值时,JFET沟道被夹断?
\begin{Figure}[栅源夹断电压的推导]
    \includegraphics{build/Chapter05B_03.fig.pdf}
\end{Figure}
如\xref{fig:栅源夹断电压的推导}所示,设JFET的沟道的总厚度为$a$,其中,设耗尽区占据的厚度为$h$。\setpeq{栅源夹断电压}

根据\fancyref{fml:反偏时的空间电荷区宽度},考虑到$V_{GS}$形式上为正偏电压,且$N_{a}\gg N_d$
\begin{Equation}&[1]
    h=\qty[\frac{2\epsilon_s(V_{bi}-V_{GS})}{eN_d}]^{1/2}
\end{Equation}
其中$V_{bi}$是栅--沟道PN结的内建电势差。

这里$a$是一个常量,当$h=a$时,就意味着耗尽区完全夹断了沟道,记此时的$V_{GS}$为$V_p$
\begin{Equation}&[2]
    a=\qty[\frac{2\epsilon_s(V_{bi}-V_{p})}{eN_d}]^{1/2}
\end{Equation}
这里还常引入$V_{p0}=V_{bi}-V_p$代换
\begin{Equation}&[3]
    a=\qty[\frac{2\epsilon_sV_{p0}}{eN_d}]^{1/2}
\end{Equation}
正式定义如下
\begin{BoxDefinition}[内建夹断电压]
    内建夹断电压$V_{p0}$定义为
    \begin{Equation}
        V_{p0}=V_{bi}-V_p
    \end{Equation}
    这里,$V_{bi}$是内建电压,$V_{p}$是夹断电压,$V_{p0}$是内建夹断电压。\footnote[2]{其中p代表的是夹断的英文pinch off。}
\end{BoxDefinition}

关于$V_p$和$V_{p0}$,我们可能会觉得很绕,可以通过以下方式记忆
\begin{itemize}
    \item $V_p$是夹断电压,它是一个负值,它的意义是当栅压达到$V_{GS}=V_p$的负值时,沟道夹断。
    \item $V_{p0}$是内建夹断电压,它和内建电压$V_{bi}$一样都是一个正值,且$V_{bi}<V_{p0}$,它的意义可以解释为,假如栅--沟道PN结的内建电压能到达$V_{p0}$,那么,在零栅压时耗尽区就已经挤占了整个沟道,发生夹断了。换言之,$V_{p0}=V_{bi}-V_{p}$代表夹断发生时的耗尽区电压。
\end{itemize}\setpeq{栅源夹断电压}
$V_{p0}$关于$V_p$的表达式为
\begin{Equation}
    V_{p0}=V_{bi}-V_p
\end{Equation}
$V_{p}$关于$V_{p0}$的表达式为
\begin{Equation}
    V_{p}=V_{bi}-V_{p0}
\end{Equation}
这是非常好记忆的:夹断电压$V_{p}$和内建夹断电压$V_{p0}$分别等于$V_{bi}$减去对方。

内建夹断电压$V_{p0}$是一个与总厚度$a$直接相关的量,由\xrefpeq{3}稍微做一些变换,就可以得到
\begin{BoxFormula}[内建夹断电压]
    内建夹断电压$V_{p0}$可以表示为
    \begin{Equation}
        V_{p0}=\frac{ea^2N_d}{2\epsilon_s}
    \end{Equation}
\end{BoxFormula}

现在,我们就可以明确写出栅源电压$V_{GS}$达到多少时,夹断会发生了。
\begin{BoxFormula}[JFET的栅源夹断电压]
    当栅源电压$V_{GS}$满足下式时,沟道夹断
    \begin{Equation}
        V_{GS}\leq V_p
    \end{Equation}
    即
    \begin{Equation}
        V_{GS}\leq V_{bi}-V_{p0}
    \end{Equation}
\end{BoxFormula}

\subsection{漏源饱和电压}
这一小节的目的是,推导$V_{DS}$为何值时,JFET沟道电流饱和?
\begin{Figure}[漏源夹断电压的推导]
    \includegraphics{build/Chapter05B_04.fig.pdf}
\end{Figure}
如\xref{fig:漏源夹断电压的推导}所示,此时耗尽区是倾斜的,宽度$h(x)$并不均等,源侧记为$h_1$,漏侧记为$h_2$。

在源侧,“反偏电压”就是$-V_{GS}$,故$h_1$和前面的$h$是相同的
\begin{BoxFormula}[JFET的源侧耗尽区厚度]
    JFET在源侧的耗尽区厚度$h_1$为
    \begin{Equation}
        h_1=
        \qty[\frac{2\epsilon_s(V_{bi}-V_{GS})}{eN_d}]^{1/2}
    \end{Equation}
\end{BoxFormula}
在漏侧,“反偏电压”增大至$V_{DS}-V_{GS}$,故$h_2$应改写为
\begin{BoxFormula}[JFET的漏侧耗尽区厚度]
    JFET在源侧的耗尽区厚度$h_2$为
    \begin{Equation}
        h_2=
        \qty[\frac{2\epsilon_s(V_{bi}+V_{DS}-V_{GS})}{eN_d}]^{1/2}
    \end{Equation}
\end{BoxFormula}\setpeq{漏源夹断电压}
很明显,漏源电压$V_{DS}$导致的那种使电流饱和的夹断的条件是$h_2=a$,即$V_{DS}$要满足
\begin{Equation}
    a=\qty[\frac{2\epsilon_s(V_{bi}+V_{DS}-V_{GS})}{eN_d}]^{1/2}
\end{Equation}
容易解出
\begin{Equation}
    V_{DS}=\frac{ea^2N_{d}}{2\varepsilon_{s}}-V_{bi}+V_{GS}
\end{Equation}
根据\fancyref{fml:内建夹断电压}
\begin{Equation}
    V_{DS}=V_{p0}-V_{bi}+V_{GS}
\end{Equation}
根据\fancyref{def:内建夹断电压}
\begin{Equation}
    V_{DS}=V_{GS}-V_{p}
\end{Equation}
由此,我们就可以归纳出电流饱和时$V_{DS}$的条件了
\begin{BoxFormula}[漏源饱和电压]
    当漏源电压$V_{DS}$满足下式时,沟道电流饱和
    \begin{Equation}
        V_{DS}\geq V_{GS}-V_{p}
    \end{Equation}
    即
    \begin{Equation}
        V_{DS}\geq V_{GS}-V_{bi}+V_{p0}
    \end{Equation}
\end{BoxFormula}

\subsection{JFET的电流--电压关系}
经过前面两个小节的准备后,现在我们可以正式开始推导JFET的电流--电压关系了。

\begin{BoxFormula}[JFET的电流--电压关系]*
    当$V_{GS}\leq V_{p}$时,JFET处于截止区
    \begin{Equation}&[1]
        I_{D1}=0
    \end{Equation}
    当$V_{GS}\geq V_{p}$且$V_{DS}\leq V_{GS}-V_{p}$时,JFET处于线性区
    \begin{Equation}&[2]
        \qquad\qquad
        I_{D1}=I_{P1}\qty[3\qty(\frac{V_{DS}}{V_{p0}})-2\qty(\frac{V_{DS}+V_{bi}-V_{GS}}{V_{p0}})^{3/2}+2\qty(\frac{V_{bi}-V_{GS}}{V_{p0}})^{3/2}]
        \qquad\qquad
    \end{Equation}
    当$V_{GS}\geq V_{p}$且$V_{DS}\geq V_{GS}-V_{p}$时,JFET处于饱和区
    \begin{Equation}&[3]
        I_{D1}=I_{P1}\qty[1-3\qty(\frac{V_{bi}-V_{GS}}{V_{p0}})+2\qty(\frac{V_{bi}-V_{GS}}{V_{p0}})^{3/2}]
    \end{Equation}
    其中$I_{P1}$为
    \begin{Equation}
        I_{P1}=e^2N_d^2\mu_nWa^3
    \end{Equation}
\end{BoxFormula}

\begin{Proof}
    现在考虑一个N沟道单边JFET,在沟道$x$处的微分电阻为
    \begin{Equation}&[1]
        \dd{R(x)}=\frac{\rho\dx}{A(x)}
    \end{Equation}
    其中,$\rho$为沟道电阻率,$A(x)$为$x$处的横截面积。

    沟道的电阻率可以由下式给出
    \begin{Equation}&[2]
        \rho=\frac{1}{\e\mu_nN_d}
    \end{Equation}
    沟道的横截面积可以由下式给出
    \begin{Equation}&[3]
        A(x)=\qty[a-h(x)]W
    \end{Equation}
    将\xrefpeq{2}和\xrefpeq{3}代入\xrefpeq{1}
    \begin{Equation}&[4]
        \dd{R(x)}=\frac{\dx}{e\mu_nN_d\qty[a-h(x)]W}
    \end{Equation}
    在$\dx$上的微分电压为
    \begin{Equation}&[5]
        \dd{V(x)}=I_{D1}\dd{R(x)}
    \end{Equation}
    在整个沟道中漏电流$I_{D1}$是一个常数。将\xrefpeq{4}代入\xrefpeq{5}
    \begin{Equation}&[6]
        \dd{V(x)}=\frac{I_{D1}\dx}{e\mu_nN_dW\qty[a-h(x)]}
    \end{Equation}
    或
    \begin{Equation}&[7]
        I_{D1}\dx=e\mu_nN_dW[a-h(x)]\dd{V(x)}
    \end{Equation}
    耗尽层宽度$h(x)$表示为
    \begin{Equation}&[8]
        h(x)=\qty[\frac{2\epsilon_s\qty(V(x)+V_{bi}-V_{GS})}{eN_d}]^{1/2}
    \end{Equation}
    两端平方得到
    \begin{Equation}&[9]
        h(x)^2=\frac{2\epsilon_s\qty(V(x)+V_{bi}-V_{GS})}{eN_d}
    \end{Equation}
    解得$V(x)$为
    \begin{Equation}&[10]
        V(x)=\frac{eN_dh(x)^2}{2\epsilon_s}+V_{GS}-V_{bi}
    \end{Equation}
    两端对$x$求微分
    \begin{Equation}&[11]
        \dd{V(x)}=\frac{eN_dh(x)\dd{h(x)}}{\epsilon_s}
    \end{Equation}
    将\xrefpeq{11}代入\xrefpeq{7},这样就将$\dd{V(x)}$都转化为了$h(x)$和$\dd{h(x)}$
    \begin{Equation}&[12]
        I_{D1}\dx=e\mu_nN_dW\qty[a-h(x)]\frac{eN_ah(x)\dd{h(x)}}{\epsilon_s}
    \end{Equation}
    整理得到
    \begin{Equation}&[13]
        I_{D1}\dx=\frac{e^2N_d^2\mu_nW}{\epsilon_s}\qty[ah(x)\dd{h(x)}-h(x)^2\dd{h(x)}]
    \end{Equation}
    两端积分,左侧对$x$积分,右侧对$h$积分 
    \begin{Equation}&[14]
        \Int[0][L]I_{D1}\dx=\frac{e^2N_d^2\mu_nW}{\epsilon_s}\qty[\Int[h_1][h_2]ah(x)\dd{h(x)}-\Int[h_1][h_2]h(x)^2\dd{h(x)}]
    \end{Equation}
    容易得到
    \begin{Equation}&[15]
        I_{D1}L=\frac{e^2N_d^2\mu_nW}{\epsilon_s}\qty[\frac{1}{2}a(h_2^2-h_1^2)-\frac{1}{3}\qty(h_2^3-h_1^3)]
    \end{Equation}
    这里$L$是沟道长度。接下来要做的就是代入$h_1$和$h_2$,但在此之前先对它们做一些化简。

    根据\fancyref{fml:JFET的源侧耗尽区厚度}
    \begin{Equation}&[16]
        h_1=
        \qty[\frac{2\epsilon_s(V_{bi}-V_{GS})}{eN_d}]^{1/2}
    \end{Equation}
    根据\fancyref{fml:JFET的漏侧耗尽区厚度}
    \begin{Equation}&[17]
        h_2=
        \qty[\frac{2\epsilon_s(V_{bi}+V_{DS}-V_{GS})}{eN_d}]^{1/2}
    \end{Equation}
    根据\fancyref{fml:内建夹断电压}
    \begin{Equation}&[18]
        V_{p0}=\frac{ea^2N_d}{2\epsilon_s}\qquad \frac{1}{V_{p0}}=\frac{2\epsilon_s}{ea^2N_d}
    \end{Equation}
    将\xrefpeq{18}代入\xrefpeq{16}
    \begin{Equation}&[19]
        h_1=
        \qty[\frac{a^2(V_{bi}-V_{GS})}{V_{p0}}]^{1/2}
    \end{Equation}
    将\xrefpeq{18}代入\xrefpeq{17}
    \begin{Equation}&[19]
        h_2=
        \qty[\frac{a^2(V_{bi}+V_{DS}-V_{GS})}{V_{p0}}]^{1/2}
    \end{Equation}
    \xrefpeq{15}右端方括号的第一项可以写为
    \begin{Equation}&[20]
        \frac{1}{2}a\qty(h_2^2-h_1^2)=
        \frac{a}{2}\qty\Bigg{\frac{a^2\qty(V_{bi}+V_{DS}-V_{GS})}{V_{p0}}-\frac{a^2\qty(V_{bi}-V_{GS})}{V_{p0}}}
    \end{Equation}
    即
    \begin{Equation}&[21]
        \frac{1}{2}a\qty(h_2^2-h_1^2)=
        \frac{a^3}{2}\qty\Bigg{\frac{\qty(V_{bi}+V_{DS}-V_{GS})}{V_{p0}}-\frac{\qty(V_{bi}-V_{GS})}{V_{p0}}}
    \end{Equation}
    或
    \begin{Equation}&[22]
        \frac{1}{2}a\qty(h_2^2-h_1^2)=
        \frac{a^3}{2}\frac{V_{DS}}{V_{p0}}
    \end{Equation}\goodbreak
    \xrefpeq{15}右端方括号的第二项可以写为
    \begin{Equation}&[23]
        \qquad\qquad
        \frac{1}{3}\qty(h_2^2-h_1^2)=
        \frac{1}{3}\qty\Bigg{\qty[\frac{a^2\qty(V_{bi}+V_{DS}-V_{GS})}{V_{p0}}]^{3/2}-\qty[\frac{a^2\qty(V_{bi}-V_{GS})}{V_{p0}}]^{3/2}}
        \qquad\qquad
    \end{Equation}
    即
    \begin{Equation}&[23]
        \qquad\qquad\qquad
        \frac{1}{3}\qty(h_2^2-h_1^2)=
        \frac{a^3}{3}\qty\Bigg{\qty[\frac{\qty(V_{bi}+V_{DS}-V_{GS})}{V_{p0}}]^{3/2}-\qty[\frac{\qty(V_{bi}-V_{GS})}{V_{p0}}]^{3/2}}
        \qquad\qquad\qquad
    \end{Equation}
    现在,让我们将\xrefpeq{21}和\xrefpeq{23}
    \begin{Equation}&[24]
        \qquad
        I_{D1}=\frac{e^2N_d^2\mu_nW}{\epsilon_sL}\qty[\frac{a^3}{2}\frac{V_{DS}}{V_{p0}}-\frac{a^3}{3}\qty(\frac{V_{bi}+V_{DS}-V_{GS}}{V_{p0}})^{3/2}+\frac{a^3}{3}\qty(\frac{V_{bi}-V_{GS}}{V_{p0}})^{3/2}]
        \qquad
    \end{Equation}
    提出一个$3a^3$
    \begin{Equation}&[25]
        \qquad
        I_{D1}=\frac{e^2N_d^2\mu_nWa^3}{6\epsilon_sL}\qty[3\qty(\frac{V_{DS}}{V_{p_0}})-2\qty(\frac{V_{bi}+V_{DS}-V_{GS}}{V_{p0}})^{3/2}+2\qty(\frac{V_{bi}-V_{GS}}{V_{p0}})^{3/2}]
        \qquad
    \end{Equation}
    若引入$I_{P1}$代换
    \begin{Equation}&[26]
        I_{P1}=\frac{e^2N_d^2\mu_nWa^3}{6\epsilon_sL}  
    \end{Equation}
    即得
    \begin{Equation}
        \qquad\qquad
        I_{D1}=I_{P1}\qty[3\qty(\frac{V_{DS}}{V_{p_0}})-2\qty(\frac{V_{bi}+V_{DS}-V_{GS}}{V_{p0}})^{3/2}+2\qty(\frac{V_{bi}-V_{GS}}{V_{p0}})^{3/2}]
        \qquad\qquad
    \end{Equation}
    以上推导是在线性区进行的,当进入饱和区时,电流不再随$V_{DS}$变化,令$V_{DS}=V_{GS}-V_{p}$
    \begin{Equation}
        \qquad\qquad\qquad
        I_{D1}=I_{P1}\qty[3\qty(\frac{V_{GS}-V_p}{V_{p0}})-2\qty(\frac{V_{bi}-V_p}{V_{p0}})^{3/2}+2\qty(\frac{V_{bi}-V_{GS}}{V_{p0}})^{3/2}]
        \qquad\qquad\qquad
    \end{Equation}
    运用$V_p=V_{bi}-V_{p0}$和$V_{p0}=V_{bi}-V_{p}$
    \begin{Equation}
        \qquad\qquad
        I_{D1}=I_{P1}\qty[3\qty(\frac{V_{GS}-V_{bi}+V_{p0}}{V_{p0}})-2\qty(\frac{V_{p0}}{V_{p0}})^{3/2}+2\qty(\frac{V_{bi}-V_{GS}}{V_{p0}})^{3/2}]
        \qquad\qquad
    \end{Equation}
    整理得到
    \begin{Equation}
        I_{D1}=I_{P1}\qty[1-3\qty(\frac{V_{bi}-V_{GS}}{V_{p0}})+2\qty(\frac{V_{bi}-V_{GS}}{V_{p0}})^{3/2}]
    \end{Equation}
    至此,我们就完成了所有的推导。
\end{Proof}

我们将\fancyref{fml:JFET的电流--电压关系}中的结论展现在\xref{fig:JFET的特性方程}中。尽管JFET的电流电压关系相较MOSFET复杂了很多,但从特性曲面和曲线上看,两者却是惊人般的相似!
\begin{Figure}[JFET的特性方程]
    \begin{FigureSub}[JFET的特性曲面]
        \includegraphics{build/Chapter05A_01a.fig.pdf}
    \end{FigureSub}\\ \vspace{0.5cm}
    \begin{FigureSub}[JFET的转移特性]
        \includegraphics[scale=0.8]{build/Chapter05A_01b.fig.pdf}
    \end{FigureSub}
    \hspace{0.25cm}
    \begin{FigureSub}[JFET的输出特性]
        \includegraphics[scale=0.8]{build/Chapter05A_01c.fig.pdf}
    \end{FigureSub}
\end{Figure}
关于JFET,我们常常引入下面的简化公式,以更方便的计算饱和区电流。

\begin{BoxFormula}[JFET的饱和电流公式]
    当JFET处于饱和区时,可以应用以下简化公式
    \begin{Equation}
        I_{D1}=I_{DSS1}\qty(1-\frac{V_{GS}}{V_p})^2
    \end{Equation}
    其中,$I_{DSS1}$是$V_{GS}=0$时的饱和电流$I_{D1}$
    \begin{Equation}
        I_{DSS1}=I_{P1}\qty[1-3\qty(\frac{V_{bi}}{V_{p0}})+2\qty(\frac{V_{bi}}{V_{p0}})^{3/2}]
    \end{Equation}
\end{BoxFormula}

注意!该公式近适用于JFET的饱和区,在线性区没有什么近似公式。\goodbreak

根据\fancyref{fml:JFET的饱和电流公式},近似的$I_{D1}$的表达式为
\begin{Equation}
    I_{D}=I_{P1}\qty[1-3\qty(\frac{V_{bi}}{V_{p0}})+2\qty(\frac{V_{bi}}{V_{p0}})^{3/2}]\qty[1-\frac{V_{GS}}{V_p}]^2
\end{Equation}\nopagebreak
根据\fancyref{fml:JFET的电流--电压关系},精确的$I_{D1}$的表达式为
\begin{Equation}
    I_{D1}=I_{P1}\qty[1-3\qty(\frac{V_{bi}-V_{GS}}{V_{p0}})+2\qty(\frac{V_{bi}-V_{GS}}{V_{p0}})^{3/2}]
\end{Equation}
由此可见,两者并没有特别明显的联系。因此这里的近似并未作推导,更多的应当视为一种经验性的结论。但从\xref{fig:JFET的输出特性}可以看出,该近似式与精确式的贴合的相当的好!只是略大些。