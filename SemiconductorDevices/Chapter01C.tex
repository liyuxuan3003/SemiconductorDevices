\section{反偏PN结}

若在P区和N区间施加电势,PN结将不再保持热平衡状态,换言之,不再具有统一的费米能级。\xref{fig:反偏PN结的能带图}展示了PN结在外加反偏电压$V_R$时的能带图,P区费米能级将比N区高$eV_R$。

反偏时,总电势差$V_\te{total}$将增加,变为\setpeq{Vtotal}
\begin{Equation}&[1]
    V_\te{total}=|\phi_{Fn}|+|\phi_{Fp}|+V_R
\end{Equation}
其中$V_R$是反偏电压。\xrefpeq{1}也可以被写为
\begin{Equation}
    V_\te{total}=V_{bi}+V_R
\end{Equation}
其中$V_{bi}$是先前定义过的热平衡状态下的内建电势差。

\begin{Figure}[反偏PN结的能带图]
    \includegraphics[scale=0.8]{build/Chapter01C_01.fig.pdf}
\end{Figure}

\subsection{反偏时的空间电荷区宽度}
\xref{fig:反偏PN结}展示了加反向电压$V_R$时的PN结的结构图,以及其空间电荷区、内建电场$\E$、外加电场$\E_\te{app}$。由于电中性的P区和N区内的电场强度为零(即便不是零,也应该为一个可以忽略的很小的值),这就意味着空间电荷区内的电场要比热平衡时增强,这个电场始于正电荷终于负电荷,也就是说,随着电场的增强,正负电荷的数量也要随之增加。在给定的杂质掺杂浓度条件下,耗尽区内的正负电荷想要增加,空间电荷区的宽度$W$就必须增大。因此可得出一个重要结论:\empx{空间电荷区随着外加反偏电压的增加而展宽}。那么$W$具体如何随$V_R$变化呢?

\begin{Figure}[反偏PN结]
    \includegraphics[scale=0.8]{build/Chapter01C_02.fig.pdf}
\end{Figure}

实际上,这只需要将前述所有公式的$V_{bi}$换成$V_{bi}+V_R$即可,故$W$可被重写为

\begin{BoxFormula}[反偏时的空间电荷区宽度]
    在反偏电压$V_R$下的空间电荷区宽度$W$为
    \begin{Equation}
        W=\qty[\frac{2\epsilon_s(V_{bi}+V_R)}{e}\qty(\frac{N_a+N_d}{N_aN_d})]^{1/2}
    \end{Equation}
    其中,$x_n$和$x_p$等于
    \begin{Gather}[10pt]
        x_n=\qty[\frac{2\epsilon_s(V_{bi}+V_R)}{e}\qty(\frac{N_a}{N_d})\qty(\frac{1}{N_a+N_d})]^{1/2}\\
        x_p=\qty[\frac{2\epsilon_s(V_{bi}+V_R)}{e}\qty(\frac{N_d}{N_a})\qty(\frac{1}{N_a+N_d})]^{1/2}
    \end{Gather}
\end{BoxFormula}

\subsection{势垒电容}
由于空间电荷区中正负电荷是分离的,所以PN结就具有了电容的充放电效应,这就是\uwave{势垒电容}(Junction Capacitance)。反偏电压增量$\dd{V_R}$会在N区和P区产生额外的正负电荷$\dd{Q}$。

\begin{BoxFormula}[二极管的势垒电容]
    二极管的势垒电容可以被表示为
    \begin{Equation}
        C_j=A\qty[\frac{e\epsilon_sN_aN_d}{2(V_{bi}+V_R)(N_a+N_d)}]^{1/2}
    \end{Equation}
\end{BoxFormula}

\begin{Proof}
    势垒电容是微分电容,因此
    \begin{Equation}&[1]
        C_j=\dv{Q}{V_R}
    \end{Equation}
    这里$\dd{Q}$应当为(其中$A$是PN结的截面积)
    \begin{Equation}
        \dd{Q}=eAN_d\dd{x_n}=eAN_a\dd{x_p}
    \end{Equation}
    将上式给出的$\dd{Q}=eAN_d\dd{x_n}$代入\xrefpeq{1}
    \begin{Equation}&[2]
        C_j=eAN_d\dv{x_n}{V_R}
    \end{Equation}
    将\xref{fml:反偏时的空间电荷区宽度}给出的$x_n$代入\xrefpeq{2}
    \begin{Equation}&[3]
        C_j=eAN_d\dv{V_R}\qty[\frac{2\epsilon_s(V_{bi}+V_R)}{e}\qty(\frac{N_a}{N_d})\qty(\frac{1}{N_a+N_d})]^{1/2}
    \end{Equation}
    即
    \begin{Equation}&[4]
        C_j=eAN_d\qty[\frac{\epsilon_s}{2e(V_{bi}+V_R)}\qty(\frac{N_a}{N_d})\qty(\frac{1}{N_a+N_d})]^{1/2}
    \end{Equation}
    或
    \begin{Equation}&[5]
        C_j=A\qty[\frac{e\epsilon_s}{2(V_{bi}+V_R)(N_a+N_d)}]^{1/2}
    \end{Equation}
    若代入$\dd{Q'}=eN_a\dd{x_p}$而不是$\dd{Q'}=eN_d\dd{x_n}$,我们会得到完全相同的结果
\end{Proof}

若比较\xref{fml:二极管的势垒电容}和\xref{fml:反偏时的空间电荷区宽度},我们发现
\begin{Equation}
    C_j=\frac{A\epsilon_s}{W}
\end{Equation}
即与平行板电容器完全相同的电容表达式。