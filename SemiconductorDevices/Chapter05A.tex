\section{JFET的基本结构}
场效应管的基础,即场效应现象早在1930年左右就已经被发现,但由于那时还没有良好的半导体材料和先进的半导体加工工艺,所以知道1950年左右,场效应管这种器件才被重新研究。

场效应现象的定义是,\empx{半导体的电导被垂直于半导体表面的电场调制}。

场效应管工作时只存在一种载流子,即多数载流子,这与\xref{chap:双极结型晶体管}的BJT即\uwave{双极晶体管}是不同的。BJT中多子少子同时导电,FET中仅多子导电,故场效应管有时也被称为\uwave{单极晶体管}。

\subsection{JFET的结构}
JFET的结构如\xref{fig:JFET的结构简图}所示(以下JFET均指PN JFET),同样是三个掺杂区和两个PN结。

JFET的三个区域依次为:\uwave{源极}(Source, S)、\uwave{栅极}(Gate, G)、\uwave{漏极}(Drain, D)。这样的电极命名场效应管通用的,对于JFET和MOSFET都是如此。但是JFET非常不同于MOSFET的一点是,JFET的源和漏是直通的!例如在\xref{fig:N沟道JFET的结构简图}我们看到,源和漏之间直接由N型半导体相连,换言之,源和漏间的沟道原本就是导通的(“生而导通”),并不需要通过栅压开启。

\begin{Figure}[JFET的结构简图]
    \begin{FigureSub}[N沟道JFET的结构简图]
        \includegraphics[scale=0.87]{build/Chapter05A_02.fig.pdf}
    \end{FigureSub}
    \hspace{0.2cm}
    \begin{FigureSub}[P沟道JFET的结构简图]
        \includegraphics[scale=0.87]{build/Chapter05A_03.fig.pdf}
    \end{FigureSub}
\end{Figure}

JFET的结构中,栅包含了上下两部分,实际工艺中两者是不对称的,如\xref{fig:JFET的实际结构},一侧的栅和源和漏一样作为表面掺杂,一侧的栅则直接由衬底充当。源漏极附近重掺是为了形成欧姆接触。
\begin{Figure}[JFET的实际结构]
    \begin{FigureSub}[N沟道JFET的实际结构]
        \includegraphics[scale=0.87]{build/Chapter05A_04.fig.pdf}
    \end{FigureSub}
    \hspace{0.2cm}
    \begin{FigureSub}[P沟道JFET的实际结构]
        \includegraphics[scale=0.87]{build/Chapter05A_05.fig.pdf}
    \end{FigureSub}
\end{Figure}

JFET的电路符号如\xref{fig:JFET的电路符号}所示,箭头代表了栅--沟道PN结的方向
\begin{Figure}[JFET的电路符号]
    \begin{FigureSub}[N沟道JFET的电路符号]
        \includegraphics{build/Chapter05A_12.fig.pdf}
    \end{FigureSub}
    \hspace{1cm}
    \begin{FigureSub}[P沟道JFET的电路符号]
        \includegraphics{build/Chapter05A_13.fig.pdf}
    \end{FigureSub}
\end{Figure}
JFET的电路符号是单线的,作为对比,MOSFET是双线的(多的一根线代表栅极氧化层)。

\subsection{JFET的原理}
正如\xref{subsec:JFET的结构}中提到,JFET的源和漏是直通的,并不需要栅来开启,那么,JFET中的栅扮演了什么样一种角色?事实是,JFET的沟道可以通过负栅压关闭!这是怎么实现的呢?

\begin{Tablex}[JFET的原理]{|c|c|}
<栅源电压$V_{GS}$的影响&漏源电压$V_{DS}$的影响\\>
    \xcell<Y>[2ex][-1ex]{\includegraphics[width=6.7cm]{build/Chapter05A_07.fig.pdf}}&
    \xcell<Y>[2ex][-1ex]{\includegraphics[width=6.7cm]{build/Chapter05A_07.fig.pdf}}\\
    \xgp[2ex][2ex]{$V_{GS}$为零}&
    \xgp[2ex][2ex]{$V_{DS}$为零}\\ \hlinelig
    \xcell<Y>[2ex][-1ex]{\includegraphics[width=6.7cm]{build/Chapter05A_08.fig.pdf}}&
    \xcell<Y>[2ex][-1ex]{\includegraphics[width=6.7cm]{build/Chapter05A_10.fig.pdf}}\\
    \xgp[2ex][2ex]{$V_{GS}$为较小的负电压$V_1^{-}$}&
    \xgp[2ex][2ex]{$V_{DS}$为较小的正电压$V_1^{+}$}\\ \hlinelig
    \xcell<Y>[2ex][-1ex]{\includegraphics[width=6.7cm]{build/Chapter05A_09.fig.pdf}}&
    \xcell<Y>[2ex][-1ex]{\includegraphics[width=6.7cm]{build/Chapter05A_11.fig.pdf}}\\
    \xgp[2ex][2ex]{$V_{GS}$为较大的负电压$V_2^{-}$}&
    \xgp[2ex][2ex]{$V_{DS}$为较大的正电压$V_2^{+}$}\\ \hlinelig
\end{Tablex}

\subsubsection{栅源电压$V_{GS}$的影响}
JFET的栅极是P型,而源漏和源漏间的沟道为N型(对于N沟道JFET),这就意味着栅和沟道间形成了一个PN结,而有PN结就意味着有耗尽区。当栅压为零时,耗尽区很薄,对沟道几乎没有什么影响。然而,当栅压的负值越来越大,栅--沟道PN结的反偏程度将会越来越大,根据\xref{chap:PN结的静电特性}的内容,我们知道,反偏电压越高,耗尽区越厚,这会挤压沟道的宽度,沟道越来越窄,最终,沟道夹断,不再导通。在该过程中,沟道的电阻会逐渐增大至无穷大。

JFET在这一点上和MOSFET的差异是比较大的,两者栅压的作用相反
\begin{itemize}
    \item JFET的沟道原本导通,负栅压关闭沟道,是\uwave{耗尽型}(Depletion Mode)器件。
    \item MOSFET的沟道原本不导通,正栅压开启沟道,是\uwave{增强型}(Enhancement Mode)器件。
\end{itemize}
当然,JFET为耗尽型,MOSFET为增强型,这也只是最典型的情况,两者都可以通过特殊的工艺方法实现耗尽型和增强型(例如MOSFET可以向栅氧掺正离子使其变为耗尽型)。

\subsubsection{漏源电压$V_{DS}$的影响}
JFET的栅--沟道PN结的反偏情况,不仅会受到$V_{GS}$的影响,也会受到$V_{DS}$的影响,这是因为,由于$V_{DS}$的电压降是在漏和源间,沟道从漏至源电压逐渐降低至零,沟道作为PN结的负端,其上的正电压意味着PN结的反偏。因此,如\xref{tab:JFET的原理}所示,当有$V_{DS}$存在时,耗尽区会发生倾斜,漏侧耗尽区最厚,源侧耗尽区最薄。因此随着$V_{DS}$的增加,漏电流$I_{DS}$增加会越来越慢,直到$V_{DS}$增大至漏测耗尽区相互接触将沟道在漏极处夹断,但这种漏源电压导致的夹断不同于栅源电压导致的夹断,并不会使电流为零,相反,漏电流$I_{DS}$将转为恒流。可以这么理解,夹断后,进一步增加的$V_{DS}$完全降落在随之增加的夹断耗尽区上,故电流不变。

再次指出,尽管两种情况都被称为“夹断”,但结果完全不同的
\begin{itemize}
    \item 由栅源电压$V_{GS}$导致的夹断,将会使得漏电流$I_D$为零。
    \item 由漏源电压$V_{DS}$导致的夹断,将会使得漏电流$I_D$转为恒流。
\end{itemize}

\subsection{MESFET的结构}
MESFET的结构和PN JFET是非常相似的,如\xref{fig:MESFET的实际结构}所示,区别仅在于,栅极由P型区域变为了金属,栅--沟道结由PN结变为了金半结,同时,器件由“双边”变为了“单边”,这就是说MESFET中栅仅出现在器件的一侧,而另一侧是电导率几乎为零的本征材料。而更有意思的是,MESFET和PN JFET间的差别比\xref{fig:MESFET的实际结构}和\xref{fig:JFET的实际结构}展现的还要小,因为金属电极其实并不只是存在于MESFET中,即便在PN JFET中,栅、源、漏上也存在金属,只不过由于相应半导体区域都是重掺,形成的是欧姆接触,省略不绘了。故MESFET相较PN JFET的唯一区别就是取消了栅极的$p^{+}$掺杂,此时原先栅的金属电极就自然与沟道轻掺的$n$区域形成了整流接触。MESFET的原理和PN JFET也基本是一致的,同样是以负栅压关闭沟道。
\begin{Figure}[MESFET的实际结构]
    \includegraphics[scale=0.87]{build/Chapter05A_06.fig.pdf}
\end{Figure}
% 我们后面的讨论主要以PN JFET为主,相关的结论和公式基本可以直接用于MESFET。