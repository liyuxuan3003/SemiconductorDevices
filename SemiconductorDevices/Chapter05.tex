\chapter{结型场效应晶体管}
在本章中,我们将
\begin{itemize}
    \item 给出pn JFET和MESFET器件的几何形状并讨论它们的基本工作方式。
    \item 依据器件的半导体材料和几何属性推导JFET的理想电流--电压特性。
    \item 探讨JFET的晶体管增益或跨导。
    \item 讨论JFET的一些非理想效应,包括沟道长度调制和速度饱和效应。
    \item 开发一个用于将器件中小信号电流和电压关联起来的小信号等效JFET电路。
    \item 检查影响JFET频率响应和限制的各种物理因素,并推导截止频率的表达式。
    \item 给出称为HEMT的专用JFET的几何形状和特性。
\end{itemize}
\uwave{结型场效应管}(Junction Field Effect Transistor, JFET)是一种类型的场效应晶体管,而顾名思义,其工作原理基于电场效应,其结构则是基于结。而取决于构成JFET的结是PN结还是金半结,JFET可以进一步分为PN JFET和MESFET,这里MESFET是指\uwave{金属--半导体场效应晶体管}(Metal Semiconductor Field Effect Transistor, MESFET),应注意的是,尽管MESFET和MOSFET的名称很相似,但毫无关系,且MESFET是属于JFET的一种。

\section{JFET的基本结构}
场效应管的基础,即场效应现象早在1930年左右就已经被发现,但由于那时还没有良好的半导体材料和先进的半导体加工工艺,所以知道1950年左右,场效应管这种器件才被重新研究。

场效应现象的定义是,\empx{半导体的电导被垂直于半导体表面的电场调制}。

场效应管工作时只存在一种载流子,即多数载流子,这与\xref{chap:双极结型晶体管}的BJT即\uwave{双极晶体管}是不同的。BJT中多子少子同时导电,FET中仅多子导电,故场效应管有时也被称为\uwave{单极晶体管}。

\subsection{JFET的结构}
JFET的结构如\xref{fig:JFET的结构简图}所示(以下JFET均指PN JFET),同样是三个掺杂区和两个PN结。

JFET的三个区域依次为:\uwave{源极}(Source, S)、\uwave{栅极}(Gate, G)、\uwave{漏极}(Drain, D)。这样的电极命名场效应管通用的,对于JFET和MOSFET都是如此。但是JFET非常不同于MOSFET的一点是,JFET的源和漏是直通的!例如在\xref{fig:N沟道JFET的结构简图}我们看到,源和漏之间直接由N型半导体相连,换言之,源和漏间的沟道原本就是导通的(“生而导通”),并不需要通过栅压开启。

\begin{Figure}[JFET的结构简图]
    \begin{FigureSub}[N沟道JFET的结构简图]
        \includegraphics[scale=0.87]{build/Chapter05A_02.fig.pdf}
    \end{FigureSub}
    \hspace{0.2cm}
    \begin{FigureSub}[P沟道JFET的结构简图]
        \includegraphics[scale=0.87]{build/Chapter05A_03.fig.pdf}
    \end{FigureSub}
\end{Figure}

JFET的结构中,栅包含了上下两部分,实际工艺中两者是不对称的,如\xref{fig:JFET的实际结构},一侧的栅和源和漏一样作为表面掺杂,一侧的栅则直接由衬底充当。源漏极附近重掺是为了形成欧姆接触。
\begin{Figure}[JFET的实际结构]
    \begin{FigureSub}[N沟道JFET的实际结构]
        \includegraphics[scale=0.87]{build/Chapter05A_04.fig.pdf}
    \end{FigureSub}
    \hspace{0.2cm}
    \begin{FigureSub}[P沟道JFET的实际结构]
        \includegraphics[scale=0.87]{build/Chapter05A_05.fig.pdf}
    \end{FigureSub}
\end{Figure}

JFET的电路符号如\xref{fig:JFET的电路符号}所示,箭头代表了栅--沟道PN结的方向
\begin{Figure}[JFET的电路符号]
    \begin{FigureSub}[N沟道JFET的电路符号]
        \includegraphics{build/Chapter05A_12.fig.pdf}
    \end{FigureSub}
    \hspace{1cm}
    \begin{FigureSub}[P沟道JFET的电路符号]
        \includegraphics{build/Chapter05A_13.fig.pdf}
    \end{FigureSub}
\end{Figure}
JFET的电路符号是单线的,作为对比,MOSFET是双线的(多的一根线代表栅极氧化层)。

\subsection{JFET的原理}
正如\xref{subsec:JFET的结构}中提到,JFET的源和漏是直通的,并不需要栅来开启,那么,JFET中的栅扮演了什么样一种角色?事实是,JFET的沟道可以通过负栅压关闭!这是怎么实现的呢?

\begin{Tablex}[JFET的原理]{|c|c|}
<栅源电压$V_{GS}$的影响&漏源电压$V_{DS}$的影响\\>
    \xcell<Y>[2ex][-1ex]{\includegraphics[width=6.7cm]{build/Chapter05A_07.fig.pdf}}&
    \xcell<Y>[2ex][-1ex]{\includegraphics[width=6.7cm]{build/Chapter05A_07.fig.pdf}}\\
    \xgp[2ex][2ex]{$V_{GS}$为零}&
    \xgp[2ex][2ex]{$V_{DS}$为零}\\ \hlinelig
    \xcell<Y>[2ex][-1ex]{\includegraphics[width=6.7cm]{build/Chapter05A_08.fig.pdf}}&
    \xcell<Y>[2ex][-1ex]{\includegraphics[width=6.7cm]{build/Chapter05A_10.fig.pdf}}\\
    \xgp[2ex][2ex]{$V_{GS}$为较小的负电压$V_1^{-}$}&
    \xgp[2ex][2ex]{$V_{DS}$为较小的正电压$V_1^{+}$}\\ \hlinelig
    \xcell<Y>[2ex][-1ex]{\includegraphics[width=6.7cm]{build/Chapter05A_09.fig.pdf}}&
    \xcell<Y>[2ex][-1ex]{\includegraphics[width=6.7cm]{build/Chapter05A_11.fig.pdf}}\\
    \xgp[2ex][2ex]{$V_{GS}$为较大的负电压$V_2^{-}$}&
    \xgp[2ex][2ex]{$V_{DS}$为较大的正电压$V_2^{+}$}\\ \hlinelig
\end{Tablex}

\subsubsection{栅源电压$V_{GS}$的影响}
JFET的栅极是P型,而源漏和源漏间的沟道为N型(对于N沟道JFET),这就意味着栅和沟道间形成了一个PN结,而有PN结就意味着有耗尽区。当栅压为零时,耗尽区很薄,对沟道几乎没有什么影响。然而,当栅压的负值越来越大,栅--沟道PN结的反偏程度将会越来越大,根据\xref{chap:PN结的静电特性}的内容,我们知道,反偏电压越高,耗尽区越厚,这会挤压沟道的宽度,沟道越来越窄,最终,沟道夹断,不再导通。在该过程中,沟道的电阻会逐渐增大至无穷大。

JFET在这一点上和MOSFET的差异是比较大的,两者栅压的作用相反
\begin{itemize}
    \item JFET的沟道原本导通,负栅压关闭沟道,是\uwave{耗尽型}(Depletion Mode)器件。
    \item MOSFET的沟道原本不导通,正栅压开启沟道,是\uwave{增强型}(Enhancement Mode)器件。
\end{itemize}
当然,JFET为耗尽型,MOSFET为增强型,这也只是最典型的情况,两者都可以通过特殊的工艺方法实现耗尽型和增强型(例如MOSFET可以向栅氧掺正离子使其变为耗尽型)。

\subsubsection{漏源电压$V_{DS}$的影响}
JFET的栅--沟道PN结的反偏情况,不仅会受到$V_{GS}$的影响,也会受到$V_{DS}$的影响,这是因为,由于$V_{DS}$的电压降是在漏和源间,沟道从漏至源电压逐渐降低至零,沟道作为PN结的负端,其上的正电压意味着PN结的反偏。因此,如\xref{tab:JFET的原理}所示,当有$V_{DS}$存在时,耗尽区会发生倾斜,漏侧耗尽区最厚,源侧耗尽区最薄。因此随着$V_{DS}$的增加,漏电流$I_{DS}$增加会越来越慢,直到$V_{DS}$增大至漏测耗尽区相互接触将沟道在漏极处夹断,但这种漏源电压导致的夹断不同于栅源电压导致的夹断,并不会使电流为零,相反,漏电流$I_{DS}$将转为恒流。可以这么理解,夹断后,进一步增加的$V_{DS}$完全降落在随之增加的夹断耗尽区上,故电流不变。

再次指出,尽管两种情况都被称为“夹断”,但结果完全不同的
\begin{itemize}
    \item 由栅源电压$V_{GS}$导致的夹断,将会使得漏电流$I_D$为零。
    \item 由漏源电压$V_{DS}$导致的夹断,将会使得漏电流$I_D$转为恒流。
\end{itemize}

\subsection{MESFET的结构}
MESFET的结构和PN JFET是非常相似的,如\xref{fig:MESFET的实际结构}所示,区别仅在于,栅极由P型区域变为了金属,栅--沟道结由PN结变为了金半结,同时,器件由“双边”变为了“单边”,这就是说MESFET中栅仅出现在器件的一侧,而另一侧是电导率几乎为零的本征材料。而更有意思的是,MESFET和PN JFET间的差别比\xref{fig:MESFET的实际结构}和\xref{fig:JFET的实际结构}展现的还要小,因为金属电极其实并不只是存在于MESFET中,即便在PN JFET中,栅、源、漏上也存在金属,只不过由于相应半导体区域都是重掺,形成的是欧姆接触,省略不绘了。故MESFET相较PN JFET的唯一区别就是取消了栅极的$p^{+}$掺杂,此时原先栅的金属电极就自然与沟道轻掺的$n$区域形成了整流接触。MESFET的原理和PN JFET也基本是一致的,同样是以负栅压关闭沟道。
\begin{Figure}[MESFET的实际结构]
    \includegraphics[scale=0.87]{build/Chapter05A_06.fig.pdf}
\end{Figure}

值得一提的是,通常而言MESFET并不是\xce{Si}基的,而是由\xce{GaAs}制成的。

% 我们后面的讨论主要以PN JFET为主,相关的结论和公式基本可以直接用于MESFET。
\section{JFET的器件特性}
现在让我们来推导JFET的器件特性。我们或许已经注意到了JFET是上下对称的,即所谓的“双边JFET”。我们可以只考虑半个JFET来让计算更简单,即“单边HFET”,如\xref{fig:JFET的}

\begin{Figure}[JFET的简化]
    \begin{FigureSub}[双边JFET]
        \includegraphics[scale=0.85]{build/Chapter05B_01.fig.pdf}
    \end{FigureSub}\\ \vspace{0.5cm}
    \begin{FigureSub}[单边JFET]
        \includegraphics[scale=0.85]{build/Chapter05B_02.fig.pdf}
    \end{FigureSub}
\end{Figure}
以$I_{D2}$和$I_{D1}$分别表示双边JFET和单边JFET上的电流,显然
\begin{Equation}
    I_{D2}=2I_{D1}
\end{Equation}
这是因为双边JFET可以视为两个单边JFET的并联。后面的分析都对单边JFET进行。

\subsection{栅源夹断电压}
这一小节的目的是,推导$V_{GS}$为何值时,JFET沟道被夹断?
\begin{Figure}[栅源夹断电压的推导]
    \includegraphics{build/Chapter05B_03.fig.pdf}
\end{Figure}
如\xref{fig:栅源夹断电压的推导}所示,设JFET的沟道的总厚度为$a$,其中,设耗尽区占据的厚度为$h$。\setpeq{栅源夹断电压}

根据\fancyref{fml:反偏时的空间电荷区宽度},考虑到$V_{GS}$形式上为正偏电压,且$N_{a}\gg N_d$
\begin{Equation}&[1]
    h=\qty[\frac{2\epsilon_s(V_{bi}-V_{GS})}{eN_d}]^{1/2}
\end{Equation}
其中$V_{bi}$是栅--沟道PN结的内建电势差。

这里$a$是一个常量,当$h=a$时,就意味着耗尽区完全夹断了沟道,记此时的$V_{GS}$为$V_p$
\begin{Equation}&[2]
    a=\qty[\frac{2\epsilon_s(V_{bi}-V_{p})}{eN_d}]^{1/2}
\end{Equation}
这里还常引入$V_{p0}=V_{bi}-V_p$代换
\begin{Equation}&[3]
    a=\qty[\frac{2\epsilon_sV_{p0}}{eN_d}]^{1/2}
\end{Equation}
正式定义如下
\begin{BoxDefinition}[内建夹断电压]
    内建夹断电压$V_{p0}$定义为
    \begin{Equation}
        V_{p0}=V_{bi}-V_p
    \end{Equation}
    这里,$V_{bi}$是内建电压,$V_{p}$是夹断电压,$V_{p0}$是内建夹断电压。\footnote[2]{其中p代表的是夹断的英文pinch off。}
\end{BoxDefinition}

关于$V_p$和$V_{p0}$,我们可能会觉得很绕,可以通过以下方式记忆
\begin{itemize}
    \item $V_p$是夹断电压,它是一个负值,它的意义是当栅压达到$V_{GS}=V_p$的负值时,沟道夹断。
    \item $V_{p0}$是内建夹断电压,它和内建电压$V_{bi}$一样都是一个正值,且$V_{bi}<V_{p0}$,它的意义可以解释为,假如栅--沟道PN结的内建电压能到达$V_{p0}$,那么,在零栅压时耗尽区就已经挤占了整个沟道,发生夹断了。换言之,$V_{p0}=V_{bi}-V_{p}$代表夹断发生时的耗尽区电压。
\end{itemize}\setpeq{栅源夹断电压}
$V_{p0}$关于$V_p$的表达式为
\begin{Equation}
    V_{p0}=V_{bi}-V_p
\end{Equation}
$V_{p}$关于$V_{p0}$的表达式为
\begin{Equation}
    V_{p}=V_{bi}-V_{p0}
\end{Equation}
这是非常好记忆的:夹断电压$V_{p}$和内建夹断电压$V_{p0}$分别等于$V_{bi}$减去对方。

内建夹断电压$V_{p0}$是一个与总厚度$a$直接相关的量,由\xrefpeq{3}稍微做一些变换,就可以得到
\begin{BoxFormula}[内建夹断电压]
    内建夹断电压$V_{p0}$可以表示为
    \begin{Equation}
        V_{p0}=\frac{ea^2N_d}{2\epsilon_s}
    \end{Equation}
\end{BoxFormula}

现在,我们就可以明确写出栅源电压$V_{GS}$达到多少时,夹断会发生了。
\begin{BoxFormula}[JFET的栅源夹断电压]
    当栅源电压$V_{GS}$满足下式时,沟道夹断
    \begin{Equation}
        V_{GS}\leq V_p
    \end{Equation}
    即
    \begin{Equation}
        V_{GS}\leq V_{bi}-V_{p0}
    \end{Equation}
\end{BoxFormula}

\subsection{漏源饱和电压}
这一小节的目的是,推导$V_{DS}$为何值时,JFET沟道电流饱和?
\begin{Figure}[漏源夹断电压的推导]
    \includegraphics{build/Chapter05B_04.fig.pdf}
\end{Figure}
如\xref{fig:漏源夹断电压的推导}所示,此时耗尽区是倾斜的,宽度$h(x)$并不均等,源侧记为$h_1$,漏侧记为$h_2$。

在源侧,“反偏电压”就是$-V_{GS}$,故$h_1$和前面的$h$是相同的
\begin{BoxFormula}[JFET的源侧耗尽区厚度]
    JFET在源侧的耗尽区厚度$h_1$为
    \begin{Equation}
        h_1=
        \qty[\frac{2\epsilon_s(V_{bi}-V_{GS})}{eN_d}]^{1/2}
    \end{Equation}
\end{BoxFormula}
在漏侧,“反偏电压”增大至$V_{DS}-V_{GS}$,故$h_2$应改写为
\begin{BoxFormula}[JFET的源侧耗尽区厚度]
    JFET在源侧的耗尽区厚度$h_2$为
    \begin{Equation}
        h_2=
        \qty[\frac{2\epsilon_s(V_{bi}+V_{DS}-V_{GS})}{eN_d}]^{1/2}
    \end{Equation}
\end{BoxFormula}\setpeq{漏源夹断电压}
很明显,漏源电压$V_{DS}$导致的那种使电流饱和的夹断的条件是$h_2=a$,即$V_{DS}$要满足
\begin{Equation}
    a=\qty[\frac{2\epsilon_s(V_{bi}+V_{DS}-V_{GS})}{eN_d}]^{1/2}
\end{Equation}
容易解出
\begin{Equation}
    V_{DS}=\frac{ea^2N_{d}}{2\varepsilon_{s}}-V_{bi}+V_{GS}
\end{Equation}
根据\fancyref{fml:内建夹断电压}
\begin{Equation}
    V_{DS}=V_{p0}-V_{bi}+V_{GS}
\end{Equation}
根据\fancyref{def:内建夹断电压}
\begin{Equation}
    V_{DS}=V_{GS}-V_{p}
\end{Equation}
由此,我们就可以归纳出电流饱和时$V_{DS}$的条件了
\begin{BoxFormula}[漏源饱和电压]
    当漏源电压$V_{DS}$满足下式时,沟道电流饱和
    \begin{Equation}
        V_{DS}\geq V_{GS}-V_{p}
    \end{Equation}
    即
    \begin{Equation}
        V_{DS}\geq V_{GS}-V_{bi}+V_{p0}
    \end{Equation}
\end{BoxFormula}