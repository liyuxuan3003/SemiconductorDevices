\section{PN结的基本结构}

\uwave{PN结}(PN Junction)是如何形成的?如\xref{fig:PN结的基本结构}所示,本征半导体中,在一部分区域掺入受主形成P区,在相邻区域掺入施主形成N区,交界面称为\uwave{冶金结}(Metallurgical Junction)。

\begin{Figure}[PN结的基本结构]
    \includegraphics{build/Chapter01A_03.fig.pdf}
\end{Figure}

简单起见,我们将首先考虑\uwave{突变结}(Step Junction)的情况。如\xref{fig:突变结的掺杂浓度曲线},突变结的特点是
\begin{itemize}
    \item 在掺杂区(P区和N区)中,杂质浓度是均匀分布的。
    \item 在掺杂区的交界面处,杂质浓度有一个突然的跃变。
\end{itemize}\goodbreak

最初,在交界面处,对于电子和空穴均存在一个很大的浓度梯度,这就导致
\begin{itemize}
    \item N区多数载流子电子开始向P区扩散。
    \item P区多数载流子空穴开始向N区扩散。
\end{itemize}

\begin{Figure}[突变结的掺杂浓度曲线]
    \includegraphics{build/Chapter01A_01.fig.pdf}
\end{Figure}


但是,这种扩散过程并不能无限持续下去
\begin{itemize}
    \item 随着N区空穴扩散进入P区,N区带正电的受主原子被留下了。
    \item 随着P区电子扩散进入N区,P区带负电的施主原子被留下了。
\end{itemize}

而N区的净正电荷和P区中净负电荷,将在交界面附近激发一个电场,电场的方向是从正电荷指向负电荷,换言之,从N区指向P区。PN结中的净正电荷区域和净负电荷区域,如\xref{fig:PN结的空间电荷区}所示,我们将这两个带电区域称为\uwave{空间电荷区}(Space Charge Region)。最重要的的是,所有的电荷和空穴均在电场的作用下被扫出空间电荷区。正因为空间电荷区内不存在任何可动的电荷,其也被称为\uwave{耗尽区}(Depletion Region)。空间电荷区与耗尽区两个术语可互换使用。

\begin{Figure}[PN结的空间电荷区]
    \includegraphics[scale=0.85]{build/Chapter01A_02.fig.pdf}
\end{Figure}

在空间电荷区的边界仍然存在多子浓度的梯度,我们可以这么认为,由于浓度梯度的存在,多数载流子受到了一个“扩散力”,但同时,由于空间电荷区产生的的电场的存在,多数载流子将受到一个方向相反的“电场力”。在热平衡条件下“扩散力”和“电场力”是相互平衡的。