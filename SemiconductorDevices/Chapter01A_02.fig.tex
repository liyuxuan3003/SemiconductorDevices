\documentclass{xStandalone}

\begin{document}
\begin{tikzpicture}
    
\path 
    (0,0) coordinate (A1) 
    (12,2) coordinate (B2)
    (A1-|B2) coordinate (A2)
    (A1|-B2) coordinate (B1);

\path 
    ($(B1)!0.25!(B2)$) coordinate (M1)
    ($(B1)!0.5!(B2)$) coordinate (M2)
    ($(B1)!0.75!(B2)$) coordinate (M3)
    (M1|-A1) coordinate (Q1)
    (M2|-A1) coordinate (Q2)
    (M3|-A1) coordinate (Q3);


\fill[red!50!white]  (A1) rectangle node[black]{\Large P} (M1);
\fill[blue!50!white] (A2) rectangle node[black]{\Large N} (M3);
\fill[black!50!white] (Q1) rectangle (M3);

\draw[dashed,thick] (M1) -- (Q1);
\draw[dotted,thin] (M2) -- (Q2);
\draw[dashed,thick] (M3) -- (Q3);
\draw[ultra thick] (A1) rectangle (B2);

\foreach \y in {0.2,0.4,0.6,0.8}
{
    \foreach \x in {0.33,0.67}
    {
        \path[white]
            ($(Q1)!\x!(Q2)$) coordinate (Q)
            ($(A1)!\y!(B1)$) coordinate (C)
            (Q|-C) node {\small $-$};
    }
}

\foreach \y in {0.2,0.4,0.6,0.8}
{
    \foreach \x in {0.33,0.67}
    {
        \path[white]
            ($(Q2)!\x!(Q3)$) coordinate (Q)
            ($(A1)!\y!(B1)$) coordinate (C)
            (Q|-C) node {\small $+$};
    }
}

\node[above,align=center] at ($(M1)!0.5!(M2)$) {$N_a$负电荷};

\node[above,align=center] at ($(M2)!0.5!(M3)$) {$N_d$正电荷};

\draw[dotted] (Q1) -- ++(0,-6) coordinate(Q1');
\draw[dotted] (Q3) -- ++(0,-6) coordinate(Q3');

\draw[<->,ultra thin]
($(Q1)!0.15!(Q1')$) --
node[above] {空间电荷区}
node[below] {耗尽区}
($(Q3)!0.15!(Q3')$);

\draw[-latex]
($(Q1)!0.4!(Q1')$) coordinate (Q1'')
($(Q3)!0.4!(Q3')$) coordinate (Q3'')
($(Q3'')!0.3!(Q1'')$)
--
node[above] {电场}
($(Q3'')!0.7!(Q1'')$);

\path
($(Q1)!0.6!(Q1')$) coordinate (Q1'6)
($(Q3)!0.6!(Q3')$) coordinate (Q3'6)
($(Q1)!0.85!(Q1')$) coordinate (Q1'8)
($(Q3)!0.85!(Q3')$) coordinate (Q3'8);

\draw[latex-,align=center]
(Q1'6)
--
node[above] {空穴的扩散力}
($(Q3'6)!1.45!(Q1'6)$);

\draw[latex-,align=center]
(Q3'6)
--
node[above] {电子的扩散力}
($(Q1'6)!1.45!(Q3'6)$);

\draw[-latex,align=center]
(Q1'8)
--
node[above] {空穴的电场力}
($(Q3'8)!0.55!(Q1'8)$);

\draw[-latex,align=center]
(Q3'8)
--
node[above] {电子的电场力}
($(Q1'8)!0.55!(Q3'8)$);


% \path ($(B1)!0.5!(M1)$) node[below,white] {\small 空穴--多子};
% \path ($(B2)!0.5!(M3)$) node[below,white] {\small 电子--多子};

% \path ($(Q1)!0.5!(Q2)$) node[below] {\scriptsize 电离受主};
% \path ($(Q2)!0.5!(Q3)$) node[below] {\scriptsize 电离施主};

% \draw[black,thick,-latex] ($(M2)!1.3!(Q2)$) coordinate (O)
% ($(O)+(0.8,0)$) -- 
% node[above] {$\Emf$} 
% node[below] {\small 内建电场}
% ($(O)+(-0.8,0)$);

% \draw[<-,thick]
%     ($(M1)!1.75!(Q1)$) coordinate (Q1x)
%     ($(Q1x)+(-1.2,0)$)--
%     node[above] {空穴漂移}
%     ($(Q1x)+(+1.2,0)$);

% \draw[<-,thick]
%     ($(M3)!1.75!(Q3)$) coordinate (Q3x)
%     ($(Q3x)+(+1.2,0)$)--
%     node[above] {电子漂移}
%     ($(Q3x)+(-1.2,0)$);

% \draw[->,thick]
%     ($(Q1)!1.5!(M1)$) coordinate (M1x)
%     ($(M1x)+(-1.2,0)$)--
%     node[above] {空穴扩散}
%     ($(M1x)+(+1.2,0)$);

% \draw[->,thick]
%     ($(Q3)!1.5!(M3)$) coordinate (M3x)
%     ($(M3x)+(+1.2,0)$)--
%     node[above] {电子扩散}
%     ($(M3x)+(-1.2,0)$);

% \path ($(B1)!0.5!(B2)$) node[above] {\small 空间电荷区/耗尽区}; 

\end{tikzpicture}
\end{document}