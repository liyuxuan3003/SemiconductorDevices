\section{JFET的器件特性}
现在让我们来推导JFET的器件特性。我们或许已经注意到了JFET是上下对称的,即所谓的“双边JFET”。我们可以只考虑半个JFET来让计算更简单,即“单边HFET”,如\xref{fig:JFET的}

\begin{Figure}[JFET的简化]
    \begin{FigureSub}[双边JFET]
        \includegraphics[scale=0.85]{build/Chapter05B_01.fig.pdf}
    \end{FigureSub}\\ \vspace{0.5cm}
    \begin{FigureSub}[单边JFET]
        \includegraphics[scale=0.85]{build/Chapter05B_02.fig.pdf}
    \end{FigureSub}
\end{Figure}
以$I_{D2}$和$I_{D1}$分别表示双边JFET和单边JFET上的电流,显然
\begin{Equation}
    I_{D2}=2I_{D1}
\end{Equation}
这是因为双边JFET可以视为两个单边JFET的并联。后面的分析都对单边JFET进行。

\subsection{栅源夹断电压}
这一小节的目的是,推导$V_{GS}$为何值时,JFET沟道被夹断?
\begin{Figure}[栅源夹断电压的推导]
    \includegraphics{build/Chapter05B_03.fig.pdf}
\end{Figure}
如\xref{fig:栅源夹断电压的推导}所示,设JFET的沟道的总厚度为$a$,其中,设耗尽区占据的厚度为$h$。\setpeq{栅源夹断电压}

根据\fancyref{fml:反偏时的空间电荷区宽度},考虑到$V_{GS}$形式上为正偏电压,且$N_{a}\gg N_d$
\begin{Equation}&[1]
    h=\qty[\frac{2\epsilon_s(V_{bi}-V_{GS})}{eN_d}]^{1/2}
\end{Equation}
其中$V_{bi}$是栅--沟道PN结的内建电势差。

这里$a$是一个常量,当$h=a$时,就意味着耗尽区完全夹断了沟道,记此时的$V_{GS}$为$V_p$
\begin{Equation}&[2]
    a=\qty[\frac{2\epsilon_s(V_{bi}-V_{p})}{eN_d}]^{1/2}
\end{Equation}
这里还常引入$V_{p0}=V_{bi}-V_p$代换
\begin{Equation}&[3]
    a=\qty[\frac{2\epsilon_sV_{p0}}{eN_d}]^{1/2}
\end{Equation}
正式定义如下
\begin{BoxDefinition}[内建夹断电压]
    内建夹断电压$V_{p0}$定义为
    \begin{Equation}
        V_{p0}=V_{bi}-V_p
    \end{Equation}
    这里,$V_{bi}$是内建电压,$V_{p}$是夹断电压,$V_{p0}$是内建夹断电压。\footnote[2]{其中p代表的是夹断的英文pinch off。}
\end{BoxDefinition}

关于$V_p$和$V_{p0}$,我们可能会觉得很绕,可以通过以下方式记忆
\begin{itemize}
    \item $V_p$是夹断电压,它是一个负值,它的意义是当栅压达到$V_{GS}=V_p$的负值时,沟道夹断。
    \item $V_{p0}$是内建夹断电压,它和内建电压$V_{bi}$一样都是一个正值,且$V_{bi}<V_{p0}$,它的意义可以解释为,假如栅--沟道PN结的内建电压能到达$V_{p0}$,那么,在零栅压时耗尽区就已经挤占了整个沟道,发生夹断了。换言之,$V_{p0}=V_{bi}-V_{p}$代表夹断发生时的耗尽区电压。
\end{itemize}\setpeq{栅源夹断电压}
$V_{p0}$关于$V_p$的表达式为
\begin{Equation}
    V_{p0}=V_{bi}-V_p
\end{Equation}
$V_{p}$关于$V_{p0}$的表达式为
\begin{Equation}
    V_{p}=V_{bi}-V_{p0}
\end{Equation}
这是非常好记忆的:夹断电压$V_{p}$和内建夹断电压$V_{p0}$分别等于$V_{bi}$减去对方。

内建夹断电压$V_{p0}$是一个与总厚度$a$直接相关的量,由\xrefpeq{3}稍微做一些变换,就可以得到
\begin{BoxFormula}[内建夹断电压]
    内建夹断电压$V_{p0}$可以表示为
    \begin{Equation}
        V_{p0}=\frac{ea^2N_d}{2\epsilon_s}
    \end{Equation}
\end{BoxFormula}

现在,我们就可以明确写出栅源电压$V_{GS}$达到多少时,夹断会发生了。
\begin{BoxFormula}[JFET的栅源夹断电压]
    当栅源电压$V_{GS}$满足下式时,沟道夹断
    \begin{Equation}
        V_{GS}\leq V_p
    \end{Equation}
    即
    \begin{Equation}
        V_{GS}\leq V_{bi}-V_{p0}
    \end{Equation}
\end{BoxFormula}

\subsection{漏源饱和电压}
这一小节的目的是,推导$V_{DS}$为何值时,JFET沟道电流饱和?
\begin{Figure}[漏源夹断电压的推导]
    \includegraphics{build/Chapter05B_04.fig.pdf}
\end{Figure}
如\xref{fig:漏源夹断电压的推导}所示,此时耗尽区是倾斜的,宽度$h(x)$并不均等,源侧记为$h_1$,漏侧记为$h_2$。

在源侧,“反偏电压”就是$-V_{GS}$,故$h_1$和前面的$h$是相同的
\begin{BoxFormula}[JFET的源侧耗尽区厚度]
    JFET在源侧的耗尽区厚度$h_1$为
    \begin{Equation}
        h_1=
        \qty[\frac{2\epsilon_s(V_{bi}-V_{GS})}{eN_d}]^{1/2}
    \end{Equation}
\end{BoxFormula}
在漏侧,“反偏电压”增大至$V_{DS}-V_{GS}$,故$h_2$应改写为
\begin{BoxFormula}[JFET的源侧耗尽区厚度]
    JFET在源侧的耗尽区厚度$h_2$为
    \begin{Equation}
        h_2=
        \qty[\frac{2\epsilon_s(V_{bi}+V_{DS}-V_{GS})}{eN_d}]^{1/2}
    \end{Equation}
\end{BoxFormula}\setpeq{漏源夹断电压}
很明显,漏源电压$V_{DS}$导致的那种使电流饱和的夹断的条件是$h_2=a$,即$V_{DS}$要满足
\begin{Equation}
    a=\qty[\frac{2\epsilon_s(V_{bi}+V_{DS}-V_{GS})}{eN_d}]^{1/2}
\end{Equation}
容易解出
\begin{Equation}
    V_{DS}=\frac{ea^2N_{d}}{2\varepsilon_{s}}-V_{bi}+V_{GS}
\end{Equation}
根据\fancyref{fml:内建夹断电压}
\begin{Equation}
    V_{DS}=V_{p0}-V_{bi}+V_{GS}
\end{Equation}
根据\fancyref{def:内建夹断电压}
\begin{Equation}
    V_{DS}=V_{GS}-V_{p}
\end{Equation}
由此,我们就可以归纳出电流饱和时$V_{DS}$的条件了
\begin{BoxFormula}[漏源饱和电压]
    当漏源电压$V_{DS}$满足下式时,沟道电流饱和
    \begin{Equation}
        V_{DS}\geq V_{GS}-V_{p}
    \end{Equation}
    即
    \begin{Equation}
        V_{DS}\geq V_{GS}-V_{bi}+V_{p0}
    \end{Equation}
\end{BoxFormula}